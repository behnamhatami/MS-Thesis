
\فصل{کارهای پیشین}

در این فصل کارهای پیشین انجام‌شده روی مسئله‌ی $k$-مرکز، در سه بخش مورد بررسی قرار می‌گیرد. در بخش اول، مسئله‌ی $k$-مرکز مورد بررسی قرار می‌گیرد. در بخش دوم، حالت جویبار داده‌ی مسئله و مجموعه هسته‌های مطرح برای این مسئله مورد بررسی قرار می‌گیرد. در نهایت، در بخش سوم، مسئله‌ی $k$-مرکز‌ با داده‌های پرت مورد بررسی قرار می‌گیرد.

\قسمت{$k$-مرکز‌ در حالت ایستا}

مسئله‌ی $k$-مرکز‌ به عنوان مسئله‌ی شناخته شده در علوم کامپیوتر مطرح است. این مسئله، در واقع یک مسئله‌ی بهینه سازی است که سعی در کاهش بیش‌ترین فاصله نقاط از مرکز دسته‌ها را دارد. سختی اصلی این مسئله در انتخاب مرکز دسته‌هاست. زیرا اگر بتوانیم مرکز دسته‌ها را به درستی تشخیص دهیم، کافی است هر نقطه را به دسته‌ای که نزدیک‌ترین مرکز  را دارد، تخصیص دهیم. به وضوح چنین تخصیصی بهینه‌ترین تخصیص ممکن است. نمونه‌ای از این تخصیص را در شکل \رجوع{شکل:تخصیص‌نقاط} نشان داده شده است.

\شروع{شکل}[ht]
\centerimg{point-assignment}{10cm}
\شرح{نمونه‌ای از تخصیص نقاط‌ به ازای مراکز آبی رنگ.}
\برچسب{شکل:تخصیص‌نقاط}
\پایان{شکل}

در سال $1979$، اثبات گردید که این مسئله یک مسئله‌ی ان‌پی-سخت است \مرجع{michael1979computers}. حتی ثابت شده است که این مسئله در صفحه‌ی دو بعدی و با متریک اقلیدسی نیز‌ ان‌پی-سخت است \مرجع{megiddo1984complexity}. فراتر از این، ثابت شده است که برای مسئله‌ی $k$-مرکز  با متریک دلخواه هیچ الگوریتم تقریبی با ضریب تقریب بهتر از $2$ وجود ندارد. ایده‌ی اصلی این کران پایین، کاهش مسئله‌ی پوشش‌ رأسی، به مسئله‌ی $k$-مرکز است. برای چنین کاهشی کافی است، از روی گراف اصلی، یک گراف کامل بسازیم به طوری که معادل یال‌های گراف اصلی یال با وزن یک و به ازای بقیه‌ یال‌های ممکن که در گراف اصلی نیستند، یال با وزن $2$ قرار می‌دهیم. حال اگر الگوریتمی بتواند مسئله‌ی $k$-مرکز را با ضریب تقریب بهتر از $2$ حل نماید، آن‌گاه گراف جدید دارای یک $k$-مرکز با شعاع کم‌تر از $2$ است، اگر و تنها اگر گراف اصلی دارای یک پوشش‌ رأسی با اندازه‌ی $k$ باشد. برای متریک $L_2$ یا فضای اقلیدسی\پاورقی{Euclidean space} نیز‌‌ ثابت شده است الگوریتم تقریبی با ضریب تقریب بهتر از $1.822$ وجود ندارد \مرجع{bern1996approximation}.

گنزالز\پاورقی{Gonzalez} اولین الگوریتم تقریبی برای مسئله‌ی $k$-مرکز‌ را ارائه داده است \مرجع{han2011data}. این الگوریتم یک الگوریتم تقریبی با ضریب ۲ است و در زمان $\cO(kn)$ قابل اجراست. الگوریتم گنزالز، یک الگوریتم حریصانه\پاورقی{greedy} است. روش اجرای این الگوریتم به این گونه است که در ابتدا یک نقطه‌ی دلخواه را به عنوان مرکز‌ دسته‌ی اول در نظر می‌گیرد. سپس دورترین نقطه‌ از آن را به عنوان مرکز دسته‌ی دوم در نظر می‌گیرد. در هر مرحله، دورترین نقطه از مرکز ‌مجموعه دسته‌های موجود را به عنوان مرکز‌ دسته‌ی جدید به مجموعه مراکز‌ دسته‌ها اضافه می‌شود. با اجرای الگوریتم تا $k$ مرحله، مراکز دسته‌ها انتخاب می‌شود. حال اگر هر نقطه را به نزدیک‌ترین مرکز انتخابی تخصیص دهیم، می‌توان نشان داد که شعاع بزرگ‌ترین دسته، حداکثر دو برابر شعاع بهینه‌ برای مسئله‌ی $k$-مرکز است. فدر\پاورقی{Feder} و سایرین، زمان اجرای الگوریتم گنزالز را به $\cO(n \log{k})$ برای هر $L_p$-متریک بهبود بخشیدند. نمونه‌ای از اجرای الگوریتم گنزالز، در شکل \رجوع{شکل:گنزالز} نشان داده شده است.

\شروع{شکل}[ht]
\centerimg{Gonzalez}{10cm}
\شرح{نمونه‌ای از حل مسئله‌ی ۳-مرکز با الگوریتم گنزالز}
\برچسب{شکل:گنزالز}
\پایان{شکل}

تا به اینجا  تنها برای $k$ و $d$ دلخواه صحبت شد. برای حالاتی که در مسئله‌ی $k$ -مرکز، $k$ تعداد دسته‌ها و $d$ ابعاد فضا ثابت باشند، آگاروال\پاورقی{Agarwal} و سایرین الگوریتمی دقیق با زمان اجرای $n^{\cO(k^{1 - \frac{1}{d}})}$ برای مسئله‌ $k$-مرکز در فضای $L_p$-متریک ارائه داده‌اند \مرجع{agarwal2002exact}. قابل توجه است که اگر $d$ ثابت نباشد، مسئله‌ی $k$-مرکز حتی برای متریک اقلیدسی($L_2$-متریک) با تعداد دسته‌ی ثابت $k \geq 2$، ان‌پی-سخت است \مرجع{megiddo1990complexity}.

تا به اینجا، نتایج کلی برای $k$ و $d$ را مورد بررسی قرار دادیم. برای بعضی از مقادیر خاص از $k$ و $d$ الگوریتم‌های بهینه‌تری وجود دارد.  به طور مثال، برای مسئله‌ی $1$-مرکز در فضای اقلیدسی با ابعاد ثابت، الگوریتم خطی با زمان اجرای $\cO((d+1)! n)$ وجود دارد \مرجع{chazelle1996linear}. الگوریتم ارائه شده بر پایه‌ی دو نکته‌ی اساسی بنا شده است. اول اینکه دایره‌ی بهینه را می‌توان با حداکثر $d+1$ نقطه‌ی واقع در پوسته‌ی کره‌ی بهینه مشخص نمود و دوم اینکه اگر نقاط ورودی را با ترتیبی تصادفی پیمایش‌ کنیم احتمال اینکه نقطه‌ی پیمایش شده جزء نقاط مرزی باشد $\cO(\frac{d}{n})$ است که با توجه به ثابت بودن $d$ این احتمال کوچک می‌باشد. 

در صفحه اقلیدسی‌($L_2$-متریک) برای مسئله‌ی $2$-مرکز, بهترین الگوریتم را چن\پاورقی{Chan}   با زمان اجرای $\cO(c \log^2{n} \log^2{\log{n}})$ و حافظه‌ی $\cO(n)$ ارائه داده است \مرجع{chan1999more}. برای فصای سه‌بعدی اقلیدسی نیز آگاروال و سایرین، الگوریتمی با متوسط‌ زمان اجرای $\cO(n^3\log^8{n})$ ارائه داده است \مرجع{agarwal20132}.

\قسمت{$k$-مرکز‌ در حالت جویبار داده}

در مدل جویبار داده، مشکل اصلی عدم امکان نگه‌داری تمام داده‌ها در حافظه است و باید سعی شود تنها داده‌هایی که ممکن است در ادامه مورد نیاز باشد را، نگه‌داریم. یکی از راه‌های رایج برای این کار نگه‌داری مجموعه‌ای از‌ نقاط (نه لزوما زیرمجموعه‌ای از نقاط‌ ورودی) به عنوان نماینده‌ی نقاط‌ به طوری که جواب مسئله‌ی $k$-مرکز برای آن‌ها منطبق‌ با جواب مسئله‌ی $k$-مرکز‌ برای نقاط‌ اصلی باشد (با تقریب قابل قبولی). به چنین مجموعه‌ای مجموعه‌ی هسته‌ی نقاط‌ گفته می‌شود. 

بهترین مجموعه‌ هسته‌ای که برای مسئله‌ی $k$-مرکز‌ ارائه شده است، روش ارائه‌شده به وسیله‌ی ضرابی‌زاده برای نگه‌داری یک $\epsilon$-هسته با حافظه‌ی $\cO(\frac{k}{\epsilon^d})$ برای $L_p$-متریک‌ها ارائه داده است \مرجع{zarrabi2008core}. در روش ارائه شده، از چند ایده‌ی ترکیبی استفاده شده است. در ابتدا، الگوریتم با استفاده یک الگوریتم تقریبی یک تقریب از جواب بهینه به دست می‌آورد. به طور مثال با استفاده از‌ الگوریتم گنزالز، یک دو تقریب از شعاع بهینه به علاوه‌ی مرکز‌ دسته‌های پیدا شده را به ما بدهد. حال کافی است که با طول شعاع الگوریتم دو تقریب حول هر مرکز‌، یک توری با شبکه‌بندی $\cO(\frac{1}{\epsilon})$ در هر بعد تشکیل دهیم و چون هر نقطه در حداقل یکی از توری‌ها قرار می‌گیرد، می‌توانیم با حداکثر $\epsilon$ تقریب در جواب نهایی نقاط را به نقاط شبکه‌بندی توری گرد نمود. با این کار، دیگر ما نیازی به نگهداری تمام نقاط ورودی نداشته و تنها نیاز به نگهداری نقاط شبکه‌بندی توری داریم. با این‌ روش می‌توان به یک $\epsilon$-هسته برای مسئله‌ی $k$-مرکز رسید. نکته‌ی اساسی برای سازگار سازی روش ارائه‌شده با مدل جویبار داده‌ی تک‌گذره استفاده از روش دوبرارسازی رایج در الگوریتم‌های جویبار داده است. نمونه‌ای از اجرای الگوریتم ضرابی‌زاده را در شکل \رجوع{شکل:توری} نشان داده شده است. برای دیدن اثبات‌ها و توضیح بیش‌تر در مورد روش ارائه شده می‌توانید به مرجع \مرجع{zarrabi2008core} مراجعه کنید.

\شروع{شکل}[ht]
\centerimg{zarrabi-zade-mesh-coreset}{10cm}
\شرح{نمونه‌ای از توری‌بندی الگوریتم ضرابی‌زاده (نقاط آبی، مراکز به دست آمده از الگوریتم تقریبی است). پس از توری‌بندی کافی است برای هر کدام از خانه‌های شبکه‌بندی، تنها یک نقطه را در نظر بگیریم.}
\برچسب{شکل:توری}
\پایان{شکل}

از جمله مشکلات وارده به الگوریتم ضرابی زاده، وابستگی سایز مجموعه‌ی هسته به ابعاد فضا است. بنابراین هسته‌ی ارائه شده به وسیله‌ی ضرابی‌زاده را نمی‌توان برای ابعاد بالا مورد استفاده قرار داد. از طرفی، حساب کردن جواب از روی هسته در زمان چند‌جمله‌ای قابل انجام نیست. با توجه با موارد گفته شده، برای قابل استفاده شدن الگوریتم‌ها برای ابعاد بالا، الگوریتم‌هایی ارائه می‌شود که ضریب تقریب بدتری دارند، اما میزان حافظه‌ی مصرفی و سایز مجموعه هسته‌‌ی آن‌ها چندجمله‌ای بر اساس $d$ و $k$ و $\log{n}$ باشد. به چنین الگوریتم‌هایی الگوریتم‌های جویبار داده برای ابعاد بالا گفته می‌شود. 

اولین الگوریتم ارائه شده برای ابعاد بالا،  الگوریتمی با ضریب تقریب $8$ و حافظه‌ی مصرفی $\cO(dk)$ است \مرجع{charikar1997incremental}. پس از آن، گوها\پاورقی{Guha}، به طور موازی با مک‌کاتن\پاورقی{McCutchen} و سایرین، الگوریتمی با ضریب تقریب $(2+\epsilon)$ با حافظه‌ی $\cO(\frac{dk}{\epsilon} \log{\frac{1}{\epsilon}})$ برای مسئله‌ی $k$-مرکز در هر فضای متریکی ارائه دادند\مرجع{mccutchen2008streaming, guha2009tight}. در سال ۲۰۱۴، اهن\پاورقی{Ahn} و سایرین, الگوریتمی با همین ضریب تقریب و حافظه‌ی $\cO((k+3)!2^k\frac{d}{\epsilon})$ ارائه داده‌اند که برای $k$‌های ثابت، حافظه را از مرتبه‌ی $\cO(\log{\frac{1}{\epsilon}})$ کاهش می‌دهد\مرجع{ahn2014computing}. 

تا به اینجا ما به بررسی مسئله‌ی $k$-مرکز‌ در حالت جویبار داده بدون محدودیت خاصی پرداختیم. برای حالت‌های خاص $k$ و متریک اقلیدسی، به خصوص $1$ و $2$، مسئله‌ی $k$-مرکز مورد توجه زیادی قرار گرفته است و راه‌حل‌های بهینه‌تری نسبت به حالت کلی برای آن‌ها پیشنهاد شده است. به طور مثال، می‌توان یک هسته با اندازه‌ی‌ $\cO(\frac{1}{\epsilon^{\frac{d-1}{2}}})$, با استفاده از نقاط حدی\پاورقی{extreme points} در تعداد مناسبی جهت، به صورت جویبار داده ساخت. 

مشکل عمده‌ی مجموعه هسته‌ی ارائه‌شده، وابستگی حافظه‌ی مصرفی آن به $d$ است. ضرابی‌زاده و سایرین\مرجع{zarrabi2006simple} برای ابعاد دلخواه و متریک اقلیدسی، الگوریتمی با ضریب تقریب $1.5$ و حافظه‌ی مصرفی $\cO(d)$ ارائه دادند. در واقع در الگوریتم آن‌ها، در هر لحظه تنها یک مرکز و یک شعاع را نگه می‌دارد، که کم‌ترین حافظه‌ی ممکن برای مسئله‌ی $1$-مرکز است. الگوریتم ارائه شده نقطه‌ی اول را به عنوان مرکز با شعاع صفر در نظر می‌گیرد. با فرا رسیدن هر نقطه‌ی جدید، اگر نقطه‌ی مد نظر، داخل دایره‌ی فعلی بیفتد که بدون هیچ‌ تغییری ادامه می‌دهیم و در صورتی که بیرون دایره فعلی بیفتد، آن را با کوچک‌ترین دایره‌ای که نقطه‌ی جدید به علاوه‌ی دایره‌ی قبلی را به طور کامل می‌پوشاند، جایگزین می‌کنیم. به وضوح در هر لحظه دایره‌ی ساخته شده تمام نقاط را می‌پوشاند. از طرفی ثابت می‌شود شعاع دایره‌ در هر لحظه، حداکثر $1.5$-برابر شعاع دایره‌ی $1$-مرکز بهینه است. نمونه‌ای از‌ اجرای الگوریتم را برروی چهار نقطه می‌توان در شکل \رجوع{شکل:یک‌مرکز‌ضرابی‌زاده} دید. برای اثبات کامل‌تر می‌توانید به مرجع \مرجع{zarrabi2006simple} مراجعه کنید. 

\شروع{شکل}[ht]
\centerimg{zarrabi-approximation-1-center}{10cm}
\شرح{نمونه‌ای از‌ اجرای الگوریتم ضرابی‌زاده بر روی چهار نقطه $P_1 \cdots P_4$ که به ترتیب اندیس در جویبار داده فرا می‌رسند و دایره‌های $B_0 \cdots B_2$ دایره‌هایی که الگوریتم به ترتیب نگه می‌دارد.}
\برچسب{شکل:یک‌مرکز‌ضرابی‌زاده}
\پایان{شکل}


در ادامه، آگاروال و سایرین\مرجع{agarwal2010streaming} الگوریتمی تقریبی با حافظه‌ی مصرفی $\cO(d)$ ارائه دادند. در الگوریتم ارائه شده، ضریب تقریب، برابر $\frac{1 + \sqrt{3}}{2}$  تخمین زده شد، اما با تحلیل دقیق‌تری که چن و سایرین \مرجع{chan2014streaming} انجام دادند، مشخص شد، همان الگوریتم دارای ضریب تقریب $1.22$ است.

الگوریتم آن‌ها از الگوریتم کلارکسون و سایرین\مرجع{badoiu2003smaller} استفاده می‌کند. الگوریتم کلارکسون، الگوریتمی کاملا مشابه الگوریتم گنزالز است و به این گونه عمل می‌کند که در ابتدا یک نقطه دلخواه را به عنوان نقطه‌ی اول انتخاب می‌کند، سپس دورترین نقطه از‌ نقطه‌ی اول را به عنوان دومین نقطه انتخاب می‌کند. ازین به بعد در هر مرحله، نقطه‌ای که از نقاط انتخاب شده‌ی قبلی بیش‌ترین فاصله را دارد به عنوان نقطه‌ی جدید انتخاب می‌کند. اگر این الگوریتم را تا $\cO(\frac{1}{\epsilon})$ مرحله ادامه بدهیم، به مجموعه‌ای با اندازه‌ی $\cO(\frac{1}{\epsilon})$ خواهیم رسید که  کلارکسون و سایرین اثبات کرده‌اند که یک $\epsilon$-هسته برای مسئله‌ی $1$-مرکز‌ است.

الگوریتم آگاروال به این گونه عمل می‌کند که اولین نقطه‌ی جویبار داده را به عنوان تنها نقطه‌ی مجموعه‌ی $K_1$ در نظر می‌گیرد. حال تا وقتی که نقاطی که فرا می‌رسند داخل $(1 + \epsilon)Meb(K_1)$ قرار بگیرند، ادامه می‌دهد. اولین نقطه‌ی که در شرایط‌ ذکر شده صدق نمی‌کند را $p_2$ بنامید. حال الگوریتم کلارکسون را بر روی $K_1 \cup \set{p_2}$ اجرا کرده و مجموع هسته‌ی به دست آمده را $K_2$ بنامید. با ادامه‌ی این روند، الگوریتم، دنباله‌ای از مجموعه هسته $\kappa = \{ K_1, \cdots, K_u \}$ نگه می‌دارد و زمانی که نقطه‌ی $p_{u+1}$ پیدا می‌شود که در هیچ کدام از $(1 + \epsilon)Meb(K_j)$ به ازای $1 \leq j \leq u$ نباشد، الگوریتم کلارکسون را برای $\cup_{j = 1}^{u} K_j \cup \set{p_{u+1}}$ اجرا نموده و مجموعه هسته‌ی به دست آمده را $K_{u+1}$ بنامید. با توجه به نحوه‌ی ساخته شدن $K_i$ها، به راحتی می‌توان نشان داد رابطه‌ی زیر برقرار است:
$$P \subset \cup_{i=1}^u (1+\epsilon)Meb(K_i)$$
حال در نهایت برای به دست آوردن جواب نهایی کافی است کوچک‌ترین دایره‌ای که $\cup_{i=1}^u (1+\epsilon)Meb(K_i)$ را می‌پوشاند را به عنوان جواب دهیم. آگاروال و سایرین ثابت کرده‌اند که دایره‌ی نهایی دارای شعاع حداکثر $1.22$ برابر شعاع بهینه است. برای مشاهده‌ جزئیات بیش‌تر، به \مرجع{agarwal2010streaming} مراجع کنید.

آگاروال نه تنها الگوریتمی ارائه داد که در نهایت، ثابت شد حداکثر جوابی با ضریب تقریب $1.22$ برابر جواب بهینه می‌دهد، بلکه نشان داد، که با حافظه‌ی چندجمله‌ای بر اساس $\log{n}$ و $d$ نمی‌توان الگوریتمی ارائه داد که ضریب تقریب بهتر از $\frac{1 + \sqrt{2}}{2}$ داشته باشد.

\قضیه{هر الگوریتم تحت مدل جویبار داده که یک $\alpha$-تقریب برای مسئله‌‌ی $1$-مرکز برای مجموعه‌ی $S$ شامل $n$ نقطه در فضای $\IR^d$ نگه‌ می‌دارد، برای
$\alpha \leq \frac{1 + \sqrt{2}}{2} (1 - \frac{2}{d^{\frac{1}{3}}})$
 با احتمال حداقل $\frac{2}{3}$ نیاز به 
$\Omega(\min \{n, e^{d^{\frac{1}{3}}} \})$
حافظه مصرف می‌کند.}


\شروع{اثبات}

ایده‌ی اصلی اثبات بر اساس قضیه‌ی معروف آلیس و باب\پاورقی{alice and bob} در نظریه انتقال اطلاعات بنا شده است. برای خواندن اثبات این قضیه می‌توانید به مرجع \مرجع{agarwal2010streaming} مراجعه کنید.

\پایان{اثبات}

علاوه بر مسئله‌ی $1$-مرکز، مسئله‌ی دو مرکز نیز در سال‌های اخیر مورد توجه قرار گرفته است و بهبود‌هایی نیز برای آن مسئله ارائه شده است. آهن و سایرین \مرجع{kim2014improved} در سال $2014$، اولین الگوریم با ضریب تقریب کم‌تر از دو را برای مسئله‌ی $2$-مرکز در فضای اقلیدسی ارائه دادند. این الگوریتم تقریبا پایه‌ی کار این پایان‌نامه برای حالت‌های مختلف است. به همین منظور، تعدادی از لم‌های داخل این الگوریتم که در آینده استفاده می‌شود این‌جا توضیح داده می‌شود.


\شروع{لم}
\برچسب{لم:ahn-segment}
فرض کنید $B$ یک کره‌ی واحد با مرکز $c$ در فضای اقلیدسی $\IR^d$ باشد. هر پاره‌خط $pq$ به طول حداقل $1.2$ که به طور کامل داخل $B$ قرار دارد، دایره‌ی $B'(c, 0.8)$ را قطع می‌کند.
\پایان{لم}

\شروع{شکل}[ht]
\centerimg{ahn-segment-lemma}{10cm}
\شرح{اثبات لم \رجوع{لم:ahn-segment}}
\برچسب{شکل:ahn-segment}
\پایان{شکل}

\شروع{اثبات}

صفحه‌ی گذرنده از پاره‌خط و مرکز دایره را نظر بگیرید. ادامه اثبات تنها به به همین صفحه محدود می‌شود، بنابراین نیاز به در نظر گرفتن ابعاد بزرگ‌تر از $2$ نیست. همان‌طور که در شکل \رجوع{شکل:ahn-segment} مشخص شده است، پای عمود از مرکز دایره بر پاره‌خط $pq$ را $h$ بنامید. بدون کم شدن از کلیت مسئله فرض می‌کنیم $\len{hp} \leq \len{hq}$. بنابراین داریم:
$$0.6 = \frac{1.2}{2} \leq \len{hq}$$
از‌ طرفی چون پاره‌خط $pq$ به طور کامل داخل کره‌ی واحد قرار گرفته است، بنابراین تمام نقاط آن، شامل دو سر آن، از مرکز دایره، فاصله‌ی حداکثر ۱ دارند. بنابراین طبق رابطه‌ی فیثاغورث، داریم:
$$\len{hc} = \sqrt{\len{qc}^2 - \len{qh}} \leq \sqrt{1 - 0.6^2} = 0.8$$
بنابراین نقطه‌ی $h$ داخل دایره‌ی $B'$ قرار می‌گیرد. از طرفی چون، $1 \leq \len{pq}$ بنابراین، $h$ داخل پاره‌خط قرار دارد و در نتیجه پاره‌خط $B'$ با پاره‌خط $pq$ تقاطع دارد. 

\پایان{اثبات}

لم بالا در واقع نشان می‌دهد اگر در طول الگوریتم بتوانیم دو نقطه‌ی دور نسبت به هم (حداقل $1.2$ برابر شعاع بهینه) از یکی از دو دایره‌ی بهینه را بیابیم، این پاره‌خط از مرکز دایره‌ی بهینه فاصله‌ی کمی‌ (حداکثر $0.8$ شعاع بهینه) را دارد.

\شروع{لم}
\برچسب{لم:ahn-segment2}
فرض کنید $B$ دایره‌ای به مرکز $c$ و شعاع واحد در $\IR^d$ باشد. پاره‌خط دلخواه $pq$ با طول حداقل $1.2$ که به طور کامل داخل $B$ قرار دارد را در نظر بگیرید.هر نقطه‌ی $x$ از پاره‌خط $pq$ که از دو سر آن حداقل $0.6$ فاصله داشته باشد، داخل دایره‌ی $B'(c, 0.8)$ قرار می‌گیرد.

\پایان{لم}

\شروع{شکل}[ht]
\centerimg{ahn-segment2-lemma}{10cm}
\شرح{اثبات لم \رجوع{لم:ahn-segment2}}
\برچسب{شکل:ahn-segment2}
\پایان{شکل}

\شروع{اثبات}

اثبات این لم نیز کاملا مشابه لم \رجوع{لم:ahn-segment} است. بدون کم شدن از کلیت  مسئله همان‌طور که در شکل \رجوع{شکل:ahn-segment2} مشخص شده است، فرض کنید زاویه‌ی $\angle{pxc}$ بزرگ‌تر مساوی $90$ درجه است. در نتیجه داریم:
$$\sqrt{\len{px}^2 + \len{xc}^2} \leq \len{pc} \leq 1$$
از طرفی طبق فرض مسئله داریم:
$$0.6 \leq \len{px} \implies \len{xc} \leq \sqrt{1 - 0.6^2} = 0.8$$

\پایان{اثبات}

الگوریتم آهن، با استفاده از دو لم بالا و تقسیم مسئله به دو حالتی که دو دایره‌ی بهینه بیش از $2r^*$ یا کم‌تر از $2r^*$ فاصله داشته باشد، دو الگوریتم کاملا جدا ارائه می‌دهد که به طور موازی اجرا می‌گردند. این دو الگوریتم، به طور موازی اجرا می‌شوند و در هر لحظه جوابی درست را به عنوان جواب نهایی ارائه می‌دهند و کافی است برای جواب نهایی بین دو شعاعی که به عنوان جواب خود می‌دهند، شعاع کم‌تر را به عنوان جواب الگوریتم بدهیم. به علت تشابه این الگوریتم، با یکی از الگوریتم‌های فصل کار‌های جدید، از تکرار آن خودداری کرده و به تفصیل در فصل نتایج جدید شرح داده می‌شود.

\قسمت{$k$-مرکز‌ با داده‌های پرت}
