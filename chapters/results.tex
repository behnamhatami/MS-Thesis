
\فصل{نتایج جدید}

در این فصل نتایج جدید به‌دست‌آمده در پایان‌نامه توضیح داده می‌شود. این فصل در سه بخش تهیه شده است. بخش اول به بیان مقدمات و نمادگذاری‌های مورد نیاز برای بخش‌های بعدی می‌پردازد. در بخش دوم، راه‌حل‌های ارائه‌شده برای مسئله‌ی $1$-مرکز با $z$ داده‌ی پرت در حالت جویبار داده مورد بررسی قرار می‌گیرد. این بخش به سه زیربخش تقسیم می‌شود. 

در زیربخش اول، دو الگوریتم جدید ارائه می‌شود. الگوریتم اول، با مصرف حافظه‌ی $\cO(z^2d)$، جوابی با ضریب تقریب $2$ ارائه می‌دهد که حافظه‌ی مصرفی الگوریتم ضرابی‌زاده و سایرین \مرجع{zarrabi2009streaming} را در صورتی که ابعاد فضا بیش‌تر از $z$ باشد بهبود می‌بخشد. از طرفی الگوریتم ارائه شده بسیار ساده‌تر از الگوریتم ضرابی‌زاده است. الگوریتم دوم یک الگوریتم با ضریب تقریب $1.8 + \epsilon$ و حافظه‌ی مصرفی $\cO(\frac{z^2}{\epsilon})$ است.

در زیر بخش دوم، به بررسی مسئله‌ی $1$-مرکز پوشاننده بدون داده‌ی پرت می‌پردازیم، به‌طوری‌که نه تنها می‌خواهیم تمام نقاط ورودی پوشیده شود، بلکه می‌خواهیم کل توپ بهینه نیز به طور کامل پوشیده شود. برای این مسئله، یک الگوریتم با ضریب تقریب $1.7$ ارائه می‌شود. در زیربخش سوم، با استفاده از زیربخش‌های قبلی، الگوریتمی با ضریب تقریب $1.7$ برای مسئله‌ی $1$-مرکز با $z$ داده‌ی پرت ارائه می‌دهیم(برای حالتی که $z$ ثابت است). 

در بخش سوم، مسئله‌ی $2$-مرکز با $z$-داده‌ی پرت مورد بررسی قرار می‌گیرد. ایده‌ی اصلی این بخش، تقسیم‌بندی مسئله به دو حالت است که حاصل، دو الگوریتم متفاوت می‌شود که به صورت موازی اجرا می‌شوند. هر دوی این الگوریم‌ها ضریب تقریب $1.8 + \epsilon$ دارند و حافظه‌ی مصرفی در کل برابر است با 
$\cO(dz^2 (d^2 + \frac{z}{\epsilon}))$.
این اولین الگوریتم ارائه‌شده‌ بعد از الگوریتمی که با ضریب تقریب $4+\epsilon$ برای $k$ کلی ارائه شده است، می‌باشد که بهبود قابل توجهی محسوب می‌شود.

\قسمت{نماد‌گذاری‌ها و تعاریف اولیه}

در این قسمت، تعدادی نمادگذاری که در بخش‌های آتی مورد استفاده قرار می‌گیرند، بیان می‌شود. علاوه بر این، تعدادی از مفاهیم و تعاریف رایج که در بخش‌های آتی به تکرار مورد استفاده قرار می‌گیرند نیز در این فصل مورد بررسی قرار می‌گیرد.


\شروع{شکل}[ht]
\centerimg{delta-definition}{10cm}
\شرح{تعریف فاصله‌ی دو توپ دلخواه}
\برچسب{شکل:دلتا}
\پایان{شکل}


\شروع{فقرات}

\فقره{در طول متن، برای مشخص کردن یک توپ از نماد $B(c, r)$ استفاده می‌کنیم که $c$ مرکز توپ و $r$ شعاع آن را مشخص می‌کند. هر جا خواستیم به شعاع توپی ارجاع دهیم از نماد $r(B)$ و هر گاه خواستیم به مرکز یک توپ اشاره کنیم از نماد $c(B)$ استفاده می‌کنیم.} 
\فقره{به ازای هر دو نقطه‌ی دلخواه $p$ و $q$ در فضا، فاصله‌ی $p$ و $q$ را با $\len{pq}$ نشان می‌دهیم.}
\فقره{همان‌طور که در شکل ~\رجوع{شکل:دلتا} نشان داده شده است، دو توپ دلخواه $B(c, r)$ و $B'(c', r')$ را در نظر بگیرید. فاصله‌ی دو توپ $B$ و $B'$ مطابق‌ زیر تعریف می‌شود:
$$\delta(B, B') = \max \{ 0, \len{cc'} - r - r'\}$$
}
\فقره{دو توپ دلخواه $B(c, r)$ و $B'(c', r')$ را $\alpha$-تفکیک‌شده گوییم اگر داشته باشیم:
$$\delta(B, B') \geq \alpha . \max \{ r(B), r(B') \}$$
 }
 
 \فقره{زمانی که می‌خواهیم در مورد توپ بهینه‌ی $1$-مرکز صحبت کنیم از $B^*(c^*, r^*)$ یا $MEB(c^*, r^*)$ استفاده می‌کنیم. برای مسئله‌ی $2$-مرکز نیز از $B_1^*(c_1^*, r^*)$ و $B_2^*(c_2^*, r^*)$ برای نشان دادن دو دایره‌ی بهینه‌ مسئله‌ی $2$-مرکز برای مجموعه‌ای از نقاط استفاده می‌کنیم.}
 
 \فقره{مجموعه‌ نقاط‌ $P$ داده شده است. $k$-دورترین نقطه از $p \in P$ نقطه‌ای از $P$ است که فاصله‌اش از نقطه‌ی $p$، $k$-امین بزرگ‌ترین فاصله را در بین تمام نقاط‌ $P$ داراست.}
\پایان{فقرات}

علاوه بر نمادگذاری‌های بالا, در بخش‌های بعدی، بعضی از تعاریف رایج در هندسه‌ی محاسباتی مورد استفاده قرار می‌گیرد. فرض کنید مجموعه‌ی $n$-عضوی $P$ از نقاط در $\IR^d$ داده شده است. نقطه‌ی $c \in \IR^d$ را نقطه‌ی مرکزی\پاورقی{Centerpoint} مجموعه‌ی $P$ می‌گویند، اگر هر نیم‌صفحه‌ی\پاورقی{Half Space} شامل $c$، حداقل شامل $\ceil{\frac{n}{d+1}}$ نقطه از نقاط $P$ باشد. مرجع ~\مرجع{danzer1963helly} ثابت کرده است که هر مجموعه‌ی متناهی از نقاط در فضای $d$-بعدی، دارای یک نقطه‌ی مرکزی است. مشاهده‌ی زیر نتیجه‌ی مستقیم این گزاره است.

\شروع{مشاهده}
\برچسب{مشاهده: نقطه‌ی مرکزی}
مجموعه‌ی $P$ با $k(d+1)$ نقطه در فضای $d$-بعدی داده شده است. هر شکل محدب که شامل نقطه‌ی مرکزی $P$ نباشد، حداقل $k$ نقطه‌ از $P$ را نیز نمی‌پوشاند.

\پایان{مشاهده}

در این پایان‌نامه، فرض می‌کنیم ذخیره‌ی هر بعد از یک نقطه‌ حافظه‌ی ثابتی مصرف می‌کند. در نتیجه، ذخیره‌سازی یک نقطه در فضای $d$-بعدی، $\cO(d)$ حافظه مصرف می‌کند و عملیات عادی بر روی نقاط نیز‌ $\cO(d)$ زمان می‌برد.

\قسمت{مسئله‌ی $1$-مرکز در حالت جویبار داده}

در این بخش به بررسی گونه‌های مختلفی از مسئله‌ی $1$-مرکز در حالت جویبار داده می‌پردازیم. مباحث این بخش، به صورت سه زیربخش دسته‌بندی شده است. در زیربخش اول مسئله‌ی $1$-مرکز با داده‌های پرت، مورد بررسی قرار می‌گیرد. در زیرقسمت دوم مسئله‌ی $1$-مرکز پوشاننده مورد بررسی قرار می‌گیرد و در نهایت در بخش سوم، با استفاده از نتایج دو بخش قبلی، مسئله‌ی $1$-مرکز با تعداد ثابتی داده‌ی پرت، مورد بررسی قرار می‌گیرد. در هر سه بخش‌، الگوریتم‌های قبلی از جنبه‌ یا جنبه‌هایی بهبود داده شده‌اند. مهم‌ترین معیار‌های مطرح، ضریب تقریب و حافظه‌ی مصرفی است که در هر الگوریتم به دقت محاسبه شده‌ و با کارهای قبلی مقایسه می‌شوند.

\زیرقسمت{مسئله‌ی $1$-مرکز با داده‌های پرت در حالت جویبار داده}

در این زیربخش، دو الگوریتم کاملا متفاوت برای مسئله‌ی $1$-مرکز‌ ارائه می‌شود که نسبت به الگوریتم‌های موجود ساده‌تر هستند و حافظه‌ی مصرفی کم‌تری دارند. از طرفی دیگر، دارای ویژگی‌هایی هستند که با استفاده از آن‌ها، در فصول بعدی، الگوریتم تفریبی برای مسئله‌ی $2$-مرکز ارائه می‌شود.

\زیرزیرقسمت{الگوریتم تقریبی با ضریب تقریب $2$}

در این قسمت، یک الگوریتم ساده‌ی جویبارداده با ضریب تقریب $2$ برای مسئله‌ی $1$-مرکز با داده‌ی پرت ارائه می‌شود. در این الگوریتم، از ایده‌ی موازی‌سازی\پاورقی{Parallelization} استفاده می‌شود که به وفور در بخش‌های آتی مورد استفاده قرار می‌گیرد. در الگوریتم ~\رجوع{الگوریتم: الگوریتم با ضریب تقریب $2$ برای مسئله‌ی $1$-مرکز با $z$ داده‌ی پرت} شبه‌کد\پاورقی{Pseudocode} الگوریتم ارائه‌شده آمده است. الگوریتم، جویبارداده‌ی $P$ و $z$ تعداد داده‌های پرت را از ورودی دریافت می‌کند. همان‌طور که می‌بینید الگوریتم فرض کرده است که نقطه‌ی اول جویبار داده، داده‌ی پرت نیست. در ادامه نشان خواهیم داد چگونه چنین فرضی را حذف نماییم. در نهایت الگوریتم توپ $B$ را بر می‌گرداند که همه‌ی نقاط $P$ به غیر از حداکثر $z$ نقطه را می‌پوشاند (دقیقا $z$ دورترین نقطه‌ از نقطه‌ی اول را نمی‌پوشاند).

\شروع{الگوریتم}{الگوریتم با ضریب تقریب $2$ برای مسئله‌ی $1$-مرکز با $z$ داده‌ی پرت}
\دستور{$c$ را اولین نقطه از جویبارداده‌ی $P$ قرار بده.}
\دستور{توپ $B(c, 0)$ را در نظر بگیر.}
\دستور{حافظه‌ی میان‌گیر خالی $Q$ را در نظر بگیر.}
\به‌ازای{هر $p$ در مجموعه‌ی $P$}
\اگر{$p \not \in B$}
\دستور{$p$ را به $Q$ اضافه کن.}
\اگر{$|Q| = z + 1$}
\دستور{$q$ را نزدیک‌ترین نقطه‌ی $Q$ به مرکز $c$ در نظر بگیر.}
\دستور{$q$ را از $Q$ حذف کن.}
\دستور{$B$ را با توپ $B(c, \len{cq})$ جایگزین کن.}
\پایان‌اگر
\پایان‌اگر
\پایان‌به‌ازای{}
\دستور{$B$ را برگردان}
\پایان{الگوریتم}

\شروع{قضیه}
\برچسب{قضیه: $1$-مرکز $2$-تقریب}
الگوریتم  ~\رجوع{الگوریتم: الگوریتم با ضریب تقریب $2$ برای مسئله‌ی $1$-مرکز با $z$ داده‌ی پرت}  با فرض این‌که نقطه‌ی اول جویبار داده، داده‌ی پرت نیست یک الگوریتم تقریبی با ضریب تقریب ۲ برای مسئله‌ی $1$-مرکز با $z$ داده‌ی پرت است.
\شروع{اثبات}

فرض کنید $B^*(c^*, r^*)$ توپ جواب بهینه باشد و $c$ نقطه‌ی دلخواهی از جویبار داده‌ی $P$ است که در جواب بهینه قرار دارد و جزء نقاط پرت نیست. بنابراین $c$ داخل $B^*$ قرار دارد. به ازای هر نقطه‌ی دلخواه $p \in B^*$ داریم:
$$\len{cp} \leq \len{cc^*} + \len{c^*p} \leq 2r^*$$
از طرفی، از بین $z+1$ دورترین نقطه از $c$، حداقل یک نقطه به نام $q$ وجود دارد که در جواب بهینه داده‌ی پرت نیست و در نتیجه داخل $B^*$ قرار دارد (به شکل ~\رجوع{شکل: $1$-مرکز $2$-تقریب} نگاه کنید). با توجه به گزاره‌ی گفته‌شده، $\len{cq} \leq 2r^*$ است. از طرفی چون شعاع جواب الگوریتم ~\رجوع{الگوریتم: الگوریتم با ضریب تقریب $2$ برای مسئله‌ی $1$-مرکز با $z$ داده‌ی پرت} به اندازه‌ی فاصله‌ی $c$ از‌ $z+1$-دورترین نقطه‌ از $c$ است، بنابراین شعاع جواب الگوریتم نیز کم‌تر مساوی $2r^*$ است و بنابراین الگوریتم ~\رجوع{الگوریتم: الگوریتم با ضریب تقریب $2$ برای مسئله‌ی $1$-مرکز با $z$ داده‌ی پرت} یک الگوریتم $2$-تقریب برای مسئله‌ی $1$-مرکز با $z$ داده‌ی پرت است.

\پایان{اثبات}

\پایان{قضیه}


\شروع{شکل}[ht]
\centerimg{2-approx-1-center}{10cm}
\شرح{اثبات قضیه‌ی ~\رجوع{قضیه: $1$-مرکز $2$-تقریب}}
\برچسب{شکل: $1$-مرکز $2$-تقریب}
\پایان{شکل}

الگوریتم ~\رجوع{الگوریتم: الگوریتم با ضریب تقریب $2$ برای مسئله‌ی $1$-مرکز با $z$ داده‌ی پرت} به طور ضمنی فرض کرده است که نقطه‌ی اول، در جواب بهینه نقطه‌ی پرت نیست. برای حذف چنین فرضی، $z+1$ نمونه از الگوریتم ~\رجوع{الگوریتم: الگوریتم با ضریب تقریب $2$ برای مسئله‌ی $1$-مرکز با $z$ داده‌ی پرت} به طور موازی اجرا می‌گردد به‌طوری‌که در هر کدام، یکی از $z+1$ نقطه‌ی اول به آن به عنوان نقطه‌ی اول جویبار داده به الگوریتم داده می‌شود و بقیه نقاط در ادامه می‌آید. به وضوح، در بین $z+1$ نقطه‌ی اول، حتما یک نقطه‌ وجود دارد که در جواب بهینه داده‌ی پرت نیست. بنابراین جواب آن نمونه از الگوریتم، یک $2$-تقریب برای جوابه بهینه است و در نتیجه، کوچک‌ترین توپ بین $z+1$ نمونه‌ی موازی، همواره یک $2$-تقریب برای جواب بهینه است. با توجه به این‌که پیچیدگی حافظه‌ی الگوریتم ~\رجوع{الگوریتم: الگوریتم با ضریب تقریب $2$ برای مسئله‌ی $1$-مرکز با $z$ داده‌ی پرت} برای یک نمونه از مرتبه‌ی $\cO(zd)$ است و زمان به‌روزرسانی‌ آن از مرتبه‌ی $\cO(d + \log{z})$ است، نتیجه زیر به دست می‌آید. 

\شروع{قضیه}
برای یک جویبار‌داده از نقاط در فضای $d$-بعدی، الگوریتم ~\رجوع{الگوریتم: الگوریتم با ضریب تقریب $2$ برای مسئله‌ی $1$-مرکز با $z$ داده‌ی پرت} یک $2$-تقریب برای مسئله‌ی $1$-مرکز با $z$ داده‌ی پرت با حافظه‌ی مصرفی $\cO(z^2d)$ و زمان به‌روزرسانی $\cO(zd + z\log{z})$ ارائه می‌دهد.
\پایان{قضیه}


\زیرزیرقسمت{الگوریتم تقریبی با ضریب تقریب $1.8$}

در این قسمت، الگوریتمی با ضریب تقریب $1.8$ برای مسئله‌ی $1$-مرکز با $z$ داده‌ی پرت ارائه می‌دهیم. برای بیان الگوریتم فرض می‌کنیم $r'$ای داده شده است که در شرایط زیر صدق می‌کند:
$$1.2r^* \leq r' \leq (1.2 + \frac{2\epsilon}{3})r^*$$
با فرض داده شدن $r'$، الگوریتم ~\رجوع{الگوریتم: الگوریتم با ضریب تقریب $1.8$ برای مسئله‌ی $1$-مرکز با $z$ داده‌ی پرت}، یک توپ با شعاع حداکثر $\frac{3}{2}r'$ ارائه می‌دهد که حداکثر $z$ نقطه از جویبارداده را نمی‌پوشاند. بدون کم شدن از کلیت مسئله، همانند قسمت قبلی فرض کنید که نقطه‌ی اول جویبار داده، جزء نقاط پرت در جواب بهینه نباشد. در نهایت برای حذف چنین فرضی کافی است $z+1$ نمونه از الگوریتم ارائه شده را به طور موازی اجرا نموده و از بین $z+1$ توپ جواب، توپ با کوچک‌ترین شعاع را به عنوان جواب نهایی بدهیم. با این تغییر، حافظه‌ی مصرفی، زمان به‌روزرسانی و زمان پاسخ‌گویی به پرسمان همگی در مرتبه‌ی $\cO(z)$ ضرب می‌شوند. برای ادامه‌ی کار به لم زیر نیاز داریم:

\شروع{شکل}[ht]
\centerimg{18-approx-1-center}{10cm}
\شرح{گسترش توپ $B(c, r')$ در راستای نقطه‌ی $q$}
\برچسب{شکل: $1.8$-تقریب $1$-مرکز}
\پایان{شکل}


\شروع{لم}

همان‌طور که در شکل ~\رجوع{شکل: $1.8$-تقریب $1$-مرکز} نشان داده شده است، نقطه‌ی $q$ از جویبارداده‌ی $P$ را در نظر بگیرید به‌طوری‌که در توپ بهینه‌ی $B^*$  قرار گرفته (داده‌ی پرت نیست) و فاصله‌ی آن از $p_1$ بزرگ‌تر مساوی $r'$ باشد. نقطه‌ی $c$ را در فاصله‌ی $\frac{1}{2}r'$ از $p_1$ برروی پاره‌خط $p_1q$ در نظر بگیرید. ثابت می‌شود توپ $B'(c, \frac{3r'}{2})$، توپ $B^*$ و $B(p_1, r')$ را به طور کامل می‌پوشاند (توپ $B^*$ و $B(p_1, r')$ به طور کامل داخل $B'$ قرار می‌گیرند). به چنین عملی گسترش توپ $B(p_1, r')$ در راستای نقطه‌ی $q$ گفته می‌شود.

\شروع{اثبات}

طبق لم ~\رجوع{لم:ahn-segment2}، مرکز‌ توپ $B^*$ که با $c^*$ نشان داده می‌شود، حداکثر $0.8r^*$ از $q$ فاصله دارد. برای هر نقطه‌ی $s$ که در توپ $B(c, r')$ قرار می‌گیرد، داریم:
$$\len{sc} \leq \len{sp_1}  + \len{p_1c} \leq r' + \frac{1}{2}r' \leq \frac{3r'}{2}$$
از طرفی برای هر نقطه‌ی $s$ داخل $B^*$  داریم:
$$\len{sc} \leq \len{sc^*}  + \len{c^*c} \leq r^* + 0.8r^* \leq \frac{3r'}{2}$$
و در نتیجه هر نقطه از $B^*$ داخل $B(c, \frac{3r'}{2})$ قرار می‌گیرد و بنابراین $B(c, \frac{3r'}{2})$ به طور کامل $B^*$ را می‌پوشاند. 

\پایان{اثبات}

\پایان{لم}

در واقع به عنوان نتیجه مستقیم لم بالا، اگر بتوانیم دو نقطه‌ی غیر پرت با فاصله‌ی بیش‌تر مساوی $r'$ پیدا کنیم، می‌توانیم یک توپ به شعاع $\frac{3}{2}r'$ ارائه دهیم که توپ $B^*$ را به طور کامل می‌پوشاند.

\شروع{الگوریتم}{الگوریتم با ضریب تقریب $1.8$ برای مسئله‌ی $1$-مرکز با $z$ داده‌ی پرت}
\دستور{فرض کنید $r'$ تقریب برای $1.2r^*$ و $z_0$ تعداد نقاط پرت قبل ار گسترش $B$ داده شده‌اند.}
\دستور{توپ $B(p_1, r')$ را در نظر بگیر.}
\دستور{حافظه‌ی میان‌گیر خالی $Q$ را در نظر بگیر.}
\به‌ازای{هر $p$ در مجموعه‌ی $P$}
\اگر{$p \not \in B$}
\دستور{$p$ را به $Q$ اضافه کن.}
\اگر{$B$ هنوز‌ گسترش‌ پیدا نکرده و $|Q| = z_0 + 1$}
\دستور{توپ $B$ را در راستای $p$ گسترش‌ بده.}
\به‌ازای{هر $q \in Q$}
\اگر{$q \in B$}
\دستور{$q$ را از $Q$ حذف کن.}
\پایان‌اگر{}
\پایان‌به‌ازای{}
\پایان‌اگر
\اگر{$Q = z+1$}
\دستور{از برنامه خارج شو.}
\پایان‌اگر{}
\پایان‌اگر{}
\پایان‌به‌ازای{}
\دستور{$B$ را برگردان}
\پایان{الگوریتم}

با توجه به لم بالا، با فرض داشتن $r'$، همان‌طور که در الگوریتم ~\رجوع{الگوریتم: الگوریتم با ضریب تقریب $1.8$ برای مسئله‌ی $1$-مرکز با $z$ داده‌ی پرت} می‌بینید، در ابتدا توپ $B(p_1, r')$ را به عنوان توپ کاندید در نظر می‌گیریم. حال نقاطی که خارج این توپ قرار می‌گیرند را داخل یک حافظه‌ی میان‌گیر قرار می‌دهیم. اگر اندازه حافظه‌ی میان‌گیر هیچ‌گاه به $z+1$ نرسید، بنابراین توپی با شعاع $r'$ پیدا کرده‌ایم که حداکثر $z$ نقطه خارج آن قرار دارد و چون  $1.2r^* \leq r' \leq (1.2 + \frac{2\epsilon}{3})r^*$ است، بنابراین یک جواب با ضریب تقریب $1.2 + \epsilon$ از جواب بهینه به دست آورده‌ایم. اگر حافظه‌ی میان‌گیر پر شود، حتما یکی از اعضای آن وجود دارد که جزء داده‌های پرت نبوده (طبق‌ اصل لانه‌کبوتری\پاورقی{Pigeonhole Principle}). بنابراین اگر نسبت به آن نقطه‌(کافی است تمام گزینه‌ها را امتحان کنیم) توپ اولیه را گسترش دهیم، به توپی می‌رسیم که تمام نقاط قبلی (غیر از نقاط داخل حافظه‌ی میان‌گیر) را پوشانده و مطمئن هستیم کل جواب بهینه را نیز می‌پوشاند. 

پس از گسترش هر کدام از  گزینه‌ها، کافی است در هر لحظه نگه‌داریم چند نقطه و کدام نقاط خارج از توپ گسترش‌‌یافته قرار می‌گیرند. توجه کنید در لحظه‌ی تشکیل توپ گسترش‌یافته، طبق لم بالا، تنها نقاط حافظه‌ی میان‌گیر که تعدادشان $z$ تاست ($z+1$ نقطه‌ی حافظه‌ی میان‌گیر به غیر از نقطه‌ای که در آن راستا توپ را گسترش داده‌ایم)، ممکن است خارج توپ گسترش‌یافته قرار بگیرند و نیازی به نگه‌داشتن نقاط قبلی نیست. اگر در ادامه‌ی جویبارداده تعداد نقاط خارج از توپ گسترش‌یافته بیش از $z$ عدد گردد، با پوشش‌ کامل $B^*$ تناقض دارد و در نتیجه با توجه به لم بالا، یا نقطه‌ی $q$ خود جزء داده‌های پرت در جواب بهینه بوده است یا شعاع $r'$ در شرایط‌ گفته شده صدق نمی‌کرده است و در هر صورت گزینه باید حذف گردد. با این حذف گزینه‌ها، در هر لحظه تعدادی گزینه داریم(همواره حداقل یک گزینه وجود دارد، چون حالتی وجود دارد که فرض برای آن درست است) و هر کدام یک جواب با شعاع حداکثر $\frac{3}{2}r'$ ارائه می‌دهند که اگر از بین آن‌ها توپ با شعار کمینه را به عنوان جواب نهایی بدهیم، مطمئنا یک جواب با تقریب حداکثر $1.8 + \epsilon$ از جواب بهینه ارائه داده‌ایم.

تا به این‌جا الگوریتمی ارائه دادیم که با فرض داشتن $r'$ و داده‌ی پرت نبودن $p_1$، با مصرف حافظه‌ی $\cO(z)$ و زمان به‌روزرسانی $\cO(z)$ در هر لحظه‌ می‌تواند یک $1.8 + \epsilon$-تقریب از جواب بهینه بدهد.

تنها قسمتی که مورد بررسی قرار نگرفته است، نحوه‌ی به‌دست‌آوردن $r'$ است که در این قسمت به بررسی آن خواهیم پرداخت. در ابتدا برای این‌که ایده‌ی اصلی را درک کنیم فرض کنید که می‌خواهیم یک الگوریتم دو گذره برای این مسئله ارائه دهیم، به‌طوری‌که در گذر اول، $r'$ محاسبه می‌شود و در گذر دوم، با استفاده از‌ $r'$ به‌دست‌آمده و الگوریتم ~\رجوع{الگوریتم: الگوریتم با ضریب تقریب $1.8$ برای مسئله‌ی $1$-مرکز با $z$ داده‌ی پرت}، یک $1.8 + \epsilon$ تقریب ارائه می‌گردد. در ادامه نشان داده می‌شود، چگونه می‌توان این دو گذر را هم‌زمان اجرا نمود. برای پیدا کردن $r'$ کافی است که با استفاده از یک الگوریتم $\alpha$-تقریب (به طور مثال الگوریتم $2$-تقریب ارائه شده در قسمت قبلی)، یک $r$ به‌دست آوریم. طبق الگوریتم استفاده شده داریم:
$$r^* \leq r \leq \alpha r^*$$ 
حال اگر بازه‌ی $[0, 1.2r]$ را به 
$$m = \ceil{\frac{3}{2} \times 1.2 \times \alpha \times \frac{1}{\epsilon}}$$
 قسمت مساوی تقسیم کنیم، طول هر قسمت برابر است با:
 $$\frac{1.2  r}{m} = \frac{2}{3} \times \frac{r}{\alpha} \times \epsilon \leq \frac{2\epsilon}{3}r^*$$
 از طرفی چون $1.2r^* \leq 1.2r$ است، بنابراین یکی از این بازه‌ها $1.2r^*$ را شامل می‌شود و انتهای آن بازه با توجه به طول بازه‌ها، کاندیدای مناسبی برای $r'$ است. بنابراین کافی است پس از پیدا کردن یک $\alpha$-تقریب برای مسئله‌ی $1$-مرکز‌ با $z$ داده‌ی پرت، به ازای سرهای تمام بازه‌ها، الگوریتم ~\رجوع{الگوریتم: الگوریتم با ضریب تقریب $1.8$ برای مسئله‌ی $1$-مرکز با $z$ داده‌ی پرت} را اجرا کنیم و از بین گزینه‌هایی که باقی می‌مانند کوچک‌ترین توپ را به عنوان جواب نهایی بدهیم. با توجه به این‌که برای سر یکی از بازه‌ها، $r$ در شرایط  $1.2r^* \leq r \leq (1.2 + \frac{2\epsilon}{3})r^*$  صدق می‌کند، در نتیجه یکی از گزینه‌ها یک $1.8 + \epsilon$-تقریب برای جواب بهینه است و در نتیجه توپ با شعاع کمینه نیز‌ همین ضریب تقریب را تضمین می‌کند.
 
 تنها قسمتی که نیاز به دقیق شدن دارد، قسمت تک‌گذره کردن الگوریتم است. همان‌طور که گفته شد، از الگوریتم ~\رجوع{الگوریتم: الگوریتم با ضریب تقریب $2$ برای مسئله‌ی $1$-مرکز با $z$ داده‌ی پرت} که الگوریتمی $2$-تقریب است، برای پیدا کردن $r'$ استفاده می‌کنیم. فرض کنید $r_i$ برابر شعاع الگوریتم $2$-تقریب برای جویبارداده تا $i$اُمین عنصر جویبار داده باشد. به وضوح دنباله‌ی $r_i$ یک دنباله صعودی است. به ازای هر $i$، $k$ را عدد صحیحی در نظر بگیرید که رابطه‌‌ی زیر برقرار باشد:
 $$2^{k-1} \leq r_i \leq 2^k$$
 با توجه به $k$ بالا $l_i = 2^k$ قرار دهید. به وضوح طبق‌ رابطه‌ی گفته شده، $l_i \leq 2r_i$ است و در نتیجه $l_i$ یک $4$-تقریب برای مسئله‌ی $1$-مرکز  با $z$ داده‌ی پرت است.
 
 حال کافی است بازه‌ی $[0, 1.2l_i]$ را به $m = \ceil{\frac{7.2}{\epsilon}}$ قسمت تقسیم کنیم. با این تقسیم‌بندی، طول هر بازه، $t_i = \frac{1.2l_i}{m}$ می‌شود و مجموعه سر بازه‌ها برابر
 $$R_i = \{ j \times t_i \ | 1 \leq j \leq m \}$$
 می‌گردد. طبق توضیحات قسمت قبل، سر یکی از بازه‌ها کاندیدای مناسبی برای $r'$ است. 
 
 حال کافی است که در هر لحظه $m$ نمونه از الگوریتم ~\رجوع{الگوریتم: الگوریتم با ضریب تقریب $1.8$ برای مسئله‌ی $1$-مرکز با $z$ داده‌ی پرت} را به ازای هر $r \in R_i$ به صورت موازی اجرا نماییم. به ازای اضافه شدن نقطه‌ی $p_i$، اگر $l_i = l_{i-1}$ باشد، $R_i = R_{i-1}$ است و در نتیجه بدون هیچ تغییری کافی است $p_i$ را به تمام نمونه‌های موازی اضافه کنیم. در حالتی که $l_{i-1} < l_{i}$ باشد، مجموعه‌ی $R_i$ را می‌توان به دو زیرمجموعه تقسیم نمود. اعضایی از $r \in R_i$ که کم‌تر مساوی $1.2l_{i-1}$ هستند و در نتیجه داخل $R_{i-1}$ نیز قرار دارند (چون $\frac{t_i}{t_{i-1}}$ همواره توان صحیحی از دو است). برای چنین اعضایی، کافی است، نمونه معادل‌ آن را در $R_{i-1}$ بدون هیچ تغییری پیدا کرد و نقطه‌ی $p_i$ را به آن اضافه کرد. اگر $r \in R_i$ باشد و در $R_{i-1}$ نباشد، در نتیجه $l_{i-1} \leq 1.2l_{i-1} \leq r$ است. با توجه با نحوه‌ی عمل‌کرد الگوریتم ~\رجوع{الگوریتم: الگوریتم با ضریب تقریب $2$ برای مسئله‌ی $1$-مرکز با $z$ داده‌ی پرت}، می‌دانیم در هر لحظه $z$ نقطه‌ای به عنوان داده‌ی پرت در نظر گرفته می‌شود که فاصله‌ی بزرگ‌تر مساوی $l_i$ دارند و بقیه نقاطی که تا کنون آمده‌اند فاصله‌ای کم‌تر مساوی $l_i$ دارند. بنابراین در بین تمام نقاط جویبار داده تا کنون، حداکثر $z$ نقطه‌ی ذخیره شده در حافظه‌ی میان‌گیر الگوریتم ~\رجوع{الگوریتم: الگوریتم با ضریب تقریب $2$ برای مسئله‌ی $1$-مرکز با $z$ داده‌ی پرت} خارج توپ $B(p_1, r)$ می‌افتند. بنابراین اگر بخواهیم به ازای این $r$ جدید الگوریتم ~\رجوع{الگوریتم: الگوریتم با ضریب تقریب $1.8$ برای مسئله‌ی $1$-مرکز با $z$ داده‌ی پرت} را برروی نقاط جویبارداده تاکنون اجرا نماییم، کافی است الگوریتم را به ازای نقاط داخل حافظه‌ی میان‌گیر اجرا نموده و مطمئن هستیم که بقیه‌ی نقاط به علت قرار گیری داخل $B(p_1, r)$، تاثیری در روند اجرای الگوریتم نخواهند داشت. 
 
 با جمع‌بندی روند توضیح داده شده، ساخت یک نمونه‌ی جدید از الگوریتم ~\رجوع{الگوریتم: الگوریتم با ضریب تقریب $1.8$ برای مسئله‌ی $1$-مرکز با $z$ داده‌ی پرت} معادل اضافه کردن حداکثر $z$ نقطه‌ی موجود در حافظه‌ی میان‌گیر الگوریتم ~\رجوع{الگوریتم: الگوریتم با ضریب تقریب $2$ برای مسئله‌ی $1$-مرکز با $z$ داده‌ی پرت} به نمونه‌ی جدید از الگوریتم  که از مرتبه‌ی $\cO(zd)$ زمان می‌برد. در هر مرحله هم حداکثر $m$ نمونه‌ی جدید ساخته می‌شود، بنابراین زمان به‌روزرسانی نمونه‌ها در هر مرحله حداکثر $\cO(\frac{zd}{\epsilon})$ است. از طرفی در هر لحظه $mz$($z$ به خاطر عدم اطمینان از نقطه‌ی دومی که داخل $B^*$ قرار می‌گیرد و $m$ به علت عدم اطمینان از محل قرارگیری $r'$ در بازه‌ها) نمونه موازی از‌ الگوریتم ~\رجوع{الگوریتم: الگوریتم با ضریب تقریب $1.8$ برای مسئله‌ی $1$-مرکز با $z$ داده‌ی پرت} در حال اجراست. بنابراین، زمان به‌روزرسانی نهایی الگوریتم برابر $\cO(\frac{z^2d}{\epsilon})$ است و حافظه‌ی مصرفی نیز متناسب با $mz$ نمونه‌ی موازی از الگوریتم ~\رجوع{الگوریتم: الگوریتم با ضریب تقریب $1.8$ برای مسئله‌ی $1$-مرکز با $z$ داده‌ی پرت} برابر $\cO(\frac{zd}{\epsilon})$ است. با دخیل کردن امکان پرت نبودن نقطه‌ی اول جویبار داده، به قضیه‌ی زیر می‌رسیم:
 
 \شروع{قضیه}
 الگوریتم ~\رجوع{الگوریتم: الگوریتم با ضریب تقریب $1.8$ برای مسئله‌ی $1$-مرکز با $z$ داده‌ی پرت} با مصرف حافظه‌ی $\cO(\frac{z^2d}{\epsilon})$ و زمان به‌روزرسانی $\cO(\frac{z^3d}{\epsilon})$، در هر لحظه با صرف زمان اجرای $\cO(\frac{z^2}{\epsilon})$ جوابی با ضریب تقریب $1.8 + \epsilon$ ارائه می‌دهد.
 \پایان{قضیه}


\شروع{شکل}
\centering
\begin{tikzpicture}
  [scale=.75,auto=left,every node/.style={circle,fill=blue!15}]
 \tikzset{edge/.style = {->,> = latex'}}
  \node (1) at (1,1) {شروع};
  \node (2) at (5,1)  {\rl{توپ اولیه}};
  \node (3) at (10,1)  {\rl{توپ گسترش‌یافته}};
  \node (e) at (15,1)  {پایان};

  \foreach \from/\to/\label in {1/2/0,2/3/1,3/e/2} 
    \draw[edge] (\from) edge node[midway, fill=white]{$z_\label$} (\to);

\end{tikzpicture}
\شرح{نحوه‌ی اجرای الگوریتم~\رجوع{الگوریتم: الگوریتم با ضریب تقریب $1.8$ برای مسئله‌ی $1$-مرکز با $z$ داده‌ی پرت}}
\برچسب{شکل:گراف‌ بدون جهت برای مسئله‌ی $1$-مرکز با ضریب تقریب $1.8$}
\پایان{شکل}
 
اگر بخواهیم الگوریتم گفته شده را جمع‌بندی کنیم، الگوریتم همان‌طور که در شکل ~\رجوع{شکل:گراف‌ بدون جهت برای مسئله‌ی $1$-مرکز با ضریب تقریب $1.8$} نشان‌داده شده است. در ابتدا $z_0$ نقطه‌ی اول را به عنوان داده‌ی پرت در نظر می‌گیرد و سپس $z_0 + 1$اُمین نقطه‌ی جویبار داده را به عنوان نقطه‌ی اول و مرکز‌ توپ $B$ به شعاع $r'$ در نظر می‌گیرد. سپس $z_1$ نقطه‌ی اولی از ادامه‌ی جویبارداده که بیرون این توپ قرار می‌گیرند را به عنوان داده‌ی پرت در نظر گرفته و به ازای $z_1+1$اُمین نقطه‌ی خارج $B$، آن را در همان راستا گسترش می‌دهد. سپس $z_2$ نقطه‌ی دیگر از ادامه‌ی جویبارداده که خارج $B-$گسترش‌یافته قرار می‌گیرند را نیز‌به عنوان داده‌ی پرت در نظر می‌گیرد، حال اگر نقطه‌ی دیگری در جویبارداده وجود داشته باشد که خارج $B$ بیفتد با توجه به اینکه $\sum_{i = 0}^{2} z_i = z$ است، تعداد نقاط پرت از‌ $z$ بیش‌تر شده و نشان می‌دهد یکی از فرض‌های اولیه اشتباه بوده و در نتیجه، گزینه حذف می‌گردد.

\زیرقسمت{مسئله‌ی $1$-مرکز‌ پوشاننده در حالت جویبار داده}

در این زیر قسمت به بررسی مسئله‌ی تقریبا جدیدی می‌پردازیم. در ابتدا به تعریف دقیق مسئله می‌پردازیم:

\شروع{تعریف}

مجموعه نقاط $P$ داده شده‌اند. به توپ $B$ یک $\alpha$-تقریب برای مسئله‌ی $1$-مرکز‌ پوشاننده گویند اگر نه تنها تمام نقاط $P$ را بپوشاند، بلکه توپ بهینه‌ی $1$-مرکز این نقاط که با $B^*$ نشان داده می‌شود را نیز به طور کامل می‌پوشاند و شعاع آن، حداکثر $\alpha$ برابر شعاع توپ بهینه باشد. 

\پایان{تعریف}

اگر این مسئله را در حالت جویبار داده در نظر بگیریم, هدف نگه‌داری مجموعه‌ای هسته که بتوان با استفاده از آن، در هر لحظه یک $\alpha$-تقریب از مسئله‌ی $1$-مرکز در حالت پوشاننده ارائه داد. برای این مسئله دو الگوریتم ارائه می‌دهیم. در الگوریتم اول، الگوریتم $1.8$-تقریب ارائه شده برای مسئله‌ی $1$-مرکز با $z$ داده‌ی پرت را به گونه‌ای تغییر می‌دهیم که الگوریتمی با ضریب تقریب $1.8$ برای مسئله‌ی $1$-مرکز پوشاننده در حالت جویبار داده ارائه دهد. در الگوریتم دوم، با استفاده از الگوریتمی دلخواه برای  مسئله‌ی $1$-مرکز در حالت جویبارداده به عنوان جعبه‌ی سیاه\پاورقی{Black Box}، یک الگوریتم برای مسئله‌ی $1$-مرکز پوشاننده در حالت جویبارداده ارائه می‌دهد.

\زیرزیرقسمت{الگوریتم $1.8 + \epsilon$-تقریب برای مسئله‌ی $1$-مرکز پوشاننده در حالت جویبارداده} 

اگر الگوریتم \رجوع{الگوریتم: الگوریتم با ضریب تقریب $1.8$ برای مسئله‌ی $1$-مرکز با $z$ داده‌ی پرت}، را با $z=0$ برروی جویبار داده اجرا کنیم، با مصرف حافظه‌ و زمان به‌روزرسانی از مرتبه‌ی $\cO(\frac{d}{\epsilon})$ می‌تواند یک جواب $1.8$-تقریب از جواب بهینه برای مسئله‌ی $1$-مرکز ارائه می‌دهد. توجه کنید که برای حالتی که $z=0$ است، تنها لازم است $m$ نمونه از الگوریتم اجرا نمود که هر نمونه نیز $\cO(d)$ حافظه مصرف می‌کند. تنها تفاوتی که با الگوریتم قبلی در این استفاده‌ی جدید وجود دارد این است که، در زمان به‌ دست آوردن جواب نهایی، از بین $m$ گزینه، آن گزینه‌ای را به عنوان جواب نهایی می‌دهیم که کم‌ترین $r'$ را دارد (نه گزینه‌ای که کم‌ترین  شعاع را داشته باشد). با توجه به این که توپ با $r'$ کم‌تری در گزینه‌ها نیست، بنابراین مطمئن هستیم که برای کم‌ترین مقدار $r'$ بین گزینه‌ها (که با $r'_m$ نشانش می‌دهیم)، داریم:
$$r_m \leq (1.2 + \frac{2\epsilon}{3})r^*$$
زیرا مطمئن هستیم $r'$ ای که در بازه‌ی $[1.2r^*, (1.2 + \frac{2\epsilon}{3})r^*]$ است در بین گزینه‌ها قرار دارد. حال اگر توپی که با استفاده از $r'_m$ ساخته شده باشد، شعاعی برابر با $\frac{3}{2}r'_m$ داشته باشد، مطمئن هستیم که $B^*$ را به طور کامل می‌پوشاند و از طرفی شعاعش حداکثر $1.8$-برابر $r^*$ است. اما اگر توپ گسترش نیافته باشند، یک توپ داریم که تمام نقاط را می‌پوشاند و شعاعش حداکثر $1.2$-برابر شعاع بهینه است. برای ادامه‌، نیاز به به دو لم زیر داریم:


\شروع{شکل}[ht]
\centerimg{empty-half-ball}{10cm}
\شرح{اثبات لم ~\رجوع{لم: نیم‌کره}}
\برچسب{شکل: نیم‌کره}
\پایان{شکل}


\شروع{لم}
\برچسب{لم: نیم‌کره}
فرض کنید مجموعه نقاط $P$ از نقاط در فضای $\IR^d$ داده شده‌اند. $B^*$ را توپی با شعاع کمینه در نظر بگیرید که تمام نقاط را می‌پوشاند. آن‌گاه پوسته‌ی هر نیم‌‌کره از $B^*$ شامل حداقل یک نقطه از $P$ است.

\شروع{اثبات}

از برهان خلف استفاده می‌کنیم. همان‌طور که در شکل ~\رجوع{شکل: نیم‌کره} نشان داده شده است، فرض کنید نیم‌کره‌ای از $B^*$ وجود داشته باشد که در پوسته‌ی آن هیچ نقطه‌ای از‌ $P$ قرار ندارد. بردار عمود بر صفحه‌ای که کره‌ی $B^*$ را به دو نیم کره تقسیم می‌کند را $\vec{s}$ در نظر بگیرید (در جهت به خارج نیم‌کره). حال کافی است توپ $B^*$ را به اندازه‌ی بسیار کمی(کم‌تر از فاصله‌ی نقاط توپ و نقاط $P$) در جهت $\vec{s}$ حرکت دهیم. با این حرکت، هیچ نقطه‌ای برروی نیم‌کره قرار نمی‌گیرد و پوسته‌ی نیم‌کره‌ی مقابل نیز کاملا خالی می‌شود. بنابراین می‌توان شعاع توپ را کاهش داد و به توپ قهوه‌ای رنگ که شعاع کم‌تری نسبت به $B^*$ دارد رسید که تمام نقاط را می‌پوشاند، که با بهینه بودن $B^*$ تناقض دارد. بنابراین فرض اولیه مبنی بر وجود نیم‌کره‌ای با پوسته‌ی خالی اشتباه بوده است.

\پایان{اثبات}

\پایان{لم}

\شروع{لم}
\برچسب{لم: افزایش شعاع}

فرض کنید مجموعه نقاط $P$ از نقاط در فضای $\IR^d$ داده شده‌اند. $B^*(c^*, r^*)$ را توپی با شعاع کمینه در نظر بگیرید که تمام نقاط را می‌پوشاند. توپ $B(c, r)$ با شعاعی $\alpha r^*$ در نظر بگیرید که تمام نقاط $P$ را می‌پوشاند. ثابت می‌شود اگر شعاع توپ $B$ را $\sqrt{2}$ برابر کنیم، کل توپ $B^*$ را نیز می‌پوشاند. 

\شروع{اثبات}

همان‌طور که در شکل ~\رجوع{شکل: افزایش شعاع} نشان داده شده است، هر دوی $B$ و $B^*$ کلیه‌ی نقاط را می‌پوشانند. حال صفحه‌ی عمود بر پاره‌خط واصل $\vec{c^*c}$ را نظر بگیرید. مرز تقاطع صفحه‌ی رسم شده با توپ $B^*$ یک دایره به نام $D$ تشکیل می‌دهد. همان‌طور که در شکل پیداست، تمام نقاط این دایره از $c$ به یک فاصله‌اند (به علت عمود بودن $\vec{c^*c}$ بر صفحه‌ی رسم شده)، بنابراین یا به طور کامل داخل $B$ قرار می‌گیرد یا بیرون آن. ادعا می‌کنیم، دایره‌ی $D$ به طور کامل داخل $B$ قرار می‌گیرد، زیرا در صورتی که خارج $B$ قرار بگیرد، پوسته‌ی نیم‌کره‌ی حاصل از تقاطع این صفحه که $c$ در آن قرار نمی‌گیرد، به طور کامل خارج از $B$ قرار گرفته و از طرفی طبق‌ لم ~\رجوع{لم: نیم‌کره}، حداقل یک نقطه‌ی $s$ از $P$ برروی پوسته‌ی این نیم‌کره قرار دارد، که می‌توان نتیجه گرفت $s$ به وسیله‌ی $B$ پوشانده نمی‌شود. تناقض، پس دایره‌ی $D$ به طور کامل داخل $B$ قرار می‌گیرد. 

\شروع{شکل}[ht]
\centerimg{cover-1-center}{10cm}
\شرح{اثبات لم ~\رجوع{لم: افزایش شعاع}}
\برچسب{شکل: افزایش شعاع}
\پایان{شکل}

نقطه‌ی دلخواه $q$ برروی این دایره را در نظر بگیرید. چون این نقطه برروی پوسته‌ی $B^*$ قرار دارد، بنابراین داریم:
$$\len{c^* q} = r^*$$
از طرفی دیگر، چون $q \in B$ است بنابراین داریم:
$$\len{c q} \leq r = \alpha r^*$$
و چون زاویه‌ی $\angle{cc^*q}$ قائم است، طبق‌ رابطه‌ی فیثاغورث داریم:
$$\len{cc^*} = \sqrt{\len{cq}^2 - \len{c^*q}^2} \leq \sqrt{\alpha^2 - 1} r^*$$
حال با توجه به این‌که $1 \leq \alpha$ است، اگر شعاع دایره‌ی $B$ را $\sqrt{\alpha ^ 2 - 1}r^*$ افزایش دهیم، دایره‌ی $B^*$ را به طور کامل می‌پوشاند. در واقع برای افزایش شعاع به این میزان کافی است، شعاع را $\frac{1 + \sqrt{\alpha ^ 2 - 1}}{\alpha}$ برابر کنیم. اگر مطابق شکل ~\رجوع{شکل: نمودار تابع} نمودار این تابع را رسم کنیم، می‌بینیم که حداکثر تابع در نقطه‌ی $\sqrt{2}$ و برابر $\sqrt{2}$ خواهد بود. توجه کنید که اگر $\alpha \leq \sqrt{2}$ باشد، تابع در بازه‌ی $[1, \alpha]$ مقداری کم‌تر از $\sqrt{2}$ به خود می‌گیرد و همان‌طور که در نمودار پیداست، چون تابع در این بازه صعودی است، مقدار بیشینیه در خود $\alpha$ به دست می‌آید.

\پایان{اثبات}

\پایان{لم}

\شروع{شکل}[ht]
\centerimg{function}{10cm}
\شرح{نمودار تابع $\frac{1 + \sqrt{\alpha ^ 2 - 1}}{\alpha}$}
\برچسب{شکل: نمودار تابع}
\پایان{شکل}

با توجه به لم ~\رجوع{لم: افزایش شعاع}، برای حالتی که توپ گسترش پیدا نکرده است، اگر شعاعش را $\sqrt{2}$ برابر کنیم، تضمین می‌کند که دایره‌ی بهینه را پوشانده است. با توجه به این‌که در این حالت شعار توپ کم‌تر  مساوی $1.2r^*$ است، با $\sqrt{2}$ برابر کردن شعاعش به توپی با شعاعی حداکثر $1.7$ برابر $r^*$ دست خواهیم یافت. بنابراین در هر دو حالت، توپی را بر می‌گردانیم که توپ بهینه را به طور کامل می‌پوشاند، شعاع آن حداکثر $1.8 + \epsilon$ برابر جواب بهینه است.

\زیرزیرقسمت{الگوریتم $1.7$-تقریب برای مسئله‌ی $1$-مرکز پوشاننده در حالت جویبارداده} 

در این قسمت با استفاده از‌ لم ~\رجوع{لم: افزایش شعاع}، الگوریتمی با ضریب تقریب $1.7$ برای مسئله‌ی $1$-مرکز پوشاننده در حالت جویبارداده ارائه می‌دهیم. ایده‌ی اصلی این الگوریتم، استفاده از یک الگوریتم $\alpha$-تقریب برای مسئله‌ی $1$-مرکز در حالت جویبار داده است و افزایش شعاع آن در زمان پاسخ‌گویی به پرسمان برای پوشش کامل توپ بهینه. همان‌طور که در فصل کارهای پیشین ذکر شده است، بهترین الگوریتم موجود برای مسئله‌ی $1$-مرکز در حالت جویبارداده، الگوریتم ارائه شده به وسیله‌ی آگاروال با حافظه‌ی مصرفی و زمان به‌روزرسانی $\cO(d)$ و ضریب تقریب $1.22$ است.
حال اگر جواب این الگوریتم را بخواهیم افزایش بدهیم، طبق‌ لر ~\رجوع{لم: افزایش شعاع}، باید شعاع آن را $\frac{1 + \sqrt{1.22^2 - 1}}{1.22}$ برابر کنیم، که توپی با شعاع حداکثر $1.7$ برابر شعاع توپ بهینه ارائه می دهد. زمان اجرا و حافظه‌ی مصرفی این الگوریتم همانند الگوریتم آگاروال، $\cO(d)$ است.

\زیرقسمت{مسئله‌ی $1$-مرکز با داده‌های پرت در حالت جویبار داده با تعداد داده‌های پرت ثابت}

در این قسمت، با استفاده از الگوریتم‌هایی که در قسمت قبل برای مسئله‌ی $1$-مرکز پوشاننده در حالت جویبارداده ارائه شده است، الگوریتم جدیدی برای مسئله‌ی $1$-مرکز با داده‌های پرت در حالت جویبار داده با تعداد داده‌های پرت ثابت ارائه می‌دهیم. در ابتدا لمی را ثابت می‌کنیم که نقش اساسی در اجرای الگوریتم دارد.

\شروع{لم}
\برچسب{لم: کران‌بالای حالت‌های داده‌های پرت}

مجموعه‌ی $P$ از نقاط در فضای $\IR^d$ در نظر بگیرید. حداکثر $(d+1)^z$ حالت برای انتخاب نقاط پرت این مجموعه وجود دارد. در واقع حداکثر $(d+1)^z$ زیرمجموعه‌ی $z$ عضوی از $P$ وجود دارد که دایره‌ی با شعاع کمینه‌ای که سایر نقاط $P$ را می‌پوشاند، هیچ کدام از اعضای زیر مجموعه‌ی انتخابی را نپوشاند.

\شروع{اثبات}

از استقراء برای اثبات لم استفاده می‌کنیم.

\شروع{فقرات}

\فقره{\مهم{پایه}: حکم برای $z = 0$ برقرار است، زیرا در این حالت تنها یک حالت برای انتخاب مجموعه‌ی داده‌های پرت وجود دارد. (مجموعه‌ی $\emptyset$)}

\فقره{\مهم{فرض}: فرض کنید که به ازای مجموعه‌ی دلخواه $P$ در $\IR^d$، و $z = k-1$ حداکثر $(d+1)^z$ حالت برای انتخاب زیرمجموعه داده‌های پرت وجود دارد.}

\فقره{\مهم{حکم}: ثابت می‌کنیم به ازای هر مجموعه‌ی دلخواه $P$ در $\IR^d$ و $z = k$ حداکثر $(d+1)^z$ حالت برای انتخاب زیرمجموعه داده‌های پرت وجود دارد. توپ به شعاع کمینه $B^*$ که تمام نقاط $P$ را می‌پوشاند را در نظر بگیرید. مجموعه‌ی $S \subset P$ را که برروی پوسته‌ی $B^*$ قرار دارند را در نظر بگیرید. از بین اعضای $S$ می‌توان زیرمجموعه‌ی حداکثر $d+1$ عضوی $S'$ را انتخاب کرد به طوری که دایره‌ی محاطی که کم‌ترین شعاع آن همان $B^*$ گردد (در فضای $d$-بعدی، هر توپ را با حداکثر $d+1$ نقطه روی پوسته‌ی آن می‌توان مشخص کرد).

زیرمجموعه‌ای دلخواه $O$ برای داده‌های پرت را  در نظر بگیرید. اگر $O \cap S' = \emptyset$ باشد، آن‌گاه کوچک‌ترین توپی که $P - O$ را می‌پوشاند همان $B^*$ است، زیرا $S'$ به طور کامل داخل $P - O$ قرار می‌گیرد. بنابراین فرض تهی بودن اشتراک $O$ و $S'$ غلط‌ است. بنابراین حداقل یکی از اعضای $S'$ داخل $O$ است. این عضو $d+1$ حالت برای انتخاب دارد. اگر این نقطه را از $O$ و $P$ حذف کنیم، مسئله به تعداد حالت‌های انتخاب داده‌های پرت با اندازه‌ی $k-1$ از مجموعه‌ی جدید تبدیل می‌شود که طبق فرض استقرا $(d+1)^{z-1}$ حالت دارد. در نتیجه در کل تعداد حالات انتخاب $O$ حداکثر برابر $(d+1)^z$ است.}

\پایان{فقرات}

\پایان{اثبات}

\پایان{لم}

الگوریتم به گونه‌ای عمل می‌کند که تعدادی حالت مختلف را به طور موازی دنبال می‌کند. در ابتدا، قبل از ورود اولین نقطه، تنها یک حالت داریم (حالتی که مجموعه نقاط غیر پرت و پرت هردو تهی هستند). به ازای ورود نقطه‌ی جدید $p$ از جویبار داده، به ازای هر کدام از حالت‌ها، آن را با دو حالت جای‌گزین می‌کنیم. اولین حالت، حالتی است که نقطه‌ی جدید را به مجموعه‌ نقاط پرت حالت اولیه اضافه می‌کنیم و دومین حالت، حالتی است که آن را به مجموعه نقاط غیر پرت حالت اولیه اضافه ‌می‌کنیم. توجه کنید که مجموعه نقاط پرت، یک حافظه‌ با اندازه‌ی حداکثر $z$ و مجموعه نقاط غیر پرت، همان اجرای الگوریتم $1$-مرکز پوشاننده در حالت جویبار داده است. 

یک گزینه در صورتی که تعداد نقاط‌ پرتش از $z$ بیش‌تر شود یا اینکه یکی از نقاط پرتش داخل توپ پوشاننده‌ی نقاط غیر پرت قرار بگیرد، حذف می‌گردد. با توجه به این‌که برای نقاط غیر پرت، از‌ الگوریتمی استفاده می‌کنیم که توپ بهینه را به طور کامل می‌پوشاند، تعداد حالات نقاط پرت نیز در این حالت، حداکثر $(d+1)^z$ حالت است. بنابراین الگوریتمی ارائه دادیم که مسئله‌ی $1$-مرکز با $z$ داده‌ی پرت در حالت جویبار داده را با ضریب تقریب $1.7$(حاصل استفاده از الگوریتم ارائه شده در قسمت قبل برای مسئله‌ی $1$-مرکز پوشاننده در حالت جویبارداده) حل می‌نماید. حافظه‌ی مصرفی و زمان به‌روزرسانی الگوریتم ارائه شده از مرتبه‌ی $\cO(d \times (d+1)^z)$ است.

\قسمت{مسئله‌ی $2$-مرکز با داده‌های پرت در حالت جویبار داده}

در این بخش، یک الگوریتم $1.8 + \epsilon$-تقریب برای مسئله‌ی $2$-مرکز با داده‌های پرت در حالت جویبار داده ارائه می‌شود. در تمام الگوریتم‌های ارائه شده، فرض شده است که نقطه‌ی اول جویبارداده $p_1$، داده‌ای غیر پرت است. این فرض را می‌توان همانند روشی که برای حذف این محدودیت در بخش دوم همین فصل انجام دادیم، با در نظر گرفتن $z+1$ نمونه از الگوریتم ارائه شده برطرف نمود.

همان‌طور که در بخش نماد‌گذاری‌ها ذکر شده بود، $B_1^*(c_1^*, r^*)$ و $B_2^*(c_2^*, r^*)$ را توپ‌های جواب بهینه برای مسئله‌ی $2$-مرکز با $z$ داده‌ی پرت برای جویبارداده‌ی $P$ در نظر بگیرید. شعاع توپ‌های بهینه را با $r^*$ و فاصله‌ی دو توپ بهینه را با $\delta^*$ نشان می‌دهیم. برای این‌که بتوانیم به نتیجه مطلوب برسیم، مسئله را به دو حالت تقسیم می‌کنیم. در زیربخش اول به بررسی حالت $\delta^* \leq \alpha r^*$ و در زیربخش دوم به بررسی حالت $\delta^* > \alpha r^*$ می‌پردازیم، که در آن $\alpha$ یک عدد ثابت است که در طول تحلیل نشان داده می‌شود $16$ یک انتخاب مناسب برای $\alpha$ است.


\زیرقسمت{حالت $\delta^* \leq \alpha r^*$}

ایده‌ی اصلی این بخش، تغییر الگوریتم ارائه شده به وسیله‌ی کیم\پاورقی{Kim} و آهن\پاورقی{Ahn} \مرجع{kim2014improved} که در اصل برای نگه‌داری مجموعه‌ی هسته برای مسئله‌ی $2$-مرکز در حالت جویبارداده با ضریب تقریب $1.8 + \epsilon$ ارائه شده است، حاصل می‌گردد. الگوریتم آهن و کیم، در واقع مبنای الگوریتم‌های ارائه شده در همین پایان‌نامه برای مسئله‌ی $1$-مرکز با داده‌های پرت در حالت جویبارداده با ضریب تقریب $1.8 + \epsilon$ و مسئله‌ی $1$-مرکز پوشاننده در حالت جویبارداده با ضریب تقریب $1.8 + \epsilon$ است. تغییرات اعمال شده نیز‌ بسیار شبیه عملکرد الگوریتم $1.8 + \epsilon$-تقریب برای مسئله‌ی $1$-مرکز با داده‌ی پرت است. بنابراین، در این قسمت، برای جلوگیری از تکرار، در ابتدا گام‌های اصلی الگوریتم کیم را بیان کرده و سپس تغییراتی که در الگوریتم جدید مورد نیاز است را ذکر می‌کنیم.

\شروع{شکل}
\وسط‌چین
\begin{tikzpicture}
  [scale=.75,auto=left,every node/.style={circle,fill=blue!15}]
 \tikzset{edge/.style = {->,> = latex'}}
  \node (1) at (1,2.5) {شروع};
  \node (2a) at (3,4)  {2a};
  \node (2b) at (5,4)  {2b};
  \node (2c) at (7,4)  {2c};
  \node (3a) at (3,1)  {3a};
  \node (3b) at (5,2)  {3b};
  \node (3c) at (7,2)  {3c};
  \node (3d) at (5,0)  {3d};
  \node (3e) at (7,0)  {3e};
 \node (E) at (9, 2) {پایان};
 \node[fill=white] (n1) at (2, 5.25) {$n_1$};
 \node[fill=white] (n2) at (4, 5.25) {$n_2$};
 \node[fill=white] (n3) at (6, 5.25) {$n_3$};
  \node[fill=white] (n4) at (8, 5.25) {$n_4$};

  \foreach \from/\to in {1/2a,1/3a,2a/2b,2b/2c,3a/3b,3a/3b,3a/3d,3b/3c,3d/3e,2c/E,3c/E,3e/E}
    \draw[edge] (\from) to (\to);

\end{tikzpicture}
\شرح{حالت‌های الگوریتم کیم و آهن \مرجع{kim2014improved}}
\برچسب{شکل: حالت‌های الگوریتم کیم و آهن}
\پایان{شکل}

همان‌طور که در شکل ~\رجوع{شکل: حالت‌های الگوریتم کیم و آهن} نشان داده شده است، الگوریتم کیم و آهن، دارای ۱۰ حالت مختلف است. متناسب با نقاطی که تا کنون در جویبارداده آمده‌اند، الگوریتم در یکی از حالت‌های بالا قرار دارد. در هر کدام‌ یک از حالت‌‌ها، الگوریتم دو توپ به عنوان نماینده‌ی جواب در این حالت در نظر می‌گیرد. انتقال بین حالت‌ها، تنها زمانی رخ می‌دهد که نقطه‌ای در جویبارداده وارد شود که در هیچ‌ کدام از توپ‌های کاندید قرار نگیرد.

الگوریتم کیم و آهن، از گره‌ی شروع، شروع می‌کند و با رسیدن نقاط جدید از جویبارداده در طول گراف مطابق‌ با یال‌ها جابه‌جا می‌گردد. در بعضی از حالت‌ها، بیش‌تر یک حالت برای حالت بعدی وجود دارد (گره‌ی معادل آن حالت، درجه‌ی خروجی بیش از یک دارد) و الگوریتم هیچ اطلاعات قبلی ندارد که کدام یک حالت را به عنوان حالت بعدی انتخاب کند. اما اگر بیش‌تر دقت کنید، تنها ۳ مسیر از گره‌ی شروع به گره‌ی انتهایی وجود دارد. بنابراین کافی است، در ابتدا سه نمونه‌ی موازی از الگوریتم به طور موازی اجرا کنیم که هر کدام به صورت قطعی\پاورقی{Deterministic} مسیر تعیین شده را دنبال می‌کند و در هر لحظه مطمئن هستیم که حداقل یکی از سه مسیر، مسیر درستی است.

در دو زیر بخش بعدی، تغییراتی که در الگوریتم آهن و کیم برای مسئله‌ی $2$-مرکز با داده‌های پرت ارائه دادیم را بیان می‌کنیم. در بخش اول، تغییرات اصلی در الگوریتم برای تشخیص داده‌های پرت را ارائه می‌دهیم و در زیر بخش دوم، به نحوه‌ی پیدا کردن $r'$ مورد نیاز الگوریتم اصلی می‌پردازیم (تعریف $'r$ کاملا مشابه $r'$ استفاده شده در الگوریتم $1.8 + \epsilon$-تقریب برای مسئله‌ی $1$-مرکز با داده‌های پرت در همین پایان‌نامه است).

\زیرزیرقسمت{الگوریتم اصلی}

در این بخش تغییراتی که برروی الگوریتم کیم و آهن ارائه دادیم را بیان می‌کنیم. الگوریتم ارائه شده، کاملا مشابه الگوریتم ارائه شده برای مسئله‌ی $1$-مرکز با داده‌های پرت است که در همین فصل مورد بررسی قرار گرفت. تغییر اصلی الگوریتم جدید، بر روی قسمت انتقال بین حالات اعمال شده است. 

در طول اجرای الگوریتم، هر نقطه اگر داخل دو توپ کاندید قرار بگیرد باعث تغییر حالت الگوریتم نمی‌گردد. بنابراین حذف چنین نقاطی در روند اجرای الگوریتم تغییری ایجاد نمی‌کند. توجه کنید اگر یک نقطه در داخل دو توپ کاندید یک حالت قرار بگیرد، در دو توپ حالت‌هایی که از این حالت قابل رسیدن هستند نیز‌ قرار می‌گیرد، زیرا زمانی که از یک حالت به حالت جدید می‌رویم، کاندیدها به گونه‌ای تغییر می‌کنند که کاندید‌های قبلی را به طور کامل می‌پوشانند.

بنابراین تنها وجود نقاطی در جویبارداده مهم هستند که خارج توپ‌های کاندید قرار می‌گیرند. با توجه به این‌که این نقاط تنها باعث افزایش شعاع توپ‌های کاندید می‌گردند و وجودشان در روند الگوریتم تاثیر دارد، بنابراین تنها گزینه‌های مطرح برای نقاط پرت محسوب می‌شوند. 

از طرفی چون در هر حالت، تعداد نقاط پرت غیر مشخص است، مجبور هستیم تمام حالت‌های ممکن برای نقات پرت را در نظر بگیریم. چون گراف تغییر حالات\پاورقی{Transition Graph}، یک گراف جهت‌دار بدون دور با عمق $4$ است، کافی است به ازای هر عمق گراف، تعداد نقاط پرت ($n_i$) مشخص کنیم و برای در نظر گرفتن تمام حالات ممکن، تمام ۴‌-تایی‌های صحیح نامنفی $(n_1, \cdots, n_4)$  که $\sum_{i =1}^{4} n_i = z$ را در نظر بگیریم. به راحتی می‌توان نشان داد که تعداد چنین $4$-تایی‌هایی از مرتبه‌ی $\cO(z^3)$ است.

\شروع{الگوریتم}{مسئله‌ی $2$-مرکز در حالت نزدیک}
\ورودی{مجموعه نقاط $P$، عدد ثابت $r'$ در بازه‌ی $[1.2r^*, (1.2r^* + \frac{2\epsilon}{2})r^*]$ و $z$ تعداد نقاط پرت}
\دستور{مجموعه جواب $S$ را برابر $\emptyset$ قرار بده.}
\به‌ازای{هر $4$-تایی $(n_1, \cdots, n_4)$ که $\sum_{i=1}^4 n_i = z$}
\به‌ازای{هر $\pi \in \set{1, 2, 3}$}
\توضیحات{انتخاب یکی از سه مسیر مختلف}
\به‌ازای{هر $i$ از بین $\set{1, \cdots, 4}$}
\دستور{$counter_i$ را برابر صفر قرار بده.}
\پایان‌به‌ازای{}
\دستور{$B_1$ را $B(p_1, r')$ قرار بده.}
\دستور{$B_2$ را مجموعه‌ی $\emptyset$ قرار بده.}
\دستور{متغیر $j$ را برابر $1$ قرار بده.}
\توضیحات{متغیر $j$ عمق حالت را مشخص می‌کند.}
\به‌ازای{ هر $p \in P$}
\اگر{$p \not \in B_1 \cup B_2$}
\دستور{$counter_j$ را یک عدد افزایش بده.}
\اگر{$counter_j > n_j$}
\دستور{$j$ را یک عدد افزایش بده.}
\توضیحات{در مسیر $\pi$ به حالت بعدی برو}
\دستور{توپ‌های کاندید $(B_1, B_2)$ را با توپ‌های کاندید حاصل از‌ انتقال حالت مطابق‌ مسیر $\pi$ در الگوریتم کیم و آهن جایگزین کن.}
\پایان‌اگر{}
\پایان‌اگر{}
\پایان‌به‌ازای{}
\اگر{$j \leq 4$}
\دستور{شعاع توپ با شعاع بیشینه از بین دو توپ $B_1$ و $B_2$ را به مجموعه $S$ اضافه کن.}
\پایان‌اگر{}
\پایان‌به‌ازای{}
\پایان‌به‌ازای{}
\دستور{کم‌ترین شعاع داخل مجموعه‌ی $S$ را برگردان}
\پایان{الگوریتم}

شبه‌کد الگوریتم ارائه شده در الگوریتم ~\رجوع{الگوریتم: مسئله‌ی $2$-مرکز در حالت نزدیک} نشان داده شده است. به ازای تمام حالت‌های ممکن برای $n_1$ تا $n_4$ و هر مسیر مجاز از‌ بین سه مسیر موجود بین گره‌ی شروع تا پایان، الگوریتم یک جواب کاندید $(B_1, B_2)$ برای پوشش نقاط غیر پرت نگه می‌دارد. متغیر $j$، برای هر حالت، عمق آن حالت را مشخص می‌کند. چهار شمارنده نیز برای شمارش تعداد نقاطی که در هر عمق به عنوان داده‌ی پرت در نظر گرفته شده‌اند استفاده می‌شود.

الگوریتم ابتدا با دو کاندید $B_1 = B(p_1, r')$ و $B_2 = \emptyset$ شرو ع می‌کند که معادل حالت شروع الگوریتم کیم و آهن است. پس از ورود هر نقطه‌ی $p$ از جویبارداده، در ابتدا بررسی می‌شود که نقطه‌ی مورد نظر در توپ‌های کاندید قرار می‌گیرند یا نه. اگر قرار بگیرند به سراغ نقطه‌ی بعدی می‌رویم. در غیر این صورت، اگر تعداد نقاط پرت در این عمق‌(فرض کنید در عمق $j$اُم هستیم) به $n_j$ نرسیده باشد، نقطه‌ی $p$ را به عنوان داده‌ی پرت در نظر می‌گیریم و به سراغ نقطه‌ی بعدی می‌رویم. در غیر این صورت، مطابق مسیر انتخاب شده، به حالت بعدی می‌رویم و توپ‌های کاندید $(B_1, B_2)$ را مطابق با الگوریتم کیم و اهن، به‌روزرسانی می‌کنیم. توجه کنید که برای انتقال حالت مطابق الگوریتم کیم و آهن، علاوه بر نقطه‌ی $p$، ما مسیر حرکت را نیز‌ می‌دهیم که به طور قطعی، حالت بعدی مشخص شود.

زمانی که تمام نقاط $P$ پردازش شدند، اگر هنوز به گره‌ی پایان وارد نشده‌ایم، توپ‌های کاندید را به عنوان یک جواب به مجموعه جواب اضافه می‌کنیم. در غیر این صورت مطابق عملکرد الگوریتم کیم و آهن این حالت، از بین حالات موجود حذف می‌شود. در نهایت از بین تمام جواب‌های ممکن، بهترین جواب را با کم‌ترین شعاع به عنوان جواب نهایی می‌دهیم. همان‌طور کیم و آهن \مرجع{kim2014improved} ثابت کرده‌اند، جوابی که با این روش محاسبه می‌شود دارای شعاعی حداکثر $\frac{3}{2}r^*$ است، با فرض اینکه $\delta^* \leq \alpha r^*$ باشد (اثبات بیان شده برای $\alpha = 2$ بیان شده است، اما می‌توان نشان داد که به ازای هر $\alpha$ ثابتی این اثبات صادق است). بنابراین با فرض داشتن $1.2r^* \leq r' \leq (1.2 + \frac{2\epsilon}{3})r^*$، قضیه زیر برقرار است:

\شروع{قضیه}

به ازای $1.2r^* \leq r' \leq (1.2 + \frac{2\epsilon}{3})r^*$ داده شده و با فرض $\delta^* \leq \alpha r^*$، الگوریتم ~\رجوع{الگوریتم: مسئله‌ی $2$-مرکز در حالت نزدیک} یک جواب $1.8 + \epsilon$-تقریب برای جواب بهینه‌ی مسئله‌ی $2$-مرکز با $z$ داده‌ی پرت ارائه می‌دهد. حافظه‌ی مصرفی این الگوریتم از مرتبه‌ی $\cO(dz^3)$ و زمان به‌رو‌زرسانی/پاسخ‌گویی به پرسمان آن، از مرتبه‌ی $\cO(dz^3)$ است (با فرض پرت نبودن داده‌ی اول).

\پایان{قضیه}

در زیر بخش بعدی، نحوه‌ی پیدا کردن $r'$ مناسب را مورد بررسی قرار می‌دهیم.

\زیرزیرقسمت{پیدا کردن $r'$}

در این زیر بخش، نشان می‌دهیم که چگونه $r'$ مناسبی را پیدا کنیم که در رابطه‌ی 
$$1.2r^* \leq r \leq (1.2 + \frac{2\epsilon}{3})r^*$$
صدق کند. لم زیر ایده‌ی اصلی را بیان می‌کند.

\شروع{لم}
\برچسب{لم: 2lt1}
مجموعه نقاط $P$ در فضای $\IR^d$ داده شده است. یک جواب بهینه برای مسئله‌ی $1$-مرکز با $z$ داده‌ی پرت برای مجموعه نقاط $P$، با فرض $\delta^* \leq \alpha r^*$، یک $(2 + \frac{\alpha}{2})$-تقریب برای مسئله‌ی $2$-مرکز با $z$ داده‌ی پرت برای مجموعه‌ی نقاط $P$ ارائه می‌دهد.

\شروع{شکل}[ht]
\centerimg{2lt1}{10cm}
\شرح{اثبات لم ~\رجوع{لم: 2lt1}}
\برچسب{شکل: 2lt1}
\پایان{شکل}

\شروع{اثبات}

فرض کنید $r_1^*$ و $r^*$ به ترتیب شعاع بهینه برای مسئله‌ی $1$-مرکز و $2$-مرکز با $z$ داده‌ی پرت برای مجموعه نقاط $P$ باشد. به وضوح $r^* \leq r_1^*$ است، زیرا هر جواب درست $B^*$ برای مسئله‌ی $1$-مرکز با $z$ داده‌ی پرت، یک جواب درست $(B^*, B^*)$ برای مسئله‌‌ی $2$-مرکز با $z$ داده‌ی پرت ارائه می‌دهد. حال $B_1^*(c_1^*, r^*)$ و $B_2^*(c_2^*, r^*)$ را دو توپ جواب مسئله‌ی $2$-مرکز با $z$ داده‌ی پرت در نظر بگیرید.  $c$ را نقطه‌ی میانی پاره‌خط واسط مراکز $c_1^*$ و $c_2^*$ در نظر بگیرید (همان‌طور که در شکل ~\رجوع{شکل: 2lt1} نشان داده شده است). به وضوح، $B(c, \frac{\delta}{2} + 2r^*)$ هر دوی $B_1^*$ و $B_2^*$ را می‌پوشاند. بنابراین یک جواب قابل قبول برای مسئله‌ی $1$-مرکز با $z$ داده‌ی پرت است، و در نتیجه داریم:
$$r_1^* \leq (2 + \frac{\alpha}{2})r^*$$ 

\پایان{اثبات}

\پایان{لم}

حال کافی است، به طور کاملا مشابه با الگوریتم ~\رجوع{الگوریتم: الگوریتم با ضریب تقریب $1.8$ برای مسئله‌ی $1$-مرکز با $z$ داده‌ی پرت} از‌ الگوریتم ~\رجوع{الگوریتم: الگوریتم با ضریب تقریب $2$ برای مسئله‌ی $1$-مرکز با $z$ داده‌ی پرت} برای تقریب $r^*$ استفاده کنیم و با تقسیم بندی بازه‌ی $[0, 1.2r]$ به $m = \ceil{\frac{1.8}{\epsilon} \times (4 + \alpha)}$ زیربازه، تمام گزینه‌های ممکن را امتحان کنیم. با استفاده از روش ارائه شده و با حذف فرض پرت نبودن داده‌ی اول جویبار داده به قضیه‌ی زیر می‌رسیم:

\شروع{قضیه}

اگر $\delta^* \leq \alpha r^*$ باشد، یک $(1.8 + \epsilon)$-تقریب برای مسئله‌ی $2$-مرکز با $z$ داده‌ی پرت با مصرف حافظه‌ی $\cO(\frac{dz^4}{\epsilon})$ و زمان به‌روزرسانی $\cO(\frac{dz^5}{\epsilon})$ قابل ارائه است.

\پایان{قضیه}

\زیرقسمت{حالت $\delta^* > \alpha r^*$}

در این بخش، الگوریتمی با ضریب تقریب $1.8$ برای حالتی که دو توپ بهینه بیش از$\alpha r^*$  یک‌دیگر فاصله دارند. با دو مشاهده‌ی ساده شروع می‌کنیم.


\شروع{شکل}[ht]
\centerimg{c+4}{12cm}
\شرح{اثبات مشاهده‌ی ~\رجوع{مشاهده: c+4}}
\برچسب{شکل: c+4}
\پایان{شکل}

\شروع{مشاهده}
\برچسب{مشاهده: c+4}
فرض کنید که توپ $B_1$ و $B_2$، دو توپ با شعاع $r$ باشند، به‌طوری‌که که فاصله‌ای بیش‌تر از $\alpha r$ دارند. به ازای هر دو نقطه‌ی $p \in B_1$ و $q \in B_2$، داریم:
$$1 \leq \frac{\len{pq}}{\delta} < \frac{4 + \alpha}{\alpha}$$

\شروع{اثبات}

همان‌طور که در شکل ~\رجوع{شکل: c+4} می‌بینید، با استفاده از نامساوی مثلثی رابطه‌ی زیر برقرار است:
$$\len{pq} \leq \len{pc_1} + \len{c_1c_2} + \len{c_2 q} \leq r + r + \delta + r + r$$ 
از‌ طرفی با توجه با نحوه‌ی تعریف $\delta$، می‌دانیم فاصله‌ی هر زوج دلخواه از $(B_1, B_2)$ از جمله $p$ و $q$، حداقل $\delta$ است. در نتیجه داریم:
$$\delta \leq \len{pq} \leq 4r + \delta$$
که با تقسیم طرفین بر $\delta$ به حکم مسئله می‌رسیم.

\پایان{اثبات}

\پایان{مشاهده}

\شروع{مشاهده}
\برچسب{مشاهده: تقاطع}
فرض کنید $B_1$ و $B_2$ دو توپ با فاصله‌ی $\delta$ باشند و $B$ یک توپ با شعاع کم‌تر از $\frac{\delta}{2}$ باشد. آن‌گاه $B$ حداکثر با یکی از $B_1$ و $B_2$ تقاطع دارد.

\پایان{مشاهده}

در ادامه، تعدادی ویژگی برای توپ‌های بهینه‌ی $B_1^*$ و $B_2^*$ ارائه می‌دهیم.

\شروع{لم}
\برچسب{لم: $c$-جدا}

فرض کنید $B_1^*$ و $B_2^*$ دو توپ $\alpha$-جداپذیر باشند ($\alpha > 4$). اگر $p$ نقطه‌ی دلخواه از $B_1^*$ و $S$ زیرمجموعه‌ی $z+1$ عضوی از $P$ شامل دورترین نقاط از $p$ باشد، آن‌گاه $S \cap B_2^*$ ناتهی است.

\شروع{اثبات}

از برهان خلف برای اثبات استفاده می‌کنیم. با برهان خلف فرض می‌کنیم که خلاف حکم مسئله برقرار و $S \cap B_2^*$ تهی باشد. چون $|S| = z+1 > z$ است، بنابراین حداقل عضوی از‌ $S$ وجود دارد که جزء داده‌های پرت در جواب بهینه نیست و چون $S$ اشتراکش با $B_2^*$ تهی است، بنابراین آن عضو داخل $B_1^*$ قرار دارد و مجموعه‌ی $B_1^* \cap S$ ناتهی است. نقطه‌ی $q$ را دورترین عضو $B_1^* \cap S$ در نظر بگیرید. توپ $B(p, \len{pq})$ را در نظر بگیرید. به ازای هر نقطه‌ی $s \in P \setminus S$، $\len{ps} \leq \len{pq}$ است، زیرا $s \not \in S$ و $q \in S$ است. بنابراین، $B$، مجموعه‌ی $P \setminus S$ را به طور کامل می‌پوشاند. از طرفی چون $p, q \in B_1^*$ است داریم:
$$\len{pq} \leq \len{pc_1^*} + \len{c_1^* q} \leq 2r^*$$
بنابراین با توجه به مشاهده‌ی ~\رجوع{مشاهده: تقاطع}، $B_2^* \cap B$ تهی است. بنباراین $B_2^* \cap P$ تهی است و در نتیجه $B_2^*$ تهی است، که با بهینه بودن $B_1^*$ و $B_2^*$ تناقض دارد.

\پایان{اثبات}
\پایان{لم}

\شروع{لم}
\برچسب{لم: $(z+1)$-دورترین}
فرض کنید $p$ نقطه‌ای دلخواه در $B_1^*$ باشد و $q$، $z+1$-دورترین نقطه از $p$ باشد. آن‌گاه $\delta^* > \frac{\alpha}{\alpha + 4}\len{pq}$.

\شروع{اثبات}

با استفاده از لم ~\رجوع{لم: $c$-جدا}، نقطه‌ی $q' \in B_2^*$ وجود دارد به‌طوری‌که $\len{pq} \leq \len{pq'}$. بنابراین با توجه به مشاهده‌ی ~\رجوع{مشاهده: c+4} داریم:
$$\frac{\len{pq}}{\delta^*} \leq \frac{\len{pq'}}{\delta^*} < \frac{\alpha + 4}{\alpha}$$

\پایان{اثبات}
\پایان{لم}


\شروع{شکل}[ht]
\centerimg{2r}{12cm}
\شرح{اثبات لم ~\رجوع{لم: 2r}}
\برچسب{شکل: 2r}
\پایان{شکل}

\شروع{لم}
\برچسب{لم: 2r}
دو نقطه‌ی $p \in B_1^*(c_1, r^*)$ و $q \in B_2^*(c_2, r^*)$ در نظر بگیرید، آن‌گاه‌ $B_1^* \subset B(p, 2r^*)$ و $B_2^* \subset B(q, 2r^*)$ است و در نتیجه حداکثر $z$ نقطه‌ از $P$ خارج $B(p, 2r^*) \cup B(q, 2r^*)$ قرار می‌گیرد.  

\شروع{اثبات}
 
 نقطه‌ی دلخواه $p' \in B_1^*$ در نظر بگیرید. داریم:
 $$\len{pp'} \leq \len{pc_1} + \len{p'c_1} \leq 2r^*$$
 و در نتیجه $B_1^* \subset B(p, 2r^*)$ است (به شکل ~\رجوع{شکل: 2r} مراجعه کنید). به طور مشابه ثابت می‌شود $B_2^* \subset B(q, 2r^*)$ است. با توجه به این‌که حداکثر $z$ نقطه‌ی پرت خارج $B_1^*$ و $B_2^*$ قرار می‌گیرد، در نتیجه اثبات کامل است.

\پایان{اثبات}

\پایان{لم}

\شروع{لم}
\برچسب{لم: نقطه مرکزی}
فرض کنید $S$ زیرمجموعه‌ای از $P$ با اندازه‌ی حداقل $(d+1)(z+1)$ باشد که توسط توپی با شعاع کم‌تر از $\frac{\delta^*}{2}$ پوشانده می‌شود. آن‌گاه $c_p$ نقطه‌ی مرکزی نقاط $S$، یا داخل $B_1^*$ قرار می‌گیرد یا داخل $B_2^*$ قرار می‌گیرد.

\شروع{اثبات}

با توجه به این که اندازه‌ی $S$ بیش‌تر از $z$ است، حداقل یک نقطه‌ی غیر پرت داخل $S$ قرار دارد. بنابراین با توجه به مشاهده‌ی ~\رجوع{مشاهده: تقاطع}، $B$ دقیقا با یکی از $B_1^*$ یا $B_2^*$ تقاطع دارد. بدون کم شدن از کلیت مسئله، فرض کنید $B$ با توپ $B_1^*$ تقاطع دارد. حال اگر $c_p \not \in B_1^*$ نباشد، بنابراین طبق مشاهده‌ی ~\رجوع{مشاهده: نقطه‌ی مرکزی} حداقل $ z+1$ نقطه‌ی دیگر خارج $B_1^*$ قرار می‌گیرند، که وجود حداکثر $z$ داده‌ی پرت را نقض می‌کند. بنابراین فرض $c_p \not \in B_1$ اشتباه بوده و حکم ثابت شد.

\پایان{اثبات}

\پایان{لم}

\زیرزیرقسمت{الگوریتم اصلی}

در این قسمت، الگوریتم اصلی برای حالتی که $\delta^* > \alpha r^*$ است. در هر لحظه، الگوریتم نقاط $P$ از جویبارداده را به سه دسته‌ی مجزای $B_1$، $B_2$ و حافظه‌ی میان‌گیر \لر{Buffer} افراز می‌کند. بدون کم شدن از کلیت مسئله، فرض کنید $p_1 \in B_1^*$ است. الگوریتم سعی می‌کند نقا‌ط را به گونه‌ای به سه دسته افراز کند که $B_1$ به طور کامل $B_1^*$ را بپوشاند و $B_2$، $B_2^*$ را به طور کامل بپوشاند و \لر{Buffer} تنها شامل تعدادی از نقاط پرت در جواب بهینه است. توجه کنید که $B_1$ و $B_2$ علاوه بر پوشش توپ متناظر در جواب بهینه، ممکن است تعدادی از نقاط پرت را نیز شامل شوند. 

با شروع الگوریتم، مرکز توپ $B_1$ را که با $c_1$ نشان می‌دهیم برابر $p_1$ قرار می‌دهد و $c_2$ را از میان نقاطی که تا کنون پردازش‌شده است به عنوان کاندیدی برای مرکز $B_2$ انتخاب می‌کند. الگوریتم همچنین دو متغیر $\delta$ و $r$ را در طول جویبارداده به‌روزرسانی می‌کند به طوری که در هر لحظه، $\delta$ کران پایینی برای $\delta^*$ است و $r$ تحت شرایطی که در ادامه گفته می‌شود، کران بالایی برای $2r^*$ است.
 
\شروع{الگوریتم}{الگوریتم برای $2$-مرکز در حالت دور}
\دستور{$c_1$ را برابر $p_1$ قرار بده.}
\دستور{$r$ و $\delta$ را برابر صفر قرار بده.}
\به‌ازای{هر نقطه‌ی $p \in P$}
\اگر{$p_1$ قابل اضافه شدن به $B_1$ و $B_2$ نبود}
\توضیحات{از دو تابع \لر{AddTo$B_1$} و \لر{AddTo$B_2$} به‌ترتیب برای اضافه کردن نقطه به $B_1$ و $B_2$ استفاده کنید.}
\دستور{$p$ را به \لر{Buffer} اضافه کن.}
\تاوقتی{$|\text{\lr{Buffer}}| \geq z$}
\اگر{$|B_2| \geq (d+1)(z+1)$}
\دستور{$B_1$ را برابر $B_1 \cup B_2$ قرار بده.}
\دستور{$B_2$ را برابر مجموعه‌ی تهی قرار بده.}
\وگرنه{}
\اگر{$c_2$ مشخص شده است}
\دستور{$c_2$ را به $B_1$ اضافه کن.}
\پایان‌اگر{}
\پایان‌اگر{}
\دستور{متغیر محلی $T$ را برابر $\text{\lr{Buffer}} \cup B_2 \setminus \set{c_2}$ قرار بده.}
\دستور{$B_2$ را تهی قرار بده.}
\دستور{$c_2$ را $(z+1)$-امین دورترین نقطه‌ از $c_1$ در $T$ قرار بده.}
\دستور{$r$ را برابر با $\frac{2}{\alpha}\len{c_1 c_2}$ قرار بده.}
\به‌ازای{هر $p \in T$}
\دستور{$p$ را به $B_2$ اضافه کن.}
\پایان‌به‌ازای{}
\دستور{\لر{Buffer} را برابر $T \setminus B_2$ قرار بده.}
\پایان‌تاوقتی{}
\پایان‌اگر{}
\پایان‌به‌ازای{}
\پایان{الگوریتم}

شبه کد الگوریتم، در الگوریتم ~\رجوع{الگوریتم: الگوریتم برای $2$-مرکز در حالت دور} آورده شده است. به محض ورود نقطه‌ی $p$ از جویبارداده‌ی $P$، الگوریتم ارائه شده، سعی می‌کند آن را به توپ $B_1$ یا $B_2$ اضافه کند. عمل اضافه کردن نقاط به توپ‌های $B_1$ و $B_2$ به ترتیب به وسیله‌ی توابع \لر{AddTo$B_1$} و \لر{AddTo$B_2$} انجام می‌شود. اگر نقطه‌ی $p$، در هیچ کدام از توپ‌ها قرار نگیرد، به \لر{Buffer} اضافه می‌شود. تابع \لر{AddTo$B_1$} نقطه‌ی $p$ را به $B_1$ اضافه می‌کند اگر فاصله‌ی $p$ از $c_1$ کم‌تر مساوی $\delta$ باشد. تابع \لر{AddTo$B_2$} نقطه‌ی $p$ را به مجموعه‌ی $B_2$ اضافه می‌کند اگر نقطه‌ی $p$ از $c_2$ حداکثر $r$ فاصله داشته باشند. هر دوی توابع، مقدارهای $\delta$ و $r$ را در صورت لزوم تغییر می‌دهند که ناورداها‌ی داخل لم ~\رجوع{لم: ناوردا} برقرار بماند.


\شروع{الگوریتم}{تابع اضافه‌کننده نقطه به $B_1$}
\اگر{حداقل $z+1$ نقطه پردازش شده است}
\دستور{$q$ را $(z+1)$-دورترین نقطه از $c_1$ در نقاطی که تا کنون آمده‌اند در نظر بگیر.}
\وگرنه{}
\دستور{$q$ را برابر $c_1$ در نظر بگیرد.}
\پایان‌اگر{}
\دستور{$\delta$ را برابر $\frac{\alpha}{\alpha + 4}\len{c_1 q}$ قرار بده.}
\اگر{$p \in B(c_1, \delta)$}
\دستور{$p$ را به مجموعه‌ی $B_1$ اضافه کن.}
\دستور{\برگردان \لر{true}}
\پایان‌اگر{}
\دستور{\برگردان \لر{false}}
\پایان{الگوریتم}

زمانی که \لر{Buffer} سرریز می‌کند (در الگوریتم  ~\رجوع{الگوریتم: الگوریتم برای $2$-مرکز در حالت دور})، الگوریتم یکی از دو عملیات زیر را وابسته به این‌که اندازه‌ی $B_2$ چقدر است، انجام می‌دهد. اگر اندازه‌ی $|B_2|$ بزرگ‌تر مساوی $(d+1)(z+1)$ باشد، تمام اعضای $B_2$ را به $B_1$ اضافه می‌شود و $B_2$ خالی می‌گردد. در غیر این صورت، $c_2$ در صورتی که قبلا تعیین شده باشد، به $B_1$ اضافه می‌شود و نقطه‌ی دیگری از $B_2 \cup \text{\lr{Buffer}} \setminus \set{c_2}$ به عنوان $c_2$ جدید انتخاب می‌گردد. حلقه‌ی مذکور، حداکثر $\cO(dz)$ بار اجرا می‌شود، زیرا پس از اولین اجرا، مطمئن هستیم که $T$ حداکثر $(d+1)(z+1) + z$ نقطه دارد و در هر مرحله حداقل یک عنصر از آن حذف می‌گردد ($c_2$). توجه داشته باشید که هر نقطه حداکثر یک‌بار به عنوان $c_2$ انتخاب می‌شود، بنابراین حلقه‌ی مذکور، به طور سرشکن\پاورقی{Amortized} یک‌بار به ازای هر نقطه از جویبارداده اجرا می‌شود.

برای تحلیل، علاوه بر $c_2$، نیاز به نقطه‌ی مرکزی به نام $c_p$ داریم. زمانی که $|B_2| < (d+1)(z+1)$، آن‌گاه $c_p$ همان $c_2$ است و در غیر این صورت، $c_p$ را نقطه‌ی مرکزی $(d+1)(z+1)$ نقطه‌ی اولی که به $B_2$ اضافه می‌شوند قرار می‌دهیم.

\شروع{الگوریتم}{تابع اضافه‌کننده نقطه به $B_2$}
\اگر{$c_2$ تعیین شده باشد و $p \in B(c_2, r)$}
\دستور{$p$ را به $B_2$ اضافه کن.}
\اگر{$|B_2| = (d+1)(z+1)$}
\دستور{$r$ را $(2 + \frac{2}{\alpha})$ برابر کن.}
\به‌ازای{$p \in B(c_2, r)$}
\اگر{$p \in B(c_2, r)$}
\دستور{$p$ را به $B_2$ اضافه کن.}
\دستور{$p$ را از \لر{Buffer} حذف کن.}
\پایان‌اگر{}
\پایان‌به‌ازای{}
\پایان‌اگر{}
\وگرنه{}
\دستور{\برگردان \لر{true}}
\پایان‌اگر{}
\دستور{\برگردان \لر{false}}

\پایان{الگوریتم}

\شروع{لم}
\برچسب{لم: ناوردا}

ثابت‌های حلقه‌ی زیر در طول اجرای الگوریتم حفظ می‌شوند:

\شروع{شمارش}

\فقره{$\delta < \delta^*$}

\فقره{$r \leq \frac{\delta}{2}$}

\فقره{$B_1 \cap B_2^* = \emptyset$}

\فقره{اگر $c_p \in B_2*$ باشد، آن‌گاه: 

\شروع{شمارش}

\فقره{$2r^* \leq r$}

\فقره{$B_2 \cap B_2^* = \emptyset$}

\فقره{تمام نقاط داخل \لر{Buffer} در جواب بهینه داده‌ی پرت هستند.}

\پایان{شمارش}}

\پایان{شمارش}

\شروع{اثبات}

\شروع{شمارش}

\فقره{در ابتدای اجرای الگوریتم $\delta = 0$ است که به وضوح حکم برقرار است. بعد از این‌که $z+1$ نقطه از جویبار داده پردازش می‌شود، تابع \لر{AddTo$B_1$} مقدار $\delta$ را به $\frac{\alpha}{\alpha+4}\len{c_1q}$ افزایش‌می‌دهد، که در آن $q$، $(z+1)$-دورترین نقطه از $c_2$ در جویبارداده است. از طرفی چون $c_1 \in B_1^*$ است، طبق لم ~\رجوع{لم: $(z+1)$-دورترین} داریم $\delta < \delta^*$}

\فقره{زمانی که متغیر $c_2$ در الگوریتم  ~\رجوع{الگوریتم: الگوریتم برای $2$-مرکز در حالت دور} تعیین می‌شود، $(z+1)$-دورترین نقطه از $c_1$ در بین اعضای $T \subset P$ است و $r$ برابر $\frac{2}{\alpha}\len{c_1 c_2}$ است. اگر $q$، $(z+1)$-دورترین نقطه از $c_1$ در جویبارداده در آن لحظه باشد، آن‌گاه $\len{c_1 c_2} \leq \len{c_1 q}$ است. با فرض این‌که $\alpha \geq 16$ باشد، داریم:
$$\frac{2}{\alpha} \len{c_1 c_2} \leq \frac{1}{6} \times \frac{\alpha \len{c_1 q}}{\alpha + 4} \leq \frac{\delta}{6} $$
و در نتیجه:
$$r \leq (2 + \frac{2}{\alpha}) \times \frac{2}{\alpha} \len{c_1 c_2} \leq 3 \times \frac{2}{\alpha} \len{c_1 c_2} \leq \frac{\delta}{2}$$
که نشان می‌دهد که صورت ناوردا درست است، حتی بعد از افزایشی که در تابع \لر{AddTo$B_2$} می‌یابد.
}

\فقره{در ابتدا لم زیر را ثابت می‌کنیم:

\شروع{ادعا}
\برچسب{ادعا: افزایش شعاع}
اگر $c_2$ تعیین شده باشد، آن‌گاه $B(c_p, \frac{2}{\alpha}\len{c_1 c_p}) \subset B_2(c_2, r)$ است.

\شروع{اثبات}

اگر $|B_2| < (d+1)(z+1)$ باشد، با توجه به این‌که $c_p = c_2$ است و $r = \frac{2}{\alpha}\len{c_1 c_2}$، در نتیجه $B_2 = B(c_p, \frac{2}{\alpha}\len{c_1 c_p})$ است و حکم برقرار است.
در حالتی که اندازه‌ی $B_2$ به $(d+1)(z+1)$ می‌رسد، نقطه‌ی $c_p$ به نقطه‌ی مرکزی نقاط $B_2$ انتقال می‌یابد و $r$، $(2 + \frac{2}{\alpha})$ برابر می‌گردد.
از طرفی چون نقطه‌ی مرکزی نقاط داخل $B_2$، داخل $B_2$ قرار می‌گیرد، بنابراین $c_p \in B(c_2, \frac{2}{\alpha}\len{c_1, c_2})$ است.
در نتیجه زمانی داریم:
$$\len{c_2 c_p} \leq \frac{2}{\alpha} \len{c_1 c_2}$$
و در نتیجه داریم:
$$\frac{2}{\alpha} \len{c_1 c_p} \leq \frac{2 (\len{c_1 c_2} + \len{c_2 c_p})}{\alpha} \leq \frac{2}{\alpha} \len{c_1 c_2} (1 + \frac{2}{\alpha})$$
که نتیجه می‌دهد:
$$B(c_p, \frac{2}{\alpha} \len{c_1 c_p}) \subset B_2(c_2, \frac{2}{\alpha} \len{c_1 c_2} (2 + \frac{2}{\alpha}))$$

\پایان{اثبات}

\پایان{ادعا}

حال با استفاده از ادعای ~\رجوع{ادعا: افزایش شعاع}، حکم را ثابت می‌کنیم.
نقطه‌ی $p$ از جویبار داده را در نظر بگیرید که به $B_1$ اضافه شده است.
این نقطه در دو شرایط می‌تواند به $B_1$ اضافه شده باشد.
حالت اول، زمانی است که تابع \لر{AddTo$B_1$} صدا زده می‌شود.
در این تابع، نقطه‌ی $p$ تنها زمانی به $B_1$ اضافه می‌شود که در فاصله‌ی $\delta$ از $c_1$ قرار داشته باشد.
با توجه به ناوردای $1$، مطمئن هستیم که $\delta < \delta^*$ و در نتیجه $\len{c_1 p} < \delta^*$ است و در نتیجه $p \not \in B_2^*$.
در حالت دوم، در الگوریتم ~\رجوع{الگوریتم: الگوریتم برای $2$-مرکز در حالت دور}، زمانی که حافظه‌ی میان‌گیر \لر{Buffer} سر ریز می‌کند و $B_2$ خالی نیست، الگوریتم وابسته به اندازه‌ی $B_2$ عمل می‌کند.
اگر اندازه‌ی $B_2$ هنوز به $(d+1)(z+1)$ نرسیده باشد، آن‌گاه $c_2 = c_p$ است.
در این حالت، الگوریتم \رجوع{ادعا: افزایش شعاع}، $c_2$ را به $B_1$ اضافه می‌کند.
با فرض خلف فرض کنید که $c_p \in B_2^*$ باشد. 
آن‌گاه با توجه ناوردای $2$ و $4$ قسمت (آ) داریم:
$$2r^* \leq r \leq \frac{\delta}{2} \leq \frac{\delta^*}{2}$$
در نتیجه با توجه به لم ~\رجوع{لم: 2r}، حداکثر باید $z$ نقطه خارج $B_1 \cup B_2$ قرار بگیرد، که با سرریز شدن حافظه‌ی میان‌گیر \لر{Buffer} تناقض دارد.
در حالتی که $|B_2| \geq (d+1)(z+1)$ است، آن‌گاه $c_p$، نقطه‌ی مرکزی $(z+1)(d+1)$ اولین نقاطی است که به $B_2$ اضافه شده‌اند.
در این حالت، تمام نقاط $B_2$ به $B_1$ اضافه می‌شود.
با استفاده از ناوردای $2$، داریم:
$$r \leq \frac{\delta}{2} < \frac{delta^*}{2}$$
در نتیجه، طبق لم ~\رجوع{لم: نقطه مرکزی}، $c_p \in B_1^*$ یا $c_p \in B_2^*$ است.
با فرض خلف، فرض کنید که $c_p \in B_2^*$ است.
در این حالت، با توجه به ناوردای $4$ قسمت (آ)، و ادعای ~\رجوع{ادعا: افزایش شعاع}، $B_2$ توپ $B(c_p, \frac{2}{\alpha}\len{c_1 c_p})$ را می‌پوشاند و $2r^* \leq r$ است.
بنابراین کاملا مشابه حالتی که $|B_2| < (d+1)(z+1)$، با سرریز شدن حافظه‌ی میان‌گیر \لر{Buffer} تناقض دارد و در نتیجه $c_p \in B_1^*$ و در نتیجه $B_2 \cap B_2^* = \emptyset$ است و اضافه کردن آن به $B_1$ صورت ناوردا را نقض نمی‌کند. 
}

\فقره{
\شروع{شمارش}

\فقره{اگر $c_1 \in B_1^*$ باشد و $c_p \in B_2^*$ باشد، طبق مشاهده‌ی ~\رجوع{مشاهده: تقاطع}، اگر $c_1 \in B_1^*$ باشد و $c_p \in B_2^*$ باشد، آن‌گاه 
$$1 \leq \frac{\len{c_1 c_p}}{\delta^*} \leq \frac{\len{c_1 c_p}}{\alpha r^*}$$
و در نتیجه:
$$2r^* \leq \frac{2}{\alpha} \len{c_1 c_p}$$
اگر $|B_2|  < (d+1)(z+1)$ باشد، $c_p = c_2$ است و با توجه به الگوریتم ~\رجوع{الگوریتم: الگوریتم برای $2$-مرکز در حالت دور}، $r = \frac{2}{\alpha} \len{c_1 c_2}$ است و در نتیجه $2r^* \leq r$ است. 
اگر $|B_2| < (d+1)(z+1)$ باشد، مشابه با ناردای $2$ داریم:
$$2r^* \leq \frac{2}{\alpha} \len{c_1 c_p} \leq (1 + \frac{2}{\alpha}) \frac{2}{\alpha} \len{c_1 c_2} \leq (2 + \frac{2}{\alpha}) \frac{2}{\alpha} \len{c_1 c_2} = r$$
}
\فقره{
بر اساس ناوردای $4$ قسمت (آ)، اگر $c_p \in B_2^*$ باشد، در نتیجه داریم:
$$2r^* \leq r \leq \frac{\delta}{2}$$
 و $c_p \in B_2$ است. حال با توجه به ناوردای $1$ و مشاهده‌ی ~\رجوع{مشاهده: تقاطع}، $B_2$ تنها $B_2^*$ را قطع می‌کند و در نتیجه $B_2 \cap B_1^* = \emptyset$.
}

\فقره{
با توجه به ناوردای $3$ و ناوردای $4$ قسمت (آ) داریم:
$$2r^* \leq r \leq \frac{\delta}{2} <‌\frac{\delta^*}{2}$$
و در نتیجه با توجه به لم ~\رجوع{لم: 2r}، تمام نقاط داخل حافظه‌ی میان‌گیر \لر{Buffer} یا خارج از $B_1 \cup B_2$ داده‌ی پرت هستند.
}

\پایان{شمارش}
}

\پایان{شمارش}

\پایان{اثبات}


\پایان{لم}

\زیرزیرقسمت{پاسخ‌گویی به پرسمان‌ها}

در این قسمت، نشان می‌دهیم با تقسیم‌بندی که الگوریتم \رجوع{الگوریتم: الگوریتم برای $2$-مرکز در حالت دور} در طول اجرای الگوریتم نگه می‌دارد، چگونه به پرسمان‌های همانند پرسمان زیر پاسخ می‌دهد.

\شروع{فقرات}

\فقره{اگر بدانیم دو توپ بهینه‌ی جواب مسئله‌ی $2$-مرکز با $z$ داده‌ی پرت، $\alpha$-جداپذیر باشند، دو توپ هم‌شعاع پیدا کنید که همه‌ی نقاط به غیر ار حداکثر $z$ نقطه از نقاطی که تاکنون در جویبار داده آمده‌اند را بپوشاند.}

\پایان{فقرات}


\شروع{الگوریتم}{پاسخ‌گویی به پرسمان}
\دستور{مجموعه‌ی \لر{solutions} را برابر 
$\set{\Call{MinCover}{B_1, B_2, \text{\lr{Buffer}}}}$ 
قرار بده.}
\اگر{$\card{B_2} < (d+1)(z+1)$ است}
\دستور{مجموعه‌ی \لر{candidates} را برابر $\text{\lr{Buffer}}$ قرار بده.}
\دستور{مجموعه‌ی $B_1$ را برابر $B_1 \cup B_2$ قرار بده.}
\دستور{مجموعه‌ی $B_2$ را تهی کن.}
\وگرنه{}
\دستور{مجموعه‌ی \لر{candidates} را برابر $B_2 \cup \text{\lr{Buffer}} \setminus \set{c_2}$ قرار بده.}
\پایان‌اگر{}
\به‌ازای {هر $c \in \text{\lr{candidates}}$}
\دستور{$r$ را برابر $\frac{2}{\alpha} \len{c_1 c}$ قرار بده.}
\دستور{$B'_1$ را برابر $B_1$ قرار بده.}
\دستور{$\delta$ را برابر $\max\set{ \delta_0, r}$ قرار بده.}
\دستور{$B_2'$ و حافظه‌ی میان‌گیر \لر{Buffer$'$} را برابر مجموعه‌ی تهی قرار بده.}
\به‌ازای{هر $p \in \text{\lr{candidates}}$}
\اگر{اگر نقطه‌ی $p$ به $B_1'$ و $B_2'$ اضافه نشد}
\دستور{$p$ را به حافظه‌ میان‌گیر \لر{Buffer$'$} اضافه کن.}
\پایان‌اگر{}
\پایان‌به‌ازای{}
\دستور{$\Call{MinCover}{B'_1, B'_2, \text{\lr{Buffer}}'}$ را به مجموعه‌ی \لر{solutions} اضافه کن.}
\پایان‌به‌ازای{}
\دستور{کم‌ترین عضو مجموعه‌ی \لر{solutions} را برگردان.}
\پایان{الگوریتم}
