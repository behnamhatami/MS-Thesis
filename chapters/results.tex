
\فصل{نتایج جدید}

در این فصل نتایج جدید به‌دست‌آمده در پایان‌نامه توضیح داده می‌شود. این فصل به سه بخش تقسیم می‌شود. قسمت اول به بیان مقدمات و نمادگذاری‌های مورد نیاز برای بخش‌های بعدی می‌پردازیم. در فصل دوم، به بررسی راه‌حل‌های ارائه‌شده برای مسئله‌ی $1$-مرکز با $z$ داده‌ی پرت در حالت جویبار داده می‌پردازیم. این بخش به سه زیربخش تقسیم می‌شود. در زیربخش اول، دو الگوریتم جدید ارائه می‌دهیم. الگوریتم اول، با حافظه‌ی مصرفی $\cO(z^2d)$ و ضریب تقریب $2$ ارائه می‌دهد که حافظه‌ی مصرفی الگوریتم ضرابی‌زاده و سایرین \مرجع{zarrabi2009streaming} را در صورتی که ابعاد فضا بیش‌تر از $z$ باشد بهبود می‌بخشد. از طرفی الگوریتم ارائه شده بسیار ساده‌تر از الگوریتم ضرابی‌زاده است. الگوریتم دوم یک الگوریتم با ضریب تقریب $1.8 + \epsilon$ و حافظه‌ی مصرفی $\cO(\frac{z^2}{\epsilon})$ است. در زیر بخش دوم، به بررسی مسئله‌ی $1$-مرکز پوشاننده بدون داده‌ی پرت می‌پردازیم، به‌طوری‌که نه تنها می‌خواهیم تمام نقاط ورودی پوشیده شود، بلکه می‌خواهیم تمام دایره‌ی بهینه نیز به طور کامل پوشیده شود. برای این مسئله، دو راه‌حل متفاوت یکی با ضریب تقریب $1.8$ و دیگری با ضریب تقریب $1.7$ ارائه می‌شود. در زیربخش بعدی، با استفاده از زیربخش قبلی، الگوریتمی با ضریب تقریب $1.7$ برای مسئله‌ی $1$-مرکز با $z$ داده‌ی پرت ارائه می‌دهیم(برای حالتی که $z$ ثابت است). 

در بخش سوم، به بررسی مسئله‌ی $2$-مرکز با $z$-داده‌ی پرت می‌پردازیم. در این قسمت با تقسیم‌بندی مسئله به دو حالت، دو الگوریتم متفاوت ارائه داده و آن‌ها به صورت موازی اجرا می‌کنیم. هر دوی این الگوریم‌ها ضریب تقریب $1.8 + \epsilon$ دارند و حافظه‌ی مصرفی در کل برابر است با 
$\cO(dz^2 (d^2 + \frac{z}{\epsilon}))$.
این اولین الگوریتم ارائه‌شده‌ایست که بعد از الگوریتمی که برای $k$ کلی، با ضریب تقریب $4+\epsilon$ ارائه شده است، که بهبود قابل توجهی محسوب می‌شود.

\قسمت{نماد‌گذاری‌ها و تعاریف اولیه}

در این فصل، تعدادی نمادگذاری که در فصول آتی مورد استفاده قرار می‌گیرند را بیان می‌کنیم. هم‌چنین بعضی از مفاهیم و تعاریف رایج که به تکرار مورد استفاده قرار می‌گیرند نیز در این فصل مورد بررسی قرار می‌گیرد.


\شروع{شکل}[ht]
\centerimg{delta-definition}{10cm}
\شرح{تعریف فاصله‌ی دو توپ دلخواه}
\برچسب{شکل:دلتا}
\پایان{شکل}


\شروع{فقرات}

\فقره{در طول متن، برای مشخص کردن یک توپ از نماد $B(c, r)$ استفاده می‌کنیم که $c$ مرکز توپ و $r$ شعاع آن را مشخص می‌کند. هر جا خواستیم به شعاع توپی ارجاع دهیم از نماد $r(B)$ و هر گاه خواستیم به مرکز یک توپ اشاره کنیم از نماد $c(B)$ استفاده می‌کنیم.} 
\فقره{به ازای هر دو نقطه‌ی دلخواه $p$ و $q$ در فضا، فاصله‌ی $p$ و $q$ را با $\len{pq}$ نشان می‌دهیم.}
\فقره{همان‌طور که در شکل ~\رجوع{شکل:دلتا} نشان داده شده است، دو توپ دلخواه $B(c, r)$ و $B'(c', r')$ را در نظر بگیرید. فاصله‌ی دو توپ $B$ و $B'$ مطابق‌ زیر تعریف می‌شود:
$$\delta(B, B') = \max \{ 0, \len{cc'} - r - r'\}$$
}
\فقره{دو توپ دلخواه $B(c, r)$ و $B'(c', r')$ را $\alpha$-تفکیک‌شده گوییم اگر داشته باشیم:
$$\delta(B, B') \geq \alpha . \max \{ r(B), r(B') \}$$
 }
 
 \فقره{زمانی که می‌خواهیم در مورد توپ بهینه‌ی $1$-مرکز صحبت کنیم از $B(c^*, r^*)$ یا $MEB(C^*, r^*)$ استفاده می‌کنیم. در حالت دو مرکز نیز از $B_1^*(c_1^*, r^*)$ و $B_2^*(c_2^*, r^*)$ برای نشان دادن دو دایره‌ی بهینه‌ مسئله‌ی $2$-مرکز برای مجموعه‌ای از نقاط استفاده می‌کنیم.}
 
 \فقره{مجموعه‌ نقاط‌ $P$ داده شده است. $k$-دورترین نقطه از $p \in P$ نقطه‌ای است که فاصله‌اش نقطه‌ی $p$، $k$-امین بزرگ‌ترین فاصله را در بین تمام نقاط‌ $P$ داشته باشد.}
\پایان{فقرات}

علاوه بر نمادگذاری‌های بالا از بعضی از تعاریف رایج در هندسه‌ی محاسباتی نیز‌ استفاده می‌شود. فرض کنید مجموعه‌ی $n$-عضوی $P$ از نقاط در $\IR^d$ داده شده است. نقطه‌ی $c \in \IR^d$ را نقطه‌ی مرکز\پاورقی{Centerpoint} مجموعه‌ی $P$ می‌گویند، اگر هر نیم‌صفحه‌ی\پاورقی{Half Space} که شامل $c$ باشد، حداقل شامل $\ceil{\frac{n}{d+1}}$ از نقاط $P$ باشد. مرجع ~\مرجع{danzer1963helly} ثابت کرده است که هر مجموعه‌ی متناهی از نقاط در فضای $d$-بعدی، دارای یک نقطه‌ی مرکزی است. مشاهده‌ی زیر نتیجه‌ی مستقیم این گزاره است.

\شروع{مشاهده}

مجموعه‌ی $P$ با $k(d+1)$ نقطه در فضای $d$-بعدی داده شده است. هر شکل محدب که شامل نقطه‌ی مرکزی $P$ نباشد، حداقل $k$ نقطه‌ی از $P$ را نیز نمی‌پوشاند.

\پایان{مشاهده}

در این پایان‌نامه، فرض می‌کنیم ذخیره‌ی هر بعد از یک نقطه‌ حافظه‌ی ثابتی مصرف می‌کند. در نتیجه، ذخیره‌سازی یک نقطه در فضای $d$-بعدی $\cO(d)$ حافظه مصرف می‌کند و عملیات عادی بر روی نقاط نیز‌ $\cO(d)$ زمان می‌برد.

\قسمت{مسئله‌ی $1$-مرکز در حالت جویبار داده}

در این بخش به بررسی گونه‌های مختلفی از مسئله‌ی $1$-مرکز در حالت جویبار داده می‌پردازیم. در این فصل، که شامل سه زیرقسمت است، در زیرقسمت اول مسئله‌ی $1$-مرکز با داده‌های پرت را مورد بررسی قرار داده می‌شود. در زیرقسمت دوم به بررسی مسئله‌ی $1$-مرکز پوشاننده می‌پردازد و در نهایت مسئله‌ی $1$-مرکز با تعدادی ثابت داده‌ی پرت، را مورد بررسی قرار می‌دهد. در هر سه قسمت‌، الگوریتم‌های قبلی از جنبه‌ یا جنبه‌هایی بهبود داده شده‌اند. مهم‌ترین معیار‌های مطرح، ضریب‌تقریب و حافظه‌ی مصرفی است که در هر الگوریتم به دقت محاسبه شده‌اند و با کارهای قبلی مقایسه شده‌اند.

\زیرقسمت{مسئله‌ی $1$-مرکز با داده‌های پرت در حالت جویبار داده}

\زیرزیرقسمت{الگوریتم تقریبی با ضریب تقریب $2$}

در این قسمت، یک الگوریتم ساده‌ی جویبارداده با ضریب تقریب $2$ برای مسئله‌ی $1$-مرکز با داده‌ی پرت ارائه می‌دهیم. در این الگوریتم، از ایده‌ی موازی‌سازی\پاورقی{Parallelization} استفاده می‌شود که به وفور در بخش‌های آتی مورد استفاده قرار می‌گیرد. در الگوریتم ~\رجوع{الگوریتم: الگوریتم با ضریب تقریب $2$ برای مسئله‌ی $1$-مرکز با $z$ داده‌ی پرت} شبه‌کد\پاورقی{Pseudocode} الگوریتم ارائه‌شده آمده است. الگوریتم، جویبارداده‌ی $P$ و $z$ تعداد داده‌های پرت را از ورودی دریافت می‌کند. همان‌طور که می‌بینید الگوریتم فرض کرده است که نقطه‌ی اول جویبار داده، داده‌ی پرت نیست. در ادامه نشان خواهیم داد چگونه چنین فرضی را حذف نماییم. در نهایت الگوریتم یک توپ $B$ را بر می‌گرداند که همه‌ی نقاط $P$ به غیر از حداکثر $z$ نقطه را می‌پوشاند.

\شروع{الگوریتم}{الگوریتم با ضریب تقریب $2$ برای مسئله‌ی $1$-مرکز با $z$ داده‌ی پرت}
\دستور{$c$ را اولین نقطه از جویبارداده‌ی $P$ قرار بده.}
\دستور{توپ $B(c, 0)$ را در نظر بگیر.}
\دستور{حافظه‌ی میان‌گیر خالی $Q$ را در نظر بگیر.}
\به‌ازای{هر $p$ در مجموعه‌ی $P$}
\اگر{$p \not \in B$}
\دستور{$p$ را به $Q$ اضافه کن.}
\اگر{$|Q| = z + 1$}
\دستور{$q$ را نزدیک‌ترین نقطه‌ی $Q$ به مرکز $c$ در نظر بگیر.}
\دستور{$q$ را از $Q$ حذف کن.}
\دستور{$B$ را با توپ $B(c, \len{cq})$ جایگزین کن.}
\پایان‌اگر
\پایان‌اگر
\پایان‌به‌ازای{}
\دستور{$B$ را برگردان}
\پایان{الگوریتم}

\شروع{قضیه}
\برچسب{قضیه: $1$-مرکز $2$-تقریب}
الگوریتم  ~\رجوع{الگوریتم: الگوریتم با ضریب تقریب $2$ برای مسئله‌ی $1$-مرکز با $z$ داده‌ی پرت}  با فرض این‌که نقطه‌ی اول جویبار داده، داده‌ی پرت نیست یک الگوریتم تقریبی با ضریب تقریب ۲ برای مسئله‌ی $1$-مرکز با $z$ داده‌ی پرت است.
\شروع{اثبات}

فرض کنید $B^*(c^*, r^*)$ توپ جواب بینه باشد و $c$ در جواب بهینه، طبق فرض مسئله، جزء نقاط پرت نباشد. بنابراین $c$ به وسلیه‌ی $B^*$ پوشیده می‌شود. به ازای هر نقطه‌ی دلخواه $p \in B^*$ داریم:
$$\len{cp} \leq \len{cc^*} + \len{c^*p} \leq 2r^*$$
از طرفی، از بین $z+1$ دورترین نقطه از $c$، حداقل یک نقطه به نام $q$ وجود دارد که در جواب بهینه داده‌ی پرت نیست و در نتیجه داخل $B^*$ قرار دارد (به شکل ~\رجوع{شکل: $1$-مرکز $2$-تقریب} نگاه کنید). در نتیجه $\len{cq} \leq 2r^*$. از طرفی چون شعاع جواب الگوریتم ~\رجوع{الگوریتم: الگوریتم با ضریب تقریب $2$ برای مسئله‌ی $1$-مرکز با $z$ داده‌ی پرت} به اندازه‌ی فاصله‌ی $c$ از‌ $z+1$-دورترین نقطه‌ از آن است، بنابراین شعاع جواب الگوریتم نیز از $2r^*$ کم‌تر مساوی است و بنابراین الگوریتم ~\رجوع{الگوریتم: الگوریتم با ضریب تقریب $2$ برای مسئله‌ی $1$-مرکز با $z$ داده‌ی پرت} یک الگوریتم $2$-تقریب برای مسئله‌ی $1$-مرکز با $z$ داده‌ی پرت است.

\پایان{اثبات}

\پایان{قضیه}


\شروع{شکل}[ht]
\centerimg{2-approx-1-center}{10cm}
\شرح{اثبات قضیه‌ی ~\رجوع{قضیه: $1$-مرکز $2$-تقریب}}
\برچسب{شکل: $1$-مرکز $2$-تقریب}
\پایان{شکل}


\زیرزیرقسمت{الگوریتم تقریبی با ضریب تقریب $1.8$}

\زیرقسمت{مسئله‌ی $1$-مرکز‌ پوشاننده در حالت جویبار داده}

\زیرقسمت{مسئله‌ی $1$-مرکز با داده‌های پرت در حالت جویبار داده با تعداد داده‌های پرت ثابت}

\قسمت{مسئله‌ی $2$-مرکز با داده‌های پرت در حالت جویبار داده}

\زیرقسمت{حالت $\delta^* \leq \alpha r^*$}

\زیرزیرقسمت{الگوریتم اصلی}

\زیرزیرقسمت{پیدا کردن $r'$}

\زیرقسمت{حالت $\delta^* \geq \alpha r^*$}

\زیرزیرقسمت{الگوریتم اصلی}

\زیرزیرقسمت{پاسخ‌گویی به پرسمان‌ها}

