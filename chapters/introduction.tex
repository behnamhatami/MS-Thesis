

\فصل{مقدمه}

مسئله‌ی \مهم{خوشه‌بندی} یکی از مهم‌ترین مسائل داده‌کاوی\پاورقی{data mining} به حساب می‌آید. در این مسئله هدف، دسته‌بندی تعدادی جسم به گونه‌ای است که اجسام در یک دسته (خوشه)، نسبت به یکدیگر در برابر  دسته‌های دیگر شبیه‌تر باشند (معیار‌های متفاوتی برای تشابه تعریف می‌گردد). این مسئله در حوزه‌های مختلفی از علوم کامپیوتر، از جمله داده‌کاوی، جست‌وجوی الگو\پاورقی{pattern recognition}، پردازش تصویر\پاورقی{image analysis}، بازیابی اطلاعات\پاورقی{information retrieval} و بایوانفورماتیک\پاورقی{bioinformatics} مورد استفاده قرار می‌گیرد \مرجع{han2006}.

مسئله‌ی خوشه‌بندی، به‌خودی‌خود یک مسئله‌ی الگوریتمی به حساب نمی‌آید. راه‌حل‌های الگوریتمی بسیار زیادی برای خوشه‌بندی تعریف شده است. به طور کلی این الگوریتم‌ها به چهار دسته‌ی زیر تقسیم‌بندی می‌گردند:
\شروع{فقرات}
\فقره{خوشه‌بندی‌های سلسه‌مراتبی\پاورقی{hierarchical clustering}}
\فقره{خوشه‌بندی‌های مرکزگرا\پاورقی{centroid-based clustering}}
\فقره{خوشه‌بندی‌های مبتنی بر توزیع\پاورقی{distribution-based} نقاط}
\فقره{خوشه‌بندی‌های مبتنی بر چگالی\پاورقی{density-based} نقاط}
\پایان{فقرات} 
در عمل هیچ کدام از راه‌حل‌های بالا بر دیگری ارجهیت ندارند و باید راه‌حل مد نظر را متناسب با کاربرد مطرح مورد استفاده قرار داد. به طور مثال استفاده از الگوریتم‌های مرکزگرا، برای خوشه‌بندی‌های غیر محدب به خوبی عمل نمی‌کند.
یکی از راه‌حل‌های شناخته شده برای مسئله‌ی خوشه‌بندی، الگوریتم \مهم{$k$-مرکز} است. در این الگوریتم هدف، پیدا کردن $k$ نقطه به عنوان مرکز دسته‌ها است به طوری‌که شعاع دسته‌ها تا حد ممکن کمینه شود. در نظریه‌ی گراف، مسئله‌ی $k$-مرکز متریک\پاورقی{metric} یا مسئله‌ی \مهم{استقرار تجهیزات متریک}\پاورقی{metric facility location} یک مسئله‌ی \مهم{بهینه‌سازی ترکیبیاتی}\پاورقی{combinatorial optimization} است. فرض کنید که $n$ شهر و فاصله‌ی دوبه‌دوی آن‌ها، داده‌شده است. می‌خواهیم $k$ انبار در شهرهای مختلف بسازیم به‌طوری‌که بیش‌ترین فاصله‌ی هر شهری از نزدیک‌ترین انبار به خود، کمینه گردد. در حالت نظریه‌ی گراف آن، این بدان معناست که مجموعه‌ای شامل $k$ رأس انتخاب کنیم به‌طوری‌که بیش‌ترین فاصله‌ی هر نقطه از نزدیک‌ترین نقطه‌اش داخل مجموعه‌ی $k$ عضوی کمینه گردد.  توجه نمایید که فاصله‌ی بین رئوس باید در فضای متریک\پاورقی{metric space} باشند و یا به زبان دیگر، یک گراف کامل داشته باشیم که فاصله‌ها در آن در رابطه‌ی مثلثی\پاورقی{triangle equation} صدق می‌کنند. مثالی از مسئله‌ی $2$-مرکز را در شکل \رجوع{شکل:دومرکز} نشان داده شده است.

در این پژوهش، مسئله‌ی $k$-مرکز با متریک‌های خاص و برای $k$های کوچک مورد بررسی قرار گرفته است و از‌ جنبه‌های متفاوتی بهبود یافته است.

\شروع{شکل}[ht]
\centerimg{k-center}{10cm}
\شرح{نمونه‌ای از‌مسئله‌ی ۲-مرکز}
\برچسب{شکل:دومرکز}
\پایان{شکل}

\قسمت{تعریف مسئله}

تعریف دقیق‌تر مسئله‌ی $k$-مرکز در زیر آمده است:

\شروع{مسئله}
یک گراف کامل بدون جهت $G = (V, E)$ با فاصله‌های $d(v_i, v_j)$ که از نامساوی مثلثی پیروی می‌کنند داده‌شده است. زیرمجموعه $S \subseteq V$ با اندازه‌ی $k$ را انتخاب کنید به‌طوری‌که عبارت زیر را کمینه کند:
$$\max_{v \in V} \{ \min_{s \in S} d(v, s) \}$$
\پایان{مسئله}

گونه‌های مختلفی از مسئله‌ی $k$-مرکز با محدودیت‌های متفاوتی به‌وسیله‌ی پژوهشگران مورد مطالعه قرارگرفته است. از جمله می‌توان حالتی که در بین داده‌های ورودی داده‌های پرت وجود دارد و قبل از خوشه‌بندی می‌توانیم تعدادی نقاط ورودی را حذف نماییم و سپس به خوشه‌بندی بپردازیم. سختی این مسئله از آنجاست که نه تنها باید مسئله‌ی خوشه‌بندی را حل نمود، بلکه در ابتدا باید تصمیم گرفت که کدام یک از داده‌ها را به‌عنوان پرت در نظر گرفت که بهترین جواب در زمان خوشه‌بندی به دست آید. در واقع اگر تعداد نقاط پرتی که مجاز‌ به حذف است، برابر صفر باشد، مسئله به مسئله‌ی $k$-مرکز تبدیل می‌شود. نمونه‌ای از مسئله‌ی $2$-مرکز با $7$ داده‌ی پرت را در شکل \رجوع{شکل:دومرکزپرت} می‌توانید ببینید. تعریف دقیق‌تر این مسئله در زیر آمده است:

\شروع{شکل}[ht]
\centerimg{with-outlier-k-center}{10cm}
\شرح{نمونه‌ای از‌مسئله‌ی ۲-مرکز با داده‌های پرت}
\برچسب{شکل:دومرکزپرت}
\پایان{شکل}


\شروع{مسئله}
یک گراف کامل بدون جهت $G = (V, E)$ با فاصله‌های $d(v_i, v_j)$ که از نامساوی مثلثی پیروی می‌کنند داده‌شده است. زیرمجموعه $S \subseteq V$ با اندازه‌ی $k$ و $Z \subseteq V$ به‌اندازه‌ی $z$ را انتخاب کنید به‌طوری‌که عبارت زیر را کمینه کند:
$$\max_{v \in (V - Z)} \{ \min_{s \in S} d(v, s) \}$$
\پایان{مسئله}

گونه‌ای دیگری که در مسئله‌ی $k$-مرکز که در سال‌های اخیر مورد توجه قرار گرفته است، حالت جویبار داده‌ی آن است. در این گونه، در ابتدا تمام نقاط در دسترس نیست، بلکه به مرور زمان نقاط در دسترس قرار می‌گیرند. محدودیت دومی که وجود دارد، محدودیت شدید حافظه است، به طوری که نمی‌توان تمام نقاط را در حافظه نگه داشت و به طور معمول باید مرتبه‌ی حافظه‌ای کم‌تر از مرتبه حافظه‌ی خطی\پاورقی{sublinear} متناسب با تعداد نقاط استفاده نمود. مدلی که ما در این پژوهش بر روی آن تمرکز داریم مدل جویبار داده تک‌گذره\مرجع{aggarwal2007data} است. یعنی تنها یک بار می‌توان از ابتدا تا انتهای داده‌ها را بررسی کرد.

یکی از دغدغه‌هایی که در مسائل جویبار داده وجود دارد، عدم داشتن تمام نقاط است. بنابراین در بعضی موارد ممکن است برای مسئله‌ی $k$-مرکز‌، مرکزی برای یک دسته انتخاب شود که در بین نقاط ورودی نیست. نمونه‌ای از مسئله‌ی $2$-مرکز در حالت پیوسته را در شکل‌ \رجوع{شکل:دومرکزپیوسته} نشان داده شده است. این مسئله تنها برای $L_p$-متریک مطرح می‌شود، زیرا مرکز دسته‌ها ممکن است در هر نقطه از فضا قرار بگیرد و ما نیاز داریم که فاصله‌ی آن را از تمام نقاط بدانیم. تعریف دقیق‌ مسئله در زیر آمده است:

\شروع{شکل}[ht]
\centerimg{continious-k-center}{10cm}
\شرح{نمونه‌ای از‌مسئله‌ی ۲-مرکز در حالت پیوسته}
\برچسب{شکل:دومرکزپیوسته}
\پایان{شکل}

\شروع{مسئله}
مجموعه‌ی $U$ از نقاط فضای $d$ بعدی داده شده است. زیرمجموعه $S \subseteq U$ با اندازه‌ی $k$ را انتخاب کنید به‌طوری‌که عبارت زیر را کمینه کند:
$$\max_{u \in U} \{ \min_{s \in S} L_p(u, s) \}$$
\پایان{مسئله}

از آنجایی که گونه‌ی جویبار داده و داده پرت مسئله‌ی $k$-مرکز به تازگی مورد توجه قرار گرفته است و نتایج به‌دست‌آمده قابل بهبود است، در این تحقیق سعی شده است که تمرکز بر روی این‌گونه‌ی خاص از مسئله باشد. همچنین در این پژوهش سعی می‌شود گونه‌های مسئله را برای انواع متریک‌ها و برای $k$های کوچک نیز مورد بررسی قرار گیرد. 

\قسمت{اهمیت موضوع}

مسئله‌ی $k$-مرکز و گونه‌های آن کاربردهای زیادی در داده‌کاوی دارند. این الگوریتم یکی از رایج‌ترین الگوریتم‌های مورد استفاده برای خوشه‌بندی محسوب می‌شود. به علت افزایش حجم داده‌ها و تولید داده‌ها در طول زمان، مدل جویبار داده‌ی مسئله در سال‌های اخیر مورد توجه قرار گرفته است. مسئله‌ی $k$-مرکز در بعضی مسائل مانند ارسال نامه از‌ تعدادی مراکز پستی به گیرنده‌ها، نیاز دارد که تمام نامه‌ها را به دست گیرنده‌ها برساند و در نتیجه باید نقاط را پوشش دهد، ولی برای بعضی از مسائل تجاری، نیازی به پوشش تمام نقاط هدف نیست و از لحاظ اقتصادی به‌صرفه است که نقاط پرت را در نظر نگرفت. به طور مثال \لر{Kmarts} اعلام کرده است که به ۸۸ درصد جمعیت آمریکا با فاصله‌ای حداکثر ۶ مایل، می‌تواند سرویس دهد. درصورتی‌که اگر این شرکت قصد داشت تمام جمعیت آمریکا را پوشش دهد، نیاز داشت تعداد شعب یا شعاع پوشش خود را به میزان زیادی افزایش دهد که از لحاظ اقتصادی به‌صرفه نیست\مرجع{charikar2001algorithms}. علاوه بر دو گونه‌ی مطرح‌شده در این قسمت، گونه‌های دیگر از مسئله‌ی $k$-مرکز در مرجع \مرجع{charikar2001algorithms} آمده است.

\قسمت{ادبیات موضوع}

همان‌طور که ذکر شد مسئله‌ی $k$-مرکز در حالت داده‌های پرت و جویبار داده، گونه‌های مختلف مسئله‌ی $k$-مرکز هستند و در حالت‌های خاصی به مسئله‌ی $k$-مرکز کاهش پیدا می‌کنند. مسئله‌ی $k$-مرکز در حوزه‌ی مسائل ان‌پی-سخت\پاورقی{NP-hard} قرار می‌گیرد و با فرض $P \neq NP$ الگوریتم دقیق با زمان چندجمله‌ای برای آن وجود ندارد. بنابراین برای حل کارای\پاورقی{efficient} این مسائل از الگوریتم‌های تقریبی\پاورقی{Approximation Algorithm}  استفاده می‌شود.

برای مسئله‌ی $k$-مرکز، دو الگوریتم تقریبی معروف وجود دارد.
در الگوریتم اول، که به روش حریصانه\پاورقی{greedy} عمل می‌کند، در هر مرحله بهترین مرکز ممکن را انتخاب می‌کند به طوری تا حد ممکن از مراکز‌ قبلی دور باشد. این الگوریتم، الگوریتم تقریبی با ضریب تقریب 2 ارائه می‌دهد.
در الگوریتم دوم، با استفاده از مسئله‌ی مجموعه‌ی غالب کمینه\پاورقی{dominating set}، الگوریتمی با ضریب تقریب ۲ ارائه می‌گردد.
همچنین ثابت شده است، که بهتر از این ضریب تقریب، الگوریتمی نمی‌توان ارائه شود مگر آن‌که $P = NP$ باشد.


\قسمت{اهداف تحقیق}

در این پایان‌نامه مسئله‌ی $k$-مرکز  در حالت جویبار داده با داده‌های پرت در حالت‌های مختلف مورد بررسی قرار می‌گیرد و سعی خواهد شد که نتایج قبلی در این مسائل را از جنبه‌های مختلفی مورد بهبود قرار دهد.

اولین مسئله‌ای که مورد بررسی قرار گرفته است، ارائه الگوریتمی تقریبی، برای مسئله‌ی $1$-مرکز در حالت جویبار داده است به طوری که، نه تنها تمام نقاط ورودی را می‌پوشاند بلکه تضمین می‌کند که دایره‌ی بهینه‌‌ی جواب $1$-مرکز برای نقاط ورودی را نیز‌ می‌پوشاند. در تلاش‌ اول، الگوریتم جدیدی با حافظه‌ و زمان به‌روزرسانی $\cO(\frac{d}{\epsilon})$ و  با ضریب تقریب $1.8 + \epsilon$، برای این مسئله ارائه گردید. با بررسی‌های بیش‌تر، الگوریتم دیگری با ضریب تقریب $1.69$، با الگوریتمی کاملا متفاوت، با حافظه‌ی $\cO(d)$ برای این مسئله ارائه گردید.

مسئله‌ی دومی که مورد بررسی قرار گرفت، مسئله‌ی $1$-مرکز‌ در حالت جویبار داده با داده‌های پرت است. در ابتدا الگوریتمی ساده با حافظه‌ی $\cO(zd)$ و زمان به‌روزرسانی $\cO(zd\log(z))$ با ضریب تقریب ۲ برای این مسئله ارائه گردید. در بررسی‌های بعدی، با استفاده از‌ نتایج بدست آمده در قسمت قبل، برای $z$ های کوچک، الگوریتمی با حافظه‌ی $\cO(dz^d)$ با ضریب تقریب $1.69$ برای این مسئله ارائه شد که ضریب تقریب بهترین الگوریتم موجود را، از $1.79$ به $1.69$ کاهش می‌دهد.

مسئله‌ی سومی که مورد بررسی قرار گرفت، مسئله‌ی $2$-مرکز در حالت جویبار داده با داده‌های پرت است. بهترین الگوریتمی که در حال حاضر برای این مسئله وجود دارد، یک الگوریتم با ضریب تقریب $4 + \epsilon$ است. ما با ارائه‌ الگوریتمی جدید، الگوریتمی با ضریب تقریب $1.8 + \epsilon$ برای این مسئله ارائه شد که بهبودی قابل توجه برای این مسئله است.
 

\قسمت{ساختار پایان‌نامه}

این پایان‌نامه در پنج‌ فصل به شرح زیر ارائه خواهد شد. در فصل دوم به بیان مفاهیم و تعاریف مرتبط با موضوعات مورد بررسی در بخش‌های دیگر خواهیم پرداخت. فصل سوم این پایان‌نامه شامل مطالعه و بررسی کارهای پیشین انجام شده مرتبط با موضوع این پایان‌نامه خواهد بود. این فصل در سه بخش تنظیم گردیده است. در بخش اول، مسئله‌ی $k$-مرکز مورد بررسی قرار می‌گیرد. در بخش دوم، حالت جویبار داده‌ی مسئله و مجموعه هسته‌های\پاورقی{coreset} مطرح برای این مسئله مورد بررسی قرار می‌گیرد. در نهایت، در بخش سوم، مسئله‌ی $k$-مرکز‌ با داده‌های پرت مورد بررسی قرار می‌گیرد.

در فصل چهارم، نتایج جدیدی که در این پایان‌نامه به دست آمده است، ارائه می‌شود. این نتایج شامل الگوریتم‌های جدید برای مسئله‌ی $1$-مرکز در حالت جویبار داده،  مسئله‌ی $1$-مرکز‌ با داده‌های پرت در حالت جویبار داده و مسئله‌ی دو مرکز در حالت جویبار داده با داده‌های پرت می‌شود.

در  فصل پنجم، به جمع‌بندی کارهای انجام شده در این پژوهش و ارائه‌ی پیشنهاد‌هایی برای انجام کارهای آتی و تعمیم‌هایی که از راه‌حل ارائه شده وجود دارد، خواهیم پرداخت.