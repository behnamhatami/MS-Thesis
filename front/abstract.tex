
% -------------------------------------------------------
%  Abstract
% -------------------------------------------------------


\pagestyle{empty}

\شروع{وسط‌چین}
\مهم{چکیده}
\پایان{وسط‌چین}
\بدون‌تورفتگی

مسئله‌ی $k$-مرکز به عنوان مسئله‌ای شناخته شده در علوم کامپیوتر مطرح است که در حوزه‌های مختلفی مورد استفاده قرار می‌گیرد. هدف مسئله‌ی $k$-مرکز پیدا کردن کم‌ترین شعاعی است که می‌توان مجموعه‌ای از نقاط را با $k$ توپ با آن شعاع پوشاند. این مسئله در حالت کلی ان‌پی-سخت است. تمرکز اصلی این پایان‌نامه برروی مسئله‌ی $k$-مرکز در حالت جویبار داده با داده‌های پرت در ابعاد بالا است. به علت افزایش روز افزون حجم داده‌ها، گونه‌ی جویبار داده‌ی مسائل برای پاسخ‌گویی به حجم وسیع داده‌ها مورد توجه قرار می‌گیرد. از طرفی دیگر، وجود خطا در بین داده‌های تجربی و وجود داده‌هایی ناهمسان با اکثریت داده‌ها، در نظر گرفتن داده‌های پرت را مهم جلوه می‌دهد.

در این پایان‌نامه، دو مسئله‌ی $1$-مرکز و $2$-مرکز در فضای اقلیدسی با $z$ داده‌ی پرت در مدل جویبار داده مورد بررسی قرار می‌گیرد. برای مسئله‌ی $1$-مرکز با داده‌ی پرت، در حالتی که تعداد داده‌های پرت ثابت است، الگوریتمی با ضریب تقریب $1.7$ ارائه می‌دهیم. برای مسئله‌ی $2$-مرکز با داده‌ی پرت، الگوریتمی با ضریب تقریب $1.8 + \epsilon$ ارائه می‌دهیم که نسبت به تنها الگوریتم قبلی برای $k$ کلی با ضریب تقریب $4 + \epsilon$، بهبود قابل توجهی محسوب می‌شود.

\پرش‌بلند
\بدون‌تورفتگی \مهم{کلیدواژه‌ها}: 
خوشه‌بندی، $k$-مرکز، جویبار داده، الگوریتم تقریبی
\صفحه‌جدید
