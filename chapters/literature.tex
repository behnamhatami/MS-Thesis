
\فصل{کارهای پیشین}

در این فصل کارهای پیشین انجام‌شده روی مسئله‌ی $k$-مرکز، در سه بخش مورد بررسی قرار می‌گیرد.
در بخش اول، مسئله‌ی $k$-مرکز مورد بررسی قرار می‌گیرد.
در بخش دوم، حالت جویبار داده‌ی مسئله و مجموعه هسته‌های مطرح برای این مسئله مورد بررسی قرار می‌گیرد و در نهایت، در بخش سوم، مسئله‌ی $k$-مرکز با داده‌های پرت مورد بررسی قرار می‌گیرد.

\قسمت{$k$-مرکز‌ در حالت ایستا}

مسئله‌ی $k$-مرکز به عنوان مسئله‌ای شناخته شده در علوم کامپیوتر مطرح است.
این مسئله، در واقع یک مسئله‌ی بهینه‌سازی است که سعی در کاهش بیش‌ترین فاصله نقاط از مرکز دسته‌ها را دارد.
سختی اصلی این مسئله در انتخاب مرکز دسته‌هاست.
زیرا اگر بتوانیم مرکز دسته‌ها را به درستی تشخیص دهیم، کافی است هر نقطه را به دسته‌ای که نزدیک‌ترین مرکز را دارد، تخصیص دهیم.
به وضوح چنین تخصیصی بهینه‌ترین تخصیص ممکن است.
نمونه‌ای از این تخصیص را در شکل \رجوع{شکل:تخصیص‌نقاط} نشان داده شده است.

\شروع{شکل}[ht]
\centerimg{point-assignment}{10cm}
\شرح{نمونه‌ای از تخصیص نقاط‌ به ازای مراکز آبی رنگ.}
\برچسب{شکل:تخصیص‌نقاط}
\پایان{شکل}

در سال $1979$، نه تنها اثبات گردید که این مسئله در حالت کلی یک مسئله‌ی ان‌پی-سخت است \مرجع{michael1979computers}، بلکه ثابت شده است که این مسئله در صفحه‌ی دو بعدی و با متریک اقلیدسی نیز ان‌پی-سخت است \مرجع{megiddo1984complexity}.
فراتر از این، ثابت شده است که برای مسئله‌ی $k$-مرکز با متریک دلخواه هیچ الگوریتم تقریبی با ضریب تقریب بهتر از $2$ وجود ندارد.

ایده‌ی اصلی این کران پایین، کاهش مسئله‌ی پوشش رأسی، به مسئله‌ی $k$-مرکز است.
برای چنین کاهشی کافی است، از روی گراف اصلی که می‌خواهیم کوچک‌ترین مجموعه‌ی پوشش رأسی را در آن پیدا کنیم، یک گراف کامل بسازیم به طوری که به ازای هر یال در گراف اصلی، یک یال با وزن یک و به ازای هر یال که در گراف اصلی وجود ندارد، یک یال با وزن $2$ قرار دهیم.
نمونه‌ای از چنین تبدیلی را در شکل \رجوع{شکل:کاهش‌ ۲-مرکز} می‌توانید مشاهده کنید.
حال اگر الگوریتمی بتواند مسئله‌ی $k$-مرکز را با ضریب تقریب بهتر از $2$ حل نماید، آن گاه گراف جدید دارای یک $k$-مرکز با شعاع کم‌تر از $2$ است، اگر و تنها اگر گراف اصلی دارای یک پوشش رأسی با اندازه‌ی $k$ باشد.
برای متریک $L_2$ یا فضای اقلیدسی\پاورقی{Euclidean space} نیز ثابت شده است برای مسئله‌ی $k$-مرکز، الگوریتم تقریبی با ضریب تقریب بهتر از $1.822$ وجود ندارد \مرجع{bern1996approximation}.

\شروع{شکل}[ht]
\centerimg{k-center-reduction}{10cm}
\شرح{نمونه‌ای از تبدیل یک گراف ورودی مسئله‌ی پوشش‌ رأسی به یک ورودی مسئله‌ی $k$-مرکز(در گراف سمت چپ، یال‌های سیاه وزن $1$ و یال‌های آبی، وزن $2$ دارند)}
\برچسب{شکل:کاهش‌ ۲-مرکز}
\پایان{شکل}

\شروع{الگوریتم}{الگوریتم گنزالز}
\ورودی{$V$ مجموعه نقاط‌ و $k$ تعداد مرکز دسته‌ها}
\دستور{$S$ را برابر مجموعه تهی قرار بده.}
\دستور{عنصر دلخواه از مجموعه نقاط $V$ را به $S$ اضافه کن.}
\به‌ازای{$i$ بین $2$ تا $k$}
\دستور{$v$ را نقطه‌ای از $V$ در نظر بگیرید که بیش‌ترین فاصله را از مجموعه‌ی $S$ دارد.}
\دستور{$v$ را به $S$ اضافه کن.}
\پایان‌به‌ازای
\دستور{$S$ را برگردان}
\پایان{الگوریتم}

یکی از اولین الگوریتم‌های تقریبی برای مسئله‌ی $k$-مرکز به وسیله‌ی گنزالز\پاورقی{Gonzalez} ارائه شده است \مرجع{han2011data}. این الگوریتم یک الگوریتم تقریبی با ضریب ۲ است و در زمان $\cO(kn)$ قابل اجراست. الگوریتم گنزالز، از نظر روش برخورد با مسئله، یک الگوریتم حریصانه\پاورقی{Greedy} محسوب می‌شود. برای این‌که عملکرد الگوریتم گنزالز را درک کنید، نیاز به تعریف فاصله‌ی یک نقطه از یک مجموعه نقطه داریم.

\شروع{تعریف}
فاصله‌ی نقطه‌ی $v$ از مجموعه‌ای ناتهی از نقاط $S$ را برابر فاصله‌ی نقطه‌ای درون $S$ از $v$ تعریف می‌کنیم، به‌گونه‌ای که از تمام نقاط $S$ به $v$ نزدیک‌تر باشد. در واقع داریم:
$$d(v, S) = \min_{u \in S} \set{ d(u, v) }$$
\پایان{تعریف}

همان‌طور که در الگوریتم ~\رجوع{الگوریتم: الگوریتم گنزالز} مشاهده می‌کنید، روش اجرای این الگوریتم به این گونه است که در ابتدا یک نقطه‌ی دلخواه را به عنوان مرکز دسته‌ی اول در نظر می‌گیرد.
سپس دورترین نقطه از آن را به عنوان مرکز دسته‌ی دوم در نظر می‌گیرد.
در هر مرحله، دورترین نقطه از مرکز مجموعه دسته‌های انتخاب شده را به عنوان مرکز دسته‌ی جدید به مجموعه مراکز دسته‌ها اضافه می‌کند.
با اجرای الگوریتم تا $k$ مرحله، مراکز دسته‌ها انتخاب می‌شوند.
حال اگر هر نقطه را به نزدیک‌ترین مرکز انتخابی تخصیص دهیم، می‌توان نشان داد که شعاع بزرگ‌ترین دسته، حداکثر دو برابر شعاع بهینه برای مسئله‌ی $k$-مرکز است.
فدر\پاورقی{Feder} و سایرین، زمان اجرای الگوریتم گنزالز را برای هر $L_p$-متریک به مرتبه‌ی $\cO(n \log{k})$ بهبود بخشیدند.
نمونه‌ای از اجرای الگوریتم گنزالز، در شکل \رجوع{شکل:گنزالز} نشان داده شده است.

\شروع{شکل}[ht]
\centerimg{Gonzalez}{10cm}
\شرح{نمونه‌ای از حل مسئله‌ی ۳-مرکز با الگوریتم گنزالز}
\برچسب{شکل:گنزالز}
\پایان{شکل}

تا به اینجا، تنها بر روی حالت کلی مسئله‌ی $k$-مرکز صحبت شد و تنها محدودیتی که مورد توجه قرار گرفت، متریک مطرح برای فاصله‌ی نقاط بوده است.
در ادامه به بررسی، حالاتی از مسئله‌ی $k$ -مرکز، که $k$ تعداد دسته‌ها یا $d$ ابعاد فضا ثابت باشند می‌پردازیم.
آگاروال\پاورقی{Agarwal} و سایرین الگوریتمی دقیق با زمان اجرای $n^{\cO(k^{1 - \frac{1}{d}})}$ برای مسئله $k$-مرکز در فضای $L_p$-متریک با ابعاد ثابت $d$ ارائه داده‌اند \مرجع{agarwal2002exact}.
قابل توجه است که اگر $d$ ثابت نباشد، مسئله‌ی $k$-مرکز حتی برای متریک اقلیدسی ($L_2$-متریک) با تعداد دسته‌ی ثابت $k \geq 2$، ان‌پی-سخت است \مرجع{megiddo1990complexity}.

علاوه بر حالاتی که ابعاد فضا یا تعداد دسته‌ها ثابت‌اند، مسئله‌ی $k$-مرکز برای حالتی که مقادیر $k$ و $d$ کوچک هستند، مورد بررسی قرار گرفته‌اند و الگوریتم‌های بهینه‌تری از الگوریتم‌های کلی برای این حالت‌های خاص ارائه شده است.
به طور مثال، برای مسئله‌ی $1$-مرکز در فضای اقلیدسی با ابعاد ثابت، الگوریتم خطی با زمان اجرای $\cO((d+1)! n)$ وجود دارد \مرجع{chazelle1996linear}.
الگوریتم ارائه شده بر پایه‌ی دو نکته‌ی اساسی بنا شده است.
اول اینکه کره‌ی بهینه را می‌توان با حداکثر $d+1$ نقطه‌ی واقع در پوسته‌ی کره‌ی بهینه مشخص نمود و دوم اینکه اگر نقاط ورودی را با ترتیبی تصادفی پیمایش کنیم احتمال اینکه نقطه‌ی پیمایش شده جزء نقاط مرزی باشد از مرتبه‌ی $\cO(\frac{d}{n})$ است که با توجه به ثابت بودن $d$ این احتمال برای $n$های بزرگ کوچک محسوب می‌شود و زمان اجرای الگوریتم، به طور میانگین خطی خواهد بود. 


برای متریک اقلیدسی در دو بعد برای مسئله‌ی $2$-مرکز, بهترین الگوریتم را چن\پاورقی{Chan} با زمان اجرای $\cO(c \log^2{n} \log^2{\log{n}})$ و حافظه‌ی $\cO(n)$ ارائه داده است \مرجع{chan1999more}.
برای فضای سه‌بعدی اقلیدسی نیز آگاروال و سایرین، الگوریتمی با متوسط زمان اجرای $\cO(n^3\log^8{n})$ ارائه داده است \مرجع{agarwal20132}.

\قسمت{$k$-مرکز‌ در حالت جویبار داده}

در مدل جویبار داده، مشکل اصلی عدم امکان نگه‌داری تمام داده‌ها در حافظه است و باید سعی شود تنها داده‌هایی که ممکن است در ادامه مورد نیاز باشند را، نگه‌داریم.
یکی از راه‌های رایج برای این کار نگه‌داری مجموعه‌ای از نقاط (نه لزوماً زیرمجموعه‌ای از نقاط ورودی) به عنوان نماینده‌ی نقاط به‌طوری‌که جواب مسئله‌ی $k$-مرکز برای آن‌ها منطبق با جواب مسئله‌ی $k$-مرکز برای نقاط اصلی باشد (با تقریب قابل قبولی).
به چنین مجموعه‌ای مجموعه‌ی هسته‌ی نقاط گفته می‌شود. 

بهترین مجموعه هسته‌ای که برای مسئله‌ی $k$-مرکز ارائه شده است، روش ارائه‌شده به وسیله‌ی ضرابی‌زاده برای نگه‌داری یک $\epsilon$-هسته با حافظه‌ی $\cO(\frac{k}{\epsilon^d})$ برای $L_p$-متریک‌ها است \مرجع{zarrabi2008core}.
در روش ارائه شده، از چند ایده‌ی ترکیبی استفاده شده است. 

در ابتدا، الگوریتم با استفاده از یک الگوریتم تقریبی با ضریب ثابت، یک تقریب از جواب بهینه به دست می‌آورد.
به طور مثال با استفاده از الگوریتم گنزالز، یک $2$-تقریب از شعاع بهینه به علاوه‌ی مرکز دسته‌های پیدا شده را به دست می‌آورد.
حال با استفاده از طول شعاع الگوریتم تقریبی، حول هر مرکز به دست آمده، یک توری با $\cO(\frac{1}{\epsilon})$ شبکه‌بندی در هر بعد تشکیل می‌دهد و چون هر نقطه در حداقل یکی از توری‌ها قرار می‌گیرد، می‌توان با حداکثر $\epsilon$ تقریب در جواب نهایی نقاط را به نقاط شبکه‌بندی توری گرد نمود.
با این کار، دیگر نیازی به نگهداری تمام نقاط ورودی نبوده و تنها نقاط شبکه‌بندی توری نگهداری می‌شود.
با این روش می‌توان به یک $\epsilon$-هسته برای مسئله‌ی $k$-مرکز رسید. 

نکته‌ی اساسی برای سازگار سازی روش ارائه‌شده با مدل جویبار داده‌ی تک‌گذره استفاده از روش دوبرارسازی\پاورقی{Doubling} رایج در الگوریتم‌های جویبار داده است.
نمونه‌ای از اجرای الگوریتم ضرابی‌زاده را در شکل \رجوع{شکل:توری} نشان داده شده است.
برای دیدن اثبات‌ها و توضیح بیش‌تر در مورد روش ارائه شده می‌توانید به مرجع \مرجع{zarrabi2008core} مراجعه کنید.

\شروع{شکل}[ht]
\centerimg{zarrabi-zade-mesh-coreset}{10cm}
\شرح{نمونه‌ای از شبکه‌بندی الگوریتم ضرابی‌زاده (نقاط آبی، مراکز به دست آمده از الگوریتم تقریبی است). پس از شبکه‌بندی کافی است برای هر کدام از خانه‌های شبکه‌بندی، تنها یکی را به عنوان نماینده در نظر بگیریم.}
\برچسب{شکل:توری}
\پایان{شکل}

از جمله مشکلات وارده به الگوریتم ضرابی‌زاده، وابستگی اندازه‌ی مجموعه‌ی هسته به ابعاد فضا است.
بنابراین هسته‌ی ارائه شده به وسیله‌ی ضرابی‌زاده را نمی‌توان برای ابعاد بالا مورد استفاده قرار داد.
از طرفی، حساب کردن جواب از روی هسته در زمان چندجمله‌ای بر اساس $k$ و $d$ قابل انجام نیست.
با توجه با موارد گفته شده، برای قابل استفاده شدن الگوریتم‌ها برای ابعاد بالا، الگوریتم‌هایی ارائه می‌شود که ضریب تقریب بدتری دارند، اما میزان حافظه‌ی مصرفی یا اندازه‌ی مجموعه هسته‌ی آن‌ها چندجمله‌ای بر اساس $d$ و $k$ و $\log{n}$ باشد.
به چنین الگوریتم‌هایی الگوریتم‌های جویبار داده برای ابعاد بالا گفته می‌شود. 

اولین الگوریتم ارائه شده برای ابعاد بالا،  الگوریتمی با ضریب تقریب $8$ و حافظه‌ی مصرفی $\cO(dk)$ است \مرجع{charikar1997incremental}.
پس از آن، گوها\پاورقی{Guha}، به طور موازی با مک‌کاتن\پاورقی{McCutchen} و سایرین، الگوریتمی با ضریب تقریب $(2+\epsilon)$ با حافظه‌ی مصرفی $\cO(\frac{dk}{\epsilon} \log{\frac{1}{\epsilon}})$ برای مسئله‌ی $k$-مرکز در هر فضای متریکی ارائه دادند \مرجع{mccutchen2008streaming, guha2009tight}.
در سال ۲۰۱۴، آهن\پاورقی{Ahn} و سایرین, الگوریتمی با همین ضریب تقریب و حافظه‌ی $\cO((k+3)!2^k\frac{d}{\epsilon})$ ارائه داده‌اند که برای $k$‌های ثابت، حافظه را از مرتبه‌ی $\cO(\log{\frac{1}{\epsilon}})$ کاهش می‌دهد \مرجع{ahn2014computing}. 

تا به اینجا ما به بررسی مسئله‌ی $k$-مرکز در حالت جویبار داده بدون محدودیت خاصی پرداختیم.
برای حالت‌های خاص $k$، به خصوص $1$ و $2$، مسئله‌ی $k$-مرکز با متریک اقلیدسی مورد بررسی زیادی قرار گرفته است و راه‌حل‌های بهینه‌تری نسبت به حالت کلی برای آن‌ها پیشنهاد شده است.
به طور مثال، می‌توان یک هسته با اندازه‌ی $\cO(\frac{1}{\epsilon^{\frac{d-1}{2}}})$, با استفاده از نقاط حدی\پاورقی{Extreme points} در تعداد مناسبی جهت، به صورت جویبار داده ساخت. 

\شروع{الگوریتم}{الگوریتم ضرابی‌زاده}
\دستور{$B$ را توپی به مرکز نقطه‌ی اول و شعاع صفر قرار دهید.}
\به‌ازای{هر نقطه‌ی $u$ در جویبار داده}
\اگر{$u$ داخل $B$ قرار می‌گیرد}
\دستور{ادامه بده.}
\وگرنه
\دستور{$B$ را با کوچک‌ترین توپ ممکن که هردوی $B$ و $u$ را می‌پوشاند، جایگزین کن. }
\پایان‌اگر
\پایان‌به‌ازای
\دستور{مجموعه‌ی $S$ را برگردان.}
\پایان{الگوریتم}

همان‌طور که برای الگوریتم ضرابی‌زاده مطرح شد، مشکل عمده‌ی مجموعه هسته‌ی ارائه‌شده، وابستگی حافظه‌ی مصرفی آن به $d$ است.
در راستای حل این مشکل، ضرابی‌زاده و سایرین \مرجع{zarrabi2006simple} برای ابعاد بالا و متریک اقلیدسی، الگوریتمی با ضریب تقریب $1.5$ و حافظه‌ی مصرفی $\cO(d)$ ارائه دادند.
در واقع در الگوریتم آن‌ها، در هر لحظه تنها یک مرکز و یک شعاع نگه داشته می‌شود، که کم‌ترین حافظه‌ی ممکن برای مسئله‌ی $1$-مرکز است.
همان‌طور که در الگوریتم ~\رجوع{الگوریتم: الگوریتم ضرابی‌زاده} مشاهده می‌کنید، نقطه‌ی اول را به عنوان مرکز کره با شعاع صفر در نظر گرفته می‌شود.
با فرا رسیدن هر نقطه‌ی جدید، اگر نقطه‌ی مد نظر، داخل کره‌ی فعلی بیفتد که بدون هیچ تغییری ادامه داده و در صورتی که بیرون کره‌ی فعلی بیفتد، آن را با کوچک‌ترین کره‌ای که نقطه‌ی جدید به علاوه‌ی کره‌ی قبلی را به طور کامل می‌پوشاند، جایگزین می‌کند. 

به وضوح در هر لحظه کره‌ی ساخته شده تمام نقاطی که تا کنون در جویبار داده آمده است را می‌پوشاند.
از طرفی ثابت می‌شود شعاع کره در هر لحظه، حداکثر $1.5$-برابر شعاع کره‌ی بهینه است.
نکته‌ی قابل توجه در مورد این الگوریتم، نگه‌داری کم‌ترین حافظه‌ی ممکن برای مسئله‌ی $1$-مرکز است.
زیرا تنها یک مرکز و یک شعاع نگه داشته می‌شود.
نمونه‌ای از اجرای الگوریتم را بر روی چهار نقطه می‌توان در شکل \رجوع{شکل:یک‌مرکز‌ضرابی‌زاده} دید.
برای اثبات کامل‌تر می‌توانید به مرجع \مرجع{zarrabi2006simple} مراجعه کنید. 

\شروع{شکل}[ht]
\centerimg{zarrabi-approximation-1-center}{10cm}
\شرح{نمونه‌ای از‌ اجرای الگوریتم ضرابی‌زاده بر روی چهار نقطه $P_1 \cdots P_4$ که به ترتیب اندیس در جویبار داده فرا می‌رسند و دایره‌های $B_0 \cdots B_2$ دایره‌هایی که الگوریتم به ترتیب نگه می‌دارد.}
\برچسب{شکل:یک‌مرکز‌ضرابی‌زاده}
\پایان{شکل}


در ادامه، آگاروال و سایرین \مرجع{agarwal2010streaming}، الگوریتمی تقریبی با حافظه‌ی مصرفی $\cO(d)$ ارائه دادند.
در الگوریتم ارائه شده، ضریب تقریب، برابر $\frac{1 + \sqrt{3}}{2}$  تخمین زده شد، اما با تحلیل دقیق‌تری که چن و سایرین \مرجع{chan2014streaming} بر روی همان الگوریتم انجام دادند، مشخص شد همان الگوریتم دارای ضریب تقریب $1.22$ است.

الگوریتم آگاروال از الگوریتم کلارکسون و سایرین \مرجع{badoiu2003smaller} به عنوان یک زیرالگوریتم استفاده می‌کند.
الگوریتم کلارکسون، الگوریتمی کاملا مشابه الگوریتم گنزالز است و به این گونه عمل می‌کند که در ابتدا یک نقطه دلخواه را به عنوان نقطه‌ی اول انتخاب می‌کند.
سپس دورترین نقطه از‌ نقطه‌ی اول را به عنوان دومین نقطه انتخاب می‌کند.
ازین به بعد در هر مرحله، نقطه‌ای که از نقاط انتخاب شده‌ی قبلی بیش‌ترین فاصله را دارد به عنوان نقطه‌ی جدید انتخاب می‌کند.
اگر این الگوریتم را تا $\cO(\frac{1}{\epsilon})$ مرحله ادامه بدهیم، به مجموعه‌ای با اندازه‌ی $\cO(\frac{1}{\epsilon})$ خواهیم رسید که کلارکسون و سایرین اثبات کرده‌اند که یک $\epsilon$-هسته برای مسئله‌ی $1$-مرکز‌ است.
در واقع آن‌ها نشان داده‌اند یک $\cO(\frac{1}{\epsilon})$-مرکز به دست آمده برای مجموعه‌ای از نقاط با استفاده از الگوریتم گنزالز، یک $\epsilon$-هسته برای مسئله‌ی $1$-مرکز برای همان مجموعه نقاط است.

الگوریتم آگاروال به این گونه عمل می‌کند که اولین نقطه‌ی جویبار داده را به عنوان تنها نقطه‌ی مجموعه‌ی $K_1$ در نظر می‌گیرد.
حال تا وقتی که نقاطی که فرا می‌رسند داخل $(1 + \epsilon)Meb(K_1)$ قرار بگیرند، ادامه می‌دهد.
اولین نقطه‌ای که در شرایط ذکر شده صدق نمی‌کند را $p_2$ بنامید.
حال الگوریتم کلارکسون را بر روی $K_1 \cup \set{p_2}$ اجرا کرده و مجموع هسته‌ی به دست آمده را $K_2$ بنامید.
با ادامه‌ی این روند، الگوریتم، دنباله‌ای از مجموعه هسته $\kappa = \{ K_1, \cdots, K_u \}$ نگه می‌دارد و زمانی که نقطه‌ی $p_{u+1}$ پیدا شود که در هیچ‌کدام از $(1 + \epsilon)Meb(K_j)$ به ازای $1 \leq j \leq u$ نباشد، الگوریتم کلارکسون را برای $\cup_{j = 1}^{u} K_j \cup \set{p_{u+1}}$ اجرا نموده و مجموعه هسته‌ی به دست آمده را $K_{u+1}$ می‌نامد.
با توجه به نحوه‌ی ساخته شدن $K_i$ها، به راحتی می‌توان نشان داد رابطه‌ی زیر برقرار است:
$$P \subset \cup_{i=1}^u (1+\epsilon)Meb(K_i)$$
حال در نهایت برای به دست آوردن جواب نهایی کافی است کوچک‌ترین کره‌ای که 
$$\cup_{i=1}^u (1+\epsilon)Meb(K_i)$$
 را می‌پوشاند را به عنوان جواب محاسبه کنیم.
 چن و سایرین ثابت کرده‌اند که کره‌ی نهایی دارای شعاع حداکثر $1.22$ برابر شعاع بهینه است. برای مشاهده‌ جزئیات بیش‌تر، به \مرجع{agarwal2010streaming, chan2014streaming} مراجع کنید.

آگاروال نه تنها الگوریتمی ارائه داد که در نهایت، ثابت شد حداکثر جوابی با ضریب تقریب $1.22$ برابر جواب بهینه می‌دهد، بلکه نشان داد، که با حافظه‌ی چندجمله‌ای بر اساس $\log{n}$ و $d$ نمی‌توان الگوریتمی ارائه داد که ضریب تقریب بهتر از $\frac{1 + \sqrt{2}}{2}$ داشته باشد.

\قضیه{هر الگوریتم تحت مدل جویبار داده که یک $\alpha$-تقریب برای مسئله‌‌ی $1$-مرکز برای مجموعه‌ی $S$ شامل $n$ نقطه در فضای $\IR^d$ نگه‌ می‌دارد، برای
$\alpha \leq \frac{1 + \sqrt{2}}{2} (1 - \frac{2}{d^{\frac{1}{3}}})$
 با احتمال حداقل $\frac{2}{3}$ نیاز به 
$\Omega(\min \{n, e^{d^{\frac{1}{3}}} \})$
حافظه مصرف می‌کند.}


\شروع{اثبات}

ایده‌ی اصلی اثبات بر اساس قضیه‌ی معروف آلیس و باب\پاورقی{alice and bob} در نظریه انتقال اطلاعات بنا شده است.
برای خواندن اثبات این قضیه می‌توانید به مرجع \مرجع{agarwal2010streaming} مراجعه کنید.

\پایان{اثبات}

علاوه بر مسئله‌ی $1$-مرکز، مسئله‌ی $2$-مرکز نیز در سال‌های اخیر مورد توجه قرار گرفته است و بهبودهایی نیز برای این مسئله ارائه شده است.
آهن و سایرین \مرجع{kim2014improved} در سال $2014$، اولین الگوریتم با ضریب تقریب کم‌تر از $2$ را برای مسئله‌ی $2$-مرکز در فضای اقلیدسی ارائه دادند.
این الگوریتم تقریباً پایه‌ی کار این پایان‌نامه برای حالت‌های مختلف است.
به همین منظور، تعدادی از لم‌های داخل این الگوریتم که در آینده استفاده می‌شود به اختصار توضیح داده می‌شوند.


\شروع{لم}
\برچسب{لم:ahn-segment}
فرض کنید $B$ یک کره‌ی واحد با مرکز $c$ در فضای اقلیدسی $\IR^d$ باشد.
هر پاره‌خط $pq$ به طول حداقل $1.2$ که به طور کامل داخل $B$ قرار دارد، کره‌ی $B'(c, 0.8)$ را قطع می‌کند.
\پایان{لم}

\شروع{شکل}[ht]
\centerimg{ahn-segment-lemma}{10cm}
\شرح{اثبات لم \رجوع{لم:ahn-segment}}
\برچسب{شکل:ahn-segment}
\پایان{شکل}

\شروع{اثبات}

صفحه‌ی گذرنده از پاره‌خط و مرکز کره را نظر بگیرید.
ادامه اثبات تنها به همین صفحه محدود می‌شود، بنابراین نیاز به در نظر گرفتن ابعاد بزرگ‌تر از $2$ نیست.
همان‌طور که در شکل \رجوع{شکل:ahn-segment} مشخص شده است، پای عمود از مرکز کره بر پاره‌خط $pq$ را $h$ بنامید.
بدون کم شدن از کلیت مسئله فرض کنید $\len{hp} \leq \len{hq}$. بنابراین داریم:
$$0.6 = \frac{1.2}{2} \leq \len{hq}$$
از‌ طرفی چون پاره‌خط $pq$ به طور کامل داخل کره‌ی واحد قرار گرفته است، بنابراین تمام نقاط آن، شامل دو سر آن، از مرکز کره، فاصله‌ی حداکثر $1$ دارند.
بنابراین طبق رابطه‌ی فیثاغورث، داریم:
$$\len{hc} = \sqrt{\len{qc}^2 - \len{qh}} \leq \sqrt{1 - 0.6^2} = 0.8$$
بنابراین نقطه‌ی $h$ داخل کره‌ی $B'$ قرار می‌گیرد.
از طرفی چون، $1 \leq \len{pq}$ است، بنابراین، $h$ داخل پاره‌خط قرار دارد و در نتیجه کره‌ی $B'$ با پاره‌خط $pq$ تقاطع دارد. 

\پایان{اثبات}

لم بالا در واقع نشان می‌دهد اگر در طول الگوریتم بتوانیم دو نقطه‌ی دور نسبت به هم (حداقل $1.2$ برابر شعاع بهینه) از یکی از دو کره‌ی بهینه را بیابیم، این پاره‌خط از مرکز کره‌ی بهینه فاصله‌ی کمی (حداکثر $0.8$ شعاع بهینه) دارد.

\شروع{لم}
\برچسب{لم:ahn-segment2}
فرض کنید $B$ کره‌ای به مرکز $c$ و شعاع واحد در $\IR^d$ باشد.
پاره‌خط دلخواه $pq$ با طول حداقل $1.2$ که به طور کامل داخل $B$ قرار دارد را در نظر بگیرید.
هر نقطه‌ی $x$ از پاره‌خط $pq$ که از دو سر آن حداقل $0.6$ فاصله داشته باشد، داخل کره‌ی $B'(c, 0.8)$ قرار می‌گیرد.

\پایان{لم}

\شروع{شکل}[ht]
\centerimg{ahn-segment2-lemma}{10cm}
\شرح{اثبات لم \رجوع{لم:ahn-segment2}}
\برچسب{شکل:ahn-segment2}
\پایان{شکل}

\شروع{اثبات}

اثبات این لم نیز کاملاً مشابه لم \رجوع{لم:ahn-segment} است.
بدون کم شدن از کلیت مسئله همان‌طور که در شکل \رجوع{شکل:ahn-segment2} مشخص شده است، فرض کنید زاویه‌ی $\angle{pxc}$ بزرگ‌تر مساوی $90$ درجه است.
در نتیجه داریم:
$$\sqrt{\len{px}^2 + \len{xc}^2} \leq \len{pc} \leq 1$$
از طرفی طبق فرض مسئله داریم:
$$0.6 \leq \len{px} \implies \len{xc} \leq \sqrt{1 - 0.6^2} = 0.8$$

\پایان{اثبات}

الگوریتم آهن، با استفاده از دو لم بالا و تقسیم مسئله به دو حالتی که دو کره‌ی بهینه بیش از $2$ برابر شعاع بهینه یا کم‌تر از $2$ برابر شعاع بهینه فاصله داشته باشند، دو الگوریتم کاملا جداگانه ارائه می‌دهد که به طور موازی اجرا می‌گردند.
اجرای موازی این دو الگوریتم، در هر لحظه دو جواب درست ارائه می‌دهد و کافی است برای جواب نهایی بین دو شعاعی که به عنوان جواب خود می‌دهند، شعاع کم‌تر را به عنوان جواب نهایی الگوریتم بدهیم.

\قسمت{$k$-مرکز‌ با داده‌های پرت}

در دنیای واقعی در میان داده‌ها، داده‌های دارای اریب‌ وجود دارند که اگر امکان تشخیص و حذف آن‌ها در حین جمع‌آوری داده‌ها وجود داشت، شعاع مسئله‌ی $k$-مرکز به میزان قابل توجهی کاهش پیدا می‌کرد و شهود بسیار بهتری از دسته‌ها ارائه می‌داد.
در عمل نیز حذف داده‌هایی که به هیچ دسته‌ای شباهت ندارند، معقول به نظر می‌رسد، زیرا هدف به دست آوردن دسته‌بندی برای کلیت نقاط است و نه دسته‌بندی که تمام نقاط در آن می‌گنجند.
همان‌طور که در شکل \رجوع{شکل:k-center-with-outlier} می‌بینید، تنها حذف دو نقطه که نسبت به بقیه نقاط داده‌ی اریب حساب می‌شوند، دسته‌بندی بسیار قابل قبول‌تری  را نتیجه می‌دهد.
با چنین رویکردی، باید تعدادی نقطه که با بقیه‌ی نقاط فاصله‌ی زیادی دارند از داده‌های ورودی حذف شده و سپس به عنوان ورودی مسئله‌ی $k$-مرکز دسته‌های مرتب را از آن استخراج کرد. 

\شروع{شکل}[ht]
\centerimg{k-center-with-outlier}{12cm}
\شرح{کاهش قابل توجه شعاع مسئله‌ی $2$-مرکز با حذف تنها دو نقطه}
\برچسب{شکل:k-center-with-outlier}
\پایان{شکل}

مسئله‌ی $k$-مرکز با داده‌های پرت، بسیار مشابه مسئله‌ی استقرار تجهیزات است.
در مسئله‌ی استقرار تجهیزات، هدف استقرار چند مرکز ارائه‌دهنده‌ی خدمات است که هزینه‌ی استقرار به علاوه‌ی انتقال تجهیزات از مراکز ارائه‌دهنده به مکان‌های متقاضی کمینه گردد.
تعریف رسمی این مسئله در زیر آمده است:

\شروع{مسئله}
\مهم{(استقرار تجهیزات)}
مجموعه‌ای نقطه به عنوان مکان‌های مجاز برای استقرار تجهیزات داده شده است.
هزینه‌ی استقرار تجهیزات در نقطه‌ی $i$اُم را برابر با $f_i$ در نظر بگیرید.
مجموعه‌ای از نقاط نیز که متقاضی تجهیزات هستند نیز داده‌شده است.
به ازای هر متقاضی $j$ و محل استقرار تجهیزات $i$، $d(i, j)$ برابر هزینه انتقال تجهیزات از محل استقرار به متقاضی است.
مجموعه‌ای $k$ عضوی به نام $K$ انتخاب کنید به طوری که هزینه کلی را کمینه نماید:
$$\sum_{i \in K} f_i + \sum_{\lr{\text{all customers}}}{\min_{i \in K} d(i, j)}$$

\پایان{مسئله}

 گونه‌های مختلفی از مسئله‌ی استقرار تجهیزات تعریف شده است.
از جمله‌ی آن می‌توان به گونه‌های زیر اشاره نمود:

\شروع{فقرات}

\فقره{هر مرکز ارائه‌دهنده حداکثر به تعداد مشخصی از متقاضیان می‌تواند تجهیزات انتقال دهد.
در واقع در این روش سعی در استقرار متوازن تجهیزات است به‌طوری‌که متناسب با قدرت ارائه‌ی تجهیزات به هر مرکز تقاضا تخصیص یابد.}

\فقره{هزینه‌ی استقرار مرکز ارائه‌دهنده متناسب با تعداد متقاضیانی که پاسخ می‌دهد است.
در این روش، معمولاً هر چه تعداد تقاضاهای یک مرکز بیش‌تر شود هزینه‌ی استقرار یا ساخت آن برای تأمین چنین میزان درخواستی بالاتر می‌رود.}

\فقره{هر متقاضی میزانی جریمه بابت عدم دریافت تقاضای خود مشخص می‌کند.
در این روش، هر متقاضی در صورت عدم دریافت تجهیزات مورد نیاز، میزانی جریمه مطالبه می‌کند و هدف کاهش مجموع هزینه‌ها به علاوه‌ی هزینه‌های قبلی است.}

\فقره{تعدادی از متقاضیان را می‌توان بدون پرداخت جریمه پوشش نداد.
در این حالت، امکان چشم‌پوشی از تعدادی از متقاضیان وجود دارد ولی امکان تخطی از‌ محدودیت تعداد آن‌ها وجود ندارد.}

\پایان{فقرات}

اگر به دو گونه‌ی آخر توجه بیش‌تری کنید، به شباهتشان به مسئله‌ی $k$-مرکز با داده‌های پرت پی خواهید برد.
می‌توان نشان داد که مسئله‌ی $k$-مرکز با $z$ داده‌ی پرت از لحاظ پیچیدگی محاسباتی هم‌ارز استقرار تجهیزات با امکان عدم پوشش $z$ متقاضی است.
همان‌طور که در زیربخش قبلی دیدیم، هیچ الگوریتمی با ضریب تقریب کم‌تر از $2$ برای مسئله‌ی $k$-مرکز در حالت کلی وجود ندارد مگر این‌که $P = NP$ باشد.
برای مسئله‌ی $k$-مرکز با داده‌های پرت، اگر تعداد مجاز داده‌های پرت صفر باشد، مسئله به همان مسئله‌ی $k$-مرکز تبدیل می‌گردد.

گونه‌ی مشابهی با مسئله‌ی $k$-مرکز با داده‌های پرت تعریف می‌گردد که در آن تعدادی از نقاط نمی‌توانند به عنوان مرکز انتخاب شوند.
به این‌گونه، مسئله‌ی $k$-مرکز با داده‌های پرت و داده‌های ممنوعه (برای قرارگیری مرکز در آن‌ها) تعریف می‌شود.
در مرجع \مرجع{charikar2001algorithms}، ثابت شده است که این مسئله در حالت کلی با ضریب کم‌تر از $3$ قابل تقریب‌پذیر نیست مگر آن‌که $P = NP$ باشد. 

\شروع{قضیه}

فرض کنید نقطه‌هایی دلخواه از یک متریک دلخواه داده شده‌اند.
مسئله‌ی $k$-مرکز با داده‌های پرت، که در آن، بعضی از رئوس امکان مرکز شدن ندارند (رئوس ممنوعه)، را نمی‌توان با ضریب تقریبی کم‌تر از $3$ تقریب زد.

\شروع{اثبات}

ایده‌ی ارائه شده برای این کران پایین، بسیار مشابه با ایده‌ی ارائه شده برای مسئله‌ی $k$-مرکز در فضای اقلیدسی است \مرجع{bern1996approximation}.
در واقع در این راه‌حل، مسئله‌ی بیشینه پوشش مجموعه‌ای، به یک گراف دوبخشی متریک تبدیل می‌گردد که در آن وزن تمام یال‌ها برابر یک است.
با استفاده از گراف ساخته شده، نشان داده می‌شود که اگر مسئله‌ی $k$-مرکز با داده‌های پرت و مراکز ممنوعه، دارای الگوریتمی با ضریب تقریب کم‌تر از ۳ داشته باشد (کوتاه‌ترین مسیر بین دو رأس غیر مجاور در دو بخش مختلف)، آنگاه می‌توان مسئله‌ی بیشینه پوشش مجموعه‌ای را در زمان چندجمله‌ای حل نمود.
برای مشاهده‌ی اثبات کامل‌تر می‌توانید به مرجع \مرجع{charikar2001algorithms} مراجعه کنید.
\پایان{اثبات}

\پایان{قضیه}

الگوریتمی که چریکار\پاورقی{Charikar} سایرین در این مقاله ارائه داده‌اند یک الگوریتم $3$-تقریب برای مسئله‌ی $k$-مرکز با داده‌های پرت است.
در ابتدا، الگوریتم چریکار، تمام شعاع‌های ممکن را پیدا می‌کند.
در حالت مسئله‌ی $k$-مرکز گسسته، کافی است تمام فاصله‌های بین دوبه‌دوی نقاط را به عنوان کاندیدا در نظر گرفت که تعدادشان از مرتبه‌ی $\cO(n^2)$ است و کافی است برروی گزینه‌های به دست آمده، جست‌وجوی دودویی زد.
در صورتی که مسئله‌ی $k$-مرکز در حالت پیوسته مد نظر باشد، به ازای هر $d+1$ نقطه‌ی دلخواه، یک کاندید اضافه می‌گردد که تعدادشان از مرتبه‌ی $\cO(n^{(d+1)})$ است.
حال اگر شعاع $r$ی داشتی باشیم که بزرگ‌تر مساوی شعاع بهینه باشد، چریکار یک الگوریتم ساده با شعاع حداکثر $3r$ ارائه می‌دهد که تمام نقاط به غیر از حداکثر $z$ نقطه را می‌پوشاند و اگر الگوریتم چریکار نتوانست با شعاع $3r$ همه‌ی نقاط به جز حداکثر $z$ نقطه را بپوشاند، بنابراین فرض اولیه‌ی ما اشتباه بوده است و $r$ از شعاع بهینه کم‌تر است.

\شروع{تعریف}

به ازای هر نقطه‌ی $v_i \in V$، $G_i$($E_i$ به طور مشابه) را برابر مجموعه نقاطی در نظر بگیرید که در فاصله‌ی حداکثر $r$($3r$ به طور مشابه) از $v_i$ قرار دارند. $G_i$ را توپ به شعاع $r$ و $E_i$ را به عنوان توپ گسترش‌یافته به شعاع $3r$ می‌نامیم.
وزن هر توپ را برابر تعداد نقاط درون آن در نظر می‌گیریم.

\پایان{تعریف}

\شروع{الگوریتم}{اولین الگوریتم با ضریب تقریب $3$ برای $k$-مرکز با $z$ داده‌ی پرت}
\به‌ازای{$i$ از 1 تا $k$}
\دستور{به ازای تمام نقاط، توپ‌ها و توپ‌های گسترش‌یافته را محاسبه کن.}
\دستور{$G_j$ را توپی در نظر بگیر که بیش‌ترین وزن را دارد(بیش‌ترین تعداد نقاط پوشش‌داده نشده را می‌پوشاند).}
\دستور{توپ $E_j$ را به عنوان توپ $i$اُم در نظر بگیر.}
\دستور{ تمام نقاط داخل $E_j$ را به مجموعه نقاط پوشش‌داده شده اضافه کن.}
\پایان‌به‌ازای
\اگر{همه‌ی نقاط به جز حداکثر $z$تای آن‌ها پوشانده شده بودند}
\دستور{$r$ اولیه بزرگ‌تر مساوی $r$ بهینه است.}
\برگردان{$E_j$های انتخاب شده}
\وگرنه{}
\دستور{$r$ اولیه کوچک‌تر از $r$ بهینه است.}
\پایان‌اگر
\پایان{الگوریتم}

الگوریتم ~\رجوع{الگوریتم: اولین الگوریتم با ضریب تقریب $3$ برای $k$-مرکز با $z$ داده‌ی پرت}، یک الگوریتم حریصانه و ساده است که با مقایسه‌ی عملکرد آن با جواب بهینه می‌توان نشان داد که به درستی عمل می‌کند.
برای مشاهده‌ی درستی اثبات، می‌توانید به مرجع  \مرجع{charikar2001algorithms} کنید.
چریکار در ادامه، با استفاده از برنامه‌ریزی خطی\پاورقی{Linear Programming} و گرد کردن\پاورقی{Rounding} جواب، یک الگوریتم $3$-تقریب برای حالت گسسته و یک $4$-تقریب برای حالت پیوسته ارائه می‌دهد.
به علت عدم استفاده از روش برنامه‌ریزی خطی، از بیان جزئیات این قسمت صرف نظر می‌کنیم.
مک‌کاتن\پاورقی{McCutchen} با استفاده از الگوریتم ارائه شده در بالا، یک الگوریتم جویبار داده ارائه داد که متناسب با اینکه از کدام الگوریتم استفاده کند، همان ضریب تقریب را  برای حالت جویبار داده می‌دهد.
ایده‌ی اصلی به‌ کار رفته در این تبدیل، پردازش نقاط به صورت دسته‌های $\cO(kz)$ تایی و در نظر گرفتن نقاط آزاد به عنوان نقاطی که هنوز مطمئن نیستیم نقطه‌ی پرت هستند و نقاطی که مطمئن هستیم باید پوشانده شوند.
این الگوریتم حافظه‌ای از مرتبه‌ی $\cO(\frac{kz}{\epsilon})$ مصرف می‌کند.
برای مشاهده جزئیات بیش‌تر به مرجع \مرجع{mccutchen2008streaming} مراجعه کنید.

تا به اینجا مسئله‌ی $k$-مرکز را در حالت کلی بررسی کردیم.
مسئله‌ی $1$-مرکز با داده‌های پرت به صورت جداگانه به‌وسیله‌ی ضرابی‌زاده و سایرین مورد بررسی قرار گرفته است \مرجع{zarrabi2009streaming}.
در الگوریتم ارائه شده برای حالتی که $z = 1$ است، ضرابی‌زاده یک الگوریتم با ضریب تقریب $1.22 \times \frac{\sqrt{2} + 1}{2} \leq 1.48$ با حافظه‌ی مصرفی $\cO(d)$ ارائه داده است.
به ازای $z$های کلی، الگوریتم دیگری با حافظه‌ی مصرفی $\cO(d^3z)$ و ضریب تقریب $1.22 \times \sqrt{2} \leq 1.73$ ارائه‌شده است.

ایده‌ی اصلی که در این مقاله ارائه شده است، ارائه‌ی یک راه‌کار کلی برای مسائل جویبار داده است.
در این راهکار، یک حافظه‌ی میان‌گیر\پاورقی{Buffer} تعریف می‌شود که هر نقطه از جویبار داده به محض ورود به آن اضافه می‌شود.
در صورتی که حافظه‌ی میان‌گیر، پر گردد، یکی از نقاط آر حافظه استخراج‌شده و به الگوریتم زیرین داده می‌شود.
الگوریتم زیرین یک الگوریتم جویبار داده برای مسئله‌ی $1$-مرکز بدون داده‌های پرت است.
همان‌طور که در زیر بخش مسئله‌ی $k$-مرکز در حالت جویبار داده بیان شد، بهترین الگوریتم موجود یک الگوریتم با حافظه‌ی مصرفی $\cO(d^3z)$ و ضریب تقریب $1.22$ است.

عامل ثانویه‌ای که در ضریب تقریب نهایی الگوریتم تأثیر به سزایی دارد نحوه‌ی استخراج نقطه از حافظه‌ی میان‌گیر است.
فرض کنید $O_x$ مجموعه نقاطی از $P$(جویبار داده) باشند که در جواب بهینه به عنوان داده‌ی پرت انتخاب شده‌اند و $O$ مجموعه نقاطی که به اشتباه از حافظه‌ی میان‌گیر استخراج شدند باشد، اگر داشته باشیم
$$Meb((P - O_x) \cup O) \leq \beta Meb(P - O_x)$$
در نتیجه ضریب تقریب نهایی برابر $1.22 \beta$ خواهد بود.
نکته‌ی قابل توجه در اینجا، تأثیر طول حافظه‌ی میان‌گیر، در ضریب $\beta$ است.

در حالت کلی $z$، ایده‌ی اصلی برای استخراج یک نقطه از حافظه‌ی میان‌گیر، نقطه‌ی مرکزی\پاورقی{Centerpoint} نقاط داخل حافظه‌ی میان‌گیر است.
در واقع نزدیک‌ترین نقطه به نقطه‌ی مرکزی نقاط داخل حافظه‌ی میان‌گیر، استخراج می‌گردد و ثابت می‌شود با این شیوه‌ی استخراج $\beta \leq \sqrt{2}$ خواهد بود.
برای مشاهده‌ی اثبات و جزئیات بیش‌تر به مرجع \مرجع{zarrabi2009streaming} مراجعه کنید.

در فصل آتی، در ابتدا به پیشرفت‌هایی که برای مسئله‌ی $1$-مرکز در حالت جویبار داده با داده‌های پرت در این پایان‌نامه ارائه شده است، خواهیم پرداخت.
سپس برای مسئله‌ی $2$-مرکز در حالت جویبار داده با داده‌های پرت، اولین کار موجود را ارائه می‌دهیم که بهبود قابل توجهی نسبت به حالت کلی است.

