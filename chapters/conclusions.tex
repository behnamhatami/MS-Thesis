
\فصل{نتیجه‌گیری}

در این پایان‌نامه گونه‌های مختلفی از دو مسئله‌ی $1$-مرکز‌ و $2$-مرکز در حالت‌ جویبار داده  مورد بررسی قرار گرفت. همان‌طور که بیان شده است این دو مسئله و حالت کلی آن، استفاده‌ی زیادی در علوم کامپیوتر دارند. از طرفی به علت افزایش روز افزون داده‌ها مدل‌ جویبار داده‌ی مسئله بسیار کاربردی می‌گردد.

در این پایان‌نامه در ابتدا مسئله‌ی $1$-مرکز با داده‌ی پرت در حالت جویبار داده ارائه گردید. برای این مسئله دو الگوریتم متفاوت ارائه گردید. الگوریتم اول، با مصرف حافظه‌ی $\cO(z^2d)$ جوابی با ضریب تقریب $2$ برای مسئله‌ی $1$-مرکز با داده‌ی پرت ارائه می‌دهد. الگوریتم ارائه شده، نسبت به الگوریتم ضرابی‌زاده \مرجع{zarrabi2009streaming} الگوریتمی ساده‌تر است و در صورتی که $d$ از $z$ بزرگ‌تر باشد، حافظه‌ی مصرفی را نیز کاهش می‌دهد. در الگوریتم دوم، با استفاده از ایده‌ی مطرح شده در مرجع \مرجع{ahn2014computing}، الگوریتمی با ضریب تقریب $1.8 + \epsilon$ ارائه می‌شود.

در بخش دوم، مسئله‌ی $1$-مرکز پوشاننده بدون داده‌ی پرت مورد بررسی قرار می‌گیرد. برای این مسئله، یک الگوریتم با ضریب تقریب $1.8 + \epsilon$ ارائه می‌گردد. در ادامه، الگوریتم دیگری با استفاده از الگوریتم \مرجع{agarwal2010streaming}، یک الگوریتم با ضریب تقریب $1.7$ ارائه می‌شود. با استفاده از همین الگوریتم، در بخش بعدی برای مسئله‌ی $1$-مرکز با $z$ داده‌ی پرت، الگوریتمی با ضریب تقریب $1.7$ ارائه می‌شود.

در نهایت، در بخش سوم، مسئله‌ی $2$-مرکز با $z$ داده‌ی پرت مورد بررسی قرار می‌گیرد و الگوریتمی با ضریب تقریب $1.8 + \epsilon$ ارائه می‌شود که نسبت به الگوریتم پیشین برای $k$ کلی، با ضریب تقریب $4 + \epsilon$ بهبود قابل توجهی محسوب می‌شود.

\قسمت{کارهای آتی}

همان‌طور که در بخش قبلی به آن اشاره شد، دو الگوریتم با ضریب تقریب $1.8 + \epsilon$ برای مسئله‌ی $2$-مرکز با $z$ داده‌ی پرت ارائه شد.
حدسی که وجود دارد، امکان تعمیم دو الگوریتم داده شده به الگوریتمی کلی برای مسئله‌ی $k$-مرکز با $z$ داده‌ی پرت به ازای $k$ دلخواه است.

از طرفی دیگر، هیچ کران پایینی غیر از کران پایین $1.2$ که برای مسئله‌ی $1$-مرکز به وسیله‌ی آگاروال ارائه شده است \مرجع{agarwal2010streaming} برای مسئله‌ی $1$-مرکز و $2$-مرکز با $z$ داده‌ی پرت وجود ندارد.
بنابراین حوزه‌ای که امکان بهبود دارد، کم کردن فاصله‌ی بین $1.8$ (بهترین الگوریتم موجود) و $1.2$ (بهترین کران پایین) است که ممکن است با اثبات کران پایین بالا‌تر یا ارائه‌ی الگوریتم جدیدی که ضریب تقریب کم‌تر از $1.8$ داشته باشد ممکن گردد.

از طرفی ممکن است، بتوان الگوریتم موجود را بدون تغییر ضریب تقریب، از لحاظ میزان حافظه‌ی مصرفی، میزان زمان مورد نیاز برای به‌روزرسانی و زمان مورد نیاز برای پاسخ‌گویی به پرسمان بهبود بخشید.

مسئله‌ی $k$-مرکز پوشاننده به عنوان مسئله‌ای که کم‌تر مورد توجه قرار گرفته است را نیز مورد بررسی بیش‌تری قرار داد و الگوریتم‌های بهتری از لحاظ ضریب تقریب یا حافظه‌ی مصرفی و زمان به‌روزرسانی ارائه داد.
از طرفی در حال حاضر، کران پایینی غیر از ضریب تقریب $1.2$ وجود ندارد که ممکن است بتوان آن را بهبود بخشید.
