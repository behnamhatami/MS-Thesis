
\فصل{مفاهیم اولیه}

در این فصل به تعریف و بیان مفاهیم پایه‌ا‌ی مورد استفاده در فصل‌های بعد می‌پردازیم. با توجه به مطالب مورد نیاز در فصل‌های آتی، مطالب این فصل به سه بخش، مسائل ان‌پی-سخت، الگوریتم‌های تقریبی و الگوریتم‌های جویبارداده تقسیم می‌شود.

\قسمت{مسائل ان‌پی-سخت}
یکی از اولین سوال‌های بنیادی مطرح در علم کامپیوتر، اثبات \مهم{عدم حل پذیری} بعضی از مسائل است. به عنوان نمونه، می‌توان از دهمین مسئله‌ی هیلبرت\پاورقی{Hilbert} در گنگره‌ی ریاضی یاد کرد. هیلبرت این مسئله را اینگونه بیان کرذ:‌ ``فرآیندی طراحی کنید که در تعداد متناهی گام بررسی کند که آیا یک چندجمله‌ای، ریشه‌ی صحیح\پاورقی{integral root} دارد یا خیر .``. با مدل محاسباتی که به‌وسیله‌ی تورینگ\پاورقی{Touring} ارائه‌ شد، این مسئله معادل پیدا کردن الگوریتمی برای این مسئله است که اثبات می‌شود امکان‌پذیر نیست \مرجع{sipser2012introduction}. برخلاف مثال بالا، عمده‌ی مسائل علوم کامپیوتر از نعد بالا نیستند و برای طیف وسیعی از آن‌ها، الگوریتم‌های پایان‌پذیر وجود دارد. بیش‌تر تمرکز علوم کامپیوتر هم بر روی چنین مسائلی است.

اگر چه برای اکثر مسائل الگوریتمی پایان‌پذیر وجود دارد، اما وجود چنین الگوریتمی لزومی بر حل شدن چنین مسائلی نیست. در عمل، علاوه بر وجود الگوریتم، میزان کارامدی\پاورقی{Efficiency} الگوریتم نیز مطرح می‌گردد. به طور مثال، اگر الگوریتم حل یک مسئله مرتبه‌ی بالا یا نمایی داشته باشد، الگوریتم ارائه شده برای آن مسئله برای ورودی‌های نسبتا بزرگ قابل اجرا نیست و نمی‌توان از آن‌ها برای حل مسئله استفاده کرد. برای تشخیص و تمیز کارآمدی الگوریتم‌های مختلف و همچنین میزان سختی مسائل در امکان ارائه‌ی الگوریتم‌های کارآمد یا غیرکارآمد، نظریه‌ی پیچیدگی\پاورقی{Complexity theory}، دسته‌بندی‌های مختلفی برای سختی مسائل و حل‌پذیری آن‌ها ارائه داده است تا بتوان به‌طور رسمی\پاورقی{formal} در مورد این معیار‌ها صحبت کرد. برای دسته‌بندی مسائل در نظریه‌ی پیچیدگی، ابتدا آن‌ها را به صورت تصمیم‌پذیر بیان می‌کنند.

\مسئله{\مهم{(مسائل تصمیم‌گیری)}\پاورقی{Decision problems} به دسته‌ای از مسائل گفته می‌شود که پاسخ آن‌ها تنها بله یا خیر است.}

به عنوان مثال، اگر بخواهیم مسئله‌ی $1$-مرکز در فضای $\IR^d$ را به صورت تصمیم پذیر بیان کنیم، به مسئله‌ی زیر می‌رسیم:

\مسئله{\مهم{(نسخه‌ی تصمیم پذیر $1$-مرکز)} مجموعه‌ی نقاط در فضا $\IR^d$ و شعاع $r$ داده شده است، آیا دایره‌ای به شعاع $r$ وجود دارد که تمام نقاط را بپوشاند؟}

در نظریه‌ی پیچیدگی، می‌توان گفت عمده‌ترین دسته‌بندی موجود، دسته‌بندی مسائل تصمیم‌گیری به مسائل پی $(P)$ و  ان‌پی $(NP)$ است. رده‌ی مسائل $P$، شامل تمامی مسائل تصمیم‌گیری است که راه‌حل چندجمله‌ای برای آن‌ها وجود دارد. از طرفی رده‌ی مسائل $NP$، شامل تمامی مسائل تصمیم‌گیری است که در زمان چندجمله‌ای قابل صحت‌سنجی\پاورقی{verifiable} اند. تعریف صحت‌سنجی در نظریه پیچیدگی، یعنی اگر جواب مسئله‌ی تصمیم‌گیری بله باشد، می‌توان اطلاعات اضافی با طول چندجمله‌ای ارائه داد، که در زمان چند‌جمله‌ای از روی آن، می‌توان جواب بله الگوریتم را تصدیق نمود. به طور مثال، برای مسئله‌ی $1$-مرکز، کافی است به عنوان تصدیق جواب بله، مرکز دایره‌ی پوشاننده را الگوریتم ارائه دهد. در این صورت، می‌توان با مرتبه‌ی خطی بررسی نمود که تمام نقاط داخل این دایره قرار می‌گیرند یا نه. برای مطالعه‌ی بیش‌تر و تعاریف دقیق‌تر می‌توان به \مرجع{sipser2012introduction} مراجعه نمود.
