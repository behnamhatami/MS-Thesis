
\فصل{نتایج جدید}

در این فصل نتایج جدید به‌دست‌آمده در پایان‌نامه توضیح داده می‌شود. این فصل به سه بخش تقسیم می‌شود. قسمت اول به بیان مقدمات و نمادگذاری‌های مورد نیاز برای بخش‌های بعدی می‌پردازیم. در فصل دوم، به بررسی راه‌حل‌های ارائه‌شده برای مسئله‌ی $1$-مرکز با $z$ داده‌ی پرت در حالت جویبار داده می‌پردازیم. این بخش به سه زیربخش تقسیم می‌شود. در زیربخش اول، دو الگوریتم جدید ارائه می‌دهیم. الگوریتم اول، با حافظه‌ی مصرفی $\cO(z^2d)$ و ضریب تقریب $2$ ارائه می‌دهد که حافظه‌ی مصرفی الگوریتم ضرابی‌زاده و سایرین \مرجع{zarrabi2009streaming} را در صورتی که ابعاد فضا بیش‌تر از $z$ باشد بهبود می‌بخشد. از طرفی الگوریتم ارائه شده بسیار ساده‌تر از الگوریتم ضرابی‌زاده است. الگوریتم دوم یک الگوریتم با ضریب تقریب $1.8 + \epsilon$ و حافظه‌ی مصرفی $\cO(\frac{z^2}{\epsilon})$ است. در زیر بخش دوم، به بررسی مسئله‌ی $1$-مرکز پوشاننده بدون داده‌ی پرت می‌پردازیم، به‌طوری‌که نه تنها می‌خواهیم تمام نقاط ورودی پوشیده شود، بلکه می‌خواهیم تمام دایره‌ی بهینه نیز به طور کامل پوشیده شود. برای این مسئله، دو راه‌حل متفاوت یکی با ضریب تقریب $1.8$ و دیگری با ضریب تقریب $1.7$ ارائه می‌شود. در زیربخش بعدی، با استفاده از زیربخش قبلی، الگوریتمی با ضریب تقریب $1.7$ برای مسئله‌ی $1$-مرکز با $z$ داده‌ی پرت ارائه می‌دهیم(برای حالتی که $z$ ثابت است). 

در بخش سوم، به بررسی مسئله‌ی $2$-مرکز با $z$-داده‌ی پرت می‌پردازیم. در این قسمت با تقسیم‌بندی مسئله به دو حالت، دو الگوریتم متفاوت ارائه داده و آن‌ها به صورت موازی اجرا می‌کنیم. هر دوی این الگوریم‌ها ضریب تقریب $1.8 + \epsilon$ دارند و حافظه‌ی مصرفی در کل برابر است با 
$\cO(dz^2 (d^2 + \frac{z}{\epsilon}))$.
این اولین الگوریتم ارائه‌شده‌ایست که بعد از الگوریتمی که برای $k$ کلی، با ضریب تقریب $4+\epsilon$ ارائه شده است، که بهبود قابل توجهی محسوب می‌شود.

\قسمت{نماد‌گذاری‌ها و تعاریف اولیه}

\قسمت{مسئله‌ی $1$-مرکز در حالت جویبار داده}

\زیرقسمت{مسئله‌ی $1$-مرکز با داده‌های پرت در حالت جویبار داده}

\زیرزیرقسمت{الگوریتم تقریبی با ضریب تقریب $2$}

\زیرزیرقسمت{الگوریتم تقریبی با ضریب تقریب $1.8$}

\زیرقسمت{مسئله‌ی $1$-مرکز‌ پوشاننده در حالت جویبار داده}

\زیرقسمت{مسئله‌ی $1$-مرکز با داده‌های پرت در حالت جویبار داده با تعداد داده‌های پرت ثابت}

\قسمت{مسئله‌ی $2$-مرکز با داده‌های پرت در حالت جویبار داده}

\زیرقسمت{حالت $\delta^* \leq \alpha r^*$}

\زیرزیرقسمت{الگوریتم اصلی}

\زیرزیرقسمت{پیدا کردن $r'$}

\زیرقسمت{حالت $\delta^* \geq \alpha r^*$}

\زیرزیرقسمت{الگوریتم اصلی}

\زیرزیرقسمت{پاسخ‌گویی به پرسمان‌ها}

