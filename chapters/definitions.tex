
\فصل{مفاهیم اولیه}

در این فصل به تعریف و بیان مفاهیم پایه‌ا‌ی مورد استفاده در فصل‌های بعد می‌پردازیم. با توجه به مطالب مورد نیاز در فصل‌های آتی، مطالب این فصل به سه بخش، مسائل ان‌پی-سخت، الگوریتم‌های تقریبی و الگوریتم‌های جویبارداده تقسیم می‌شود.

\قسمت{مسائل ان‌پی-سخت}
یکی از اولین سوال‌های بنیادی مطرح در علم کامپیوتر، اثبات \مهم{عدم حل پذیری} بعضی از مسائل است. به عنوان نمونه، می‌توان از دهمین مسئله‌ی هیلبرت\پاورقی{Hilbert} در گنگره‌ی ریاضی یاد کرد. هیلبرت این مسئله را اینگونه بیان کرذ:‌ ``فرآیندی طراحی کنید که در تعداد متناهی گام بررسی کند که آیا یک چندجمله‌ای، ریشه‌ی صحیح\پاورقی{integral root} دارد یا خیر .``. با مدل محاسباتی که به‌وسیله‌ی تورینگ\پاورقی{Touring} ارائه‌ شد، این مسئله معادل پیدا کردن الگوریتمی برای این مسئله است که اثبات می‌شود امکان‌پذیر نیست \مرجع{sipser2012introduction}. برخلاف مثال بالا، عمده‌ی مسائل علوم کامپیوتر از نعد بالا نیستند و برای طیف وسیعی از آن‌ها، الگوریتم‌های پایان‌پذیر وجود دارد. بیش‌تر تمرکز علوم کامپیوتر هم بر روی چنین مسائلی است.

اگر چه برای اکثر مسائل الگوریتمی پایان‌پذیر وجود دارد، اما وجود چنین الگوریتمی لزومی بر حل شدن چنین مسائلی نیست. در عمل، علاوه بر وجود الگوریتم، میزان کارامدی\پاورقی{Efficiency} الگوریتم نیز مطرح می‌گردد. به طور مثال، اگر الگوریتم حل یک مسئله مرتبه‌ی بالا یا نمایی داشته باشد، الگوریتم ارائه شده برای آن مسئله برای ورودی‌های نسبتا بزرگ قابل اجرا نیست و نمی‌توان از آن‌ها برای حل مسئله استفاده کرد. برای تشخیص و تمیز کارآمدی الگوریتم‌های مختلف و همچنین میزان سختی مسائل در امکان ارائه‌ی الگوریتم‌های کارآمد یا غیرکارآمد، نظریه‌ی پیچیدگی\پاورقی{Complexity theory}، دسته‌بندی‌های مختلفی برای سختی مسائل و حل‌پذیری آن‌ها ارائه داده است تا بتوان به‌طور رسمی\پاورقی{formal} در مورد این معیار‌ها صحبت کرد. برای دسته‌بندی مسائل در نظریه‌ی پیچیدگی، ابتدا آن‌ها را به صورت تصمیم‌پذیر بیان می‌کنند.

\مسئله{\مهم{(مسائل تصمیم‌گیری)}\پاورقی{Decision problems} به دسته‌ای از مسائل گفته می‌شود که پاسخ آن‌ها تنها بله یا خیر است.}

به عنوان مثال، اگر بخواهیم مسئله‌ی $1$-مرکز در فضای $\IR^d$ را به صورت تصمیم پذیر بیان کنیم، به مسئله‌ی زیر می‌رسیم:

\مسئله{\مهم{(نسخه‌ی تصمیم پذیر $1$-مرکز)} مجموعه‌ی نقاط در فضا $\IR^d$ و شعاع $r$ داده شده است، آیا دایره‌ای به شعاع $r$ وجود دارد که تمام نقاط را بپوشاند؟}

در نظریه‌ی پیچیدگی، می‌توان گفت عمده‌ترین دسته‌بندی موجود، دسته‌بندی مسائل تصمیم‌گیری به مسائل پی $(P)$ و  ان‌پی $(NP)$ است. رده‌ی مسائل $P$، شامل تمامی مسائل تصمیم‌گیری است که راه‌حل چندجمله‌ای برای آن‌ها وجود دارد. از طرفی رده‌ی مسائل $NP$، شامل تمامی مسائل تصمیم‌گیری است که در زمان چندجمله‌ای قابل صحت‌سنجی\پاورقی{verifiable} اند. تعریف صحت‌سنجی در نظریه پیچیدگی، یعنی اگر جواب مسئله‌ی تصمیم‌گیری بله باشد، می‌توان اطلاعات اضافی با طول چندجمله‌ای ارائه داد، که در زمان چند‌جمله‌ای از روی آن، می‌توان جواب بله الگوریتم را تصدیق نمود. به طور مثال، برای مسئله‌ی $1$-مرکز، کافی است به عنوان تصدیق جواب بله، مرکز دایره‌ی پوشاننده را الگوریتم ارائه دهد. در این صورت، می‌توان با مرتبه‌ی خطی بررسی نمود که تمام نقاط داخل این دایره قرار می‌گیرند یا نه. برای مطالعه‌ی بیش‌تر و تعاریف دقیق‌تر می‌توان به \مرجع{sipser2012introduction} مراجعه نمود.

همان‌طور که می‌دانید درستی یا عدم درستی $P \subset NP$ از جمله معروف‌ترین مسائل حل نشده\پاورقی{Open problem} در نظریه پیچیدگی است. حدس بسیار قوی وجود دارد که $P \neq NP$ و بسیاری ار مسائل، با این فرض حل می‌شوند و در صورتی که زمانی، خلاف این فرض اثبات گردد، آن‌گاه قسمت عمده‌ای از علوم کامپیوتر زیر سوال می‌رود.

در نظریه‌ی پیچیدگی، برای دسته‌بندی مسائل، یکی از روش‌های دسته‌بندی کاهش چند‌جمله‌ای\پاورقی{Polynomial Reduction} مسائل به یک‌دیگر است. 

\تعریف{می‌گوییم مسئله‌ی $A$ در زمان چندجمله‌ای به مسئله‌ی $B$ کاهش می‌یابد، اگر وجود داشته باشد الگوریتم چندجمله‌ای $C$ که به ازای هر ورودی $\alpha$ برای مسئله‌ی $A$، یک ورودی $\beta$ در زمان چندجمله‌ای برای مسئله‌ی $B$ بسازد، به طوری که $A$، $\alpha$ را می‌پذیرد اگر و تنها اگر $B$، $\beta$ را بپذیرد. در این‌جا منظور از پذیرفتن جواب بله به ورودی است.}

از این به بعد برای سادگی به جای \مهم{کاهش چندجمله‌ای} از واژه‌ی \مهم{کاهش} استفاده می‌کنیم. در پی جست‌جوهایی که برای برابری دسته‌ی پی و ان‌پی صورت گرفت، مجموعه‌ای از مسائل که عمدتا داخل ان‌پی هستند استراج گردید که اگر ثابت شود یکی از آن‌ها متعلق به پی است، آن‌گاه تمام مسائل دسته‌ی ان‌پی متعلق به پی خواهند بود و در نتیجه $P = NP$ خواهد بود. به این مجموعه مسائل ان‌پی-سخت می‌گویند. در واقع مسائل این دسته، مسائلی هستند که تمام مسائل داخل دسته‌ی ان‌پی، به آن‌ها کاهش می‌یابند.

کوک و لوین در قضیه‌ای به نام \مهم{قضیه‌ی کوک-لوین} ثابت کردند مسئله‌ی صدق‌پذیری\پاورقی{Satisfiability problem} یک مسئله‌ی ان‌پی-سخت است \مرجع{sipser2012introduction}. با پایه قرار دادن این اثبات و استفاده از تکنیک کاهش، اثبات ان‌پی-سخت بودن سایر مسائل، بسیار ساده‌تر گردید. در ادامه مسئله‌ی پوشش رأسی\پاورقی{Vertex Coverage} را تعریف می‌کنیم.

\زیرقسمت{پوشش رأسی}
در این پایان‌نامه، از این مسئله به عنوان مسئله‌ی پایه برای اثبات ان‌پی-سخت بودن مسئله‌ی $k$-مرکز استفاده می‌شود. تعریف این مسئله مطابق زیر است:

\تعریف{گراف بدون جهت $G(V, E)$ داده شده است. هدف مسئله پیدا کردن مجموعه‌ی $S \subset VS$ با کم‌ترین تعداد اعضا است به طوری که هر رأس $v \in V$ در یکی از شرایط زیر صدق کند:

\شروع{فقرات}

\فقره{$v \in S$}

\فقره{وجود دارد رأسی $u \in S$ به طوری $(v, u) \in E$}

\پایان{فقرات}

به عبارت ساده‌تر هر رأسی یا خودش یا یک از همسایگانش داخل مجموعه‌ی $S$ قرار دارد.
}

نسخه‌ی تصمیم‌گیری این مسئله به این گونه تعریف می‌شود که آیا گراف داده‌شده دارای پوشش رأسی با اندازه‌ی $k$ است.

\قضیه{مسئله‌ی پوشش‌ رأسی، یک مسئله‌ی ان‌پی-سخت است.}

\شروع{اثبات}

برای مشاهده‌ی اثبات ان‌پی-سخت بودن مسئله‌ی پوشش رأسی, نیاز به زنجیره‌ای از مسائل که از مسئله‌ی صدق‌پذیری شروع می‌شود است. برای مطالعه‌ی روند اثبات به مرجع \مرجع{sipser2012introduction} مراجعه کنید. 

\پایان{اثبات}

\قسمت{الگوریتم‌های تقریبی}

تا این‌جا با رده‌بندی مسائل به دو دسته‌ی پی و ان‌پی آشنا شدیم. نه تنها مسائل ان‌پی، بلکه بعضی از مسائل پی نیز دارای الگوریتم کارامدی نیستند. در عمل، عمده‌ی مسائل کاربردی به این دسته تعلق می‌گیرند و هیچ راه‌حل یا الگوریتم کارامدی برایشان وجود ندارد. یکی از رویکردهای رایج در برابر چنین مسائلی، صرف نظر کردن از دقت راه‌حل‌هاست. به طور مثال راه‌حل‌های مکاشفه‌ای\پاورقی{heuristic} گوناگونی برای مسائل مختلف ان‌پی بیان شده است. این راه‌حل‌ها بدون این‌که تضمین کنند راه‌حل خوبی ارائه می‌دهند یا حتی جوابشان به جواب بهینه نزدیک است، اما با معیار‌هایی سعی در خوب عمل کردن دارند و در عمل معمولا برای دسته‌ای از کاربردها پاسخ قابل قبولی ارائه می‌دهند. 

مشکل عمده‌ی راه‌حل راه‌حل‌های مکاشفه‌ای، عدم امکان استفاده‌ از آن‌ها برای تمام کاربردها است. بنابراین در رویکرد دوم که اخیرا مطرح شد، سعی در ارائه‌ی الگوریتم‌های مکاشفه‌ای شد که تضمین می‌کنند اختلاف زیادی با الگوریتمی که جواب بهینه می‌دهد، ندارند. در واقع این الگوریتم‌ها همواره و در هر شرایطی، تقریبی از جواب بهینه را ارائه می‌دهند. به چنین الگوریتم‌هایی، \مهم{الگوریتم‌های تقریبی}\پاورقی{Approximation Algorithm} می‌گویند. ضریب تقریب یک الگوریتم تقریبی، به حداکثر نسبت جواب الگوریتم تقریبی  به جواب بهینه گفته می‌شود.

الگوریتم‌های تقریبی تنها به علت محدودیت کارایی الگوریتم‌هایی که جواب بهینه می‌دهند، مورد استفاده قرار نمی‌گیرند. هر نوع محدودیتی ممکن است، استفاده از الگوریتم تقریبی را نسبت به الگوریتمی که جواب بهینه می‌دهد، مقرون به صرفه کند. به طور مثال از جمله عوامل دیگری که ممکن است باعث این انتخاب شود میزان حافظه‌ی مصرفی باشد. برای طیف وسیعی از مسائل، کمبود حافظه، باعث می‌شود الگوریتم‌هایی با حافظه‌ی مصرفی کم‌تر طراحی شود که به دقت الگوریتم‌های بهینه عمل نمی‌کند. معمولا چنین الگوریتم‌هایی حافظه‌ی مصرفی از مرتبه‌ی زیرخطی\پاورقی{sublinear} دارند و به همین دلیل برای داده‌های حجیم بسیار کاربرد دارند.


\شروع{لوح}[t]
\وسط‌چین
\شروع{جدول}{|c|c|}
\خط‌پر
مسئله & کران پایین تقریب‌پذیری
\\
\خط‌پر
\خط‌پر
پوشش رأسی &‪1.3606 \مرجع{dinur2005hardness} \\
$k$-مرکز & 2\مرجع{vazirani2013approximation} \\ 
$1$-مرکز در حالت جویبار داده & $\frac{1 + \sqrt{2}}{2}$ \مرجع{agarwal2010streaming} \\
$k$-مرکز با نقاط پرت و نقاط اجباری & 3\مرجع{charikar2001algorithms}\\
\خط‌پر
\پایان{جدول}

\شرح{نمونه‌هایی از ضرایب تقریب برای مسائل بهینه‌سازی}
\برچسب{جدول:تقریب‌پذیری}
\پایان{لوح}

\زیرقسمت{میزان تقریب‌پذیری مسائل}

همان‌طور که تا این‌جا دیدیم، یکی ار راه‌کارهایی که برای کارآمد کردن راه‌حل ارائه شده برای یک مسئله است استفاده از الگوریتم‌های تقریبی برای حل آن مسئله است. یکی از عمده‌ترین دغدغه‌های مطرح در الگوریتم‌های تقریبی کاهش ضریب تقریب است و یا حتی امکان ارائه‌ی الگوریتم تقریبی با ضریبی ثابت. به طور مثال، همان‌طور که در فصل کارهای پیشین بیان خواهد شد، الگوریتم تقریبی با ضریب تقریب کم‌تر از $2$، برای مسئله‌ی $k$-مرکز وجود ندارد مگر اینکه $P = NP$ باشد. برای مسائل مختلف، معمولا می‌توان کران پایینی برای میز ان تقریب‌پذیری آن‌ها ارائه داد. در واقع هر مسئله‌ی ان‌پی، علاوه بر این الگوریتم کارآمدی برای حل آن وجود ندارد، بعضا الگوریتم تقریبی برای حل آن با ضریبی تقریب کمی نیز وجود ندارد. در جدول \رجوع{جدول:تقریب‌پذیری} از میزان تقریبی مسائل مختلفی که در این پایان‌نامه مورد استفاده قرار می‌گیرد ببینید.

\قسمت{الگوریتم‌های جویبارداده}