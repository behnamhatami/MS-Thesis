
% -------------------------------------------------------
%  Abstract
% -------------------------------------------------------


\pagestyle{empty}

\شروع{وسط‌چین}
\مهم{چکیده}
\پایان{وسط‌چین}
\بدون‌تورفتگی

مسئله‌ی $k$-مرکز به عنوان مسئله‌ای شناخته شده در علوم کامپیوتر مطرح است و در حوزه‌های مختلفی مورد استفاده قرار می‌گیرد. هدف مسئله‌ی $k$-مرکز پیدا کردن کم‌ترین شعاعی است که می‌توان مجموعه‌ی داده شده از نقاط را با $k$ کره با آن شعاع پوشاند. این مسئله در حالت کلی ان‌پی-سخت است. تمرکز اصلی این پایان‌نامه بر روی مسئله‌ی $k$-مرکز در حالت جویبار داده با داده‌ی پرت در ابعاد بالا است. به علت افزایش روز افزون حجم داده‌ها، گونه‌ی جویبار داده‌ی مسئله برای پاسخ‌گویی به حجم وسیع داده‌ها مورد توجه قرار می‌گیرد. از طرفی دیگر، وجود خطا در بین داده‌های تجربی، در نظر گرفتن داده‌های پرت را پر اهمیت می‌کند.

در این پایان‌نامه، دو مسئله‌ی $1$-مرکز و $2$-مرکز در فضای اقلیدسی با داده‌ی پرت در مدل جویبار داده مورد بررسی قرار می‌گیرد. برای مسئله‌ی $1$-مرکز با داده‌ی پرت، در حالتی که تعداد داده‌های پرت ثابت باشد، الگوریتمی با ضریب تقریب $1.7$ ارائه می‌دهیم که الگوریتم قبلی با ضریب تقریب $1.73$ را بهبود می‌بخشد. برای مسئله‌ی $2$-مرکز با داده‌ی پرت، الگوریتمی با ضریب تقریب $1.8 + \epsilon$ ارائه می‌دهیم که نسبت به الگوریتم قبلی با ضریب تقریب $4 + \epsilon$، بهبود قابل توجهی محسوب می‌شود. حافظه‌ی مصرفی و زمان به‌روزرسانی هر دو الگوریتم از مرتبه‌ی چندجمله‌ای نسبت به $d$، $z$ و $\frac{1}{\epsilon}$ است که مستقل از طول جویبار داده است.

\پرش‌بلند
\بدون‌تورفتگی \مهم{کلیدواژه‌ها}: 
خوشه‌بندی، $k$-مرکز، جویبار داده، الگوریتم تقریبی
\صفحه‌جدید
