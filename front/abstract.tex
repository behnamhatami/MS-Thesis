
% -------------------------------------------------------
%  Abstract
% -------------------------------------------------------


\pagestyle{empty}

\شروع{وسط‌چین}
\مهم{چکیده}
\پایان{وسط‌چین}
\بدون‌تورفتگی

مسئله‌ی $k$-مرکز\پاورقی{$k$-Center} به عنوان مسئله‌ای شناخته شده در علوم کامپیوتر مطرح است که در حوزه‌های مختلفی مورد استفاده قرار می‌گیرد.  این مسئله در حالت کلی ان‌پی-سخت\پاورقی{NP-Hard} است. تمرکز اصلی این مقاله برروی مسئله‌ی $k$-مرکز در حالت جویبار داده با داده‌های پرت در ابعاد بالا است. به علت افزایش روز افزون حجم داده‌ها، گونه‌ی جویبار داده‌ی مسائل برای پاسخ‌گویی به حجم وسیع داده‌ها مورد توجه قرار می‌گیرد. از طرفی دیگر، وجود داده‌های اریب در بین داده‌های تجربی، در نظر گرفتن داده‌های پرت را مهم جلوه می‌دهد.

در این پایان‌نامه، دو مسئله‌ی $1$-مرکز و $2$-مرکز در فضای اقلیدسی با $z$ داده‌ی پرت با ورودی جویبار داده مورد بررسی قرار می‌گیرد. در بررسی مسئله‌ی $1$-مرکز با داده‌های پرت مسائل $1$-مرکز بدون داده‌ی پرت و مسئله‌ی $1$-مرکز پوشاننده مورد بررسی قرار می‌گیرد که با استفاده از نتایج ارائه شده برای این مسائل، الگوریتمی با ضریب تقریب $1.7$ برای مسئله‌ی $1$-مرکز با داده‌ی پرت ارائه می‌دهیم. برای مسئله‌ی $2$-مرکز با داده‌های پرت، الگوریتمی با ضریب تقریب $1.8 + \epsilon$ ارائه می‌دهیم که نسبت به الگوریتم قبلی برای $k$ کلی با ضریب تقریب $4 + \epsilon$، بهبود قابل توجهی محسوب می‌شود.

\پرش‌بلند
\بدون‌تورفتگی \مهم{کلیدواژه‌ها}: 
خوشه‌بندی، $k$-مرکز، جویبار داده، الگوریتم تقریبی
\صفحه‌جدید
